\setcounter{page}{1}
\chapter{Introduction}

As the title of this thesis suggests its background is twofold. On one hand it is founded on the theory of diffusion controlled reaction rates, on the other hand it is based on transition rate theory over fluctuating barriers. \\

Diffusion controlled reaction rates have been studied in physics, chemistry and biology since the early 19th century. All latter inquiries on the topic are founded on the pioneering work of Marian von Smoluchowski from 1916 and 1916 who held a series of talks \cite{Smoluchowski1916} describing the motion of Brownian particles in solution and used it to describe the coagulation of gold particles \cite{Smoluchowski1917a}. Thereby he obtain the famous Smoluchowski reaction rate of ideal Brownian particles being absorbed in a spherical sink.\\
Later in the 1940s Debye \cite{Debye1942} extended this basic model to include inter particle interactions when he investigated the rate for diffusion limited reactions between charged particles is while . \\
In the 1880s Szabo et. al. \cite{Szabo1982} further extended the concept to describe gating mechanisms. Therefore, the sink was no longer considered to be ideal but was taken to fluctuate between different states of surface reactivity. \\

The subject of transition rate theory has been studied empirically even since the mid 18th century by Van't Hoff \cite{hoff1884} and Arrhenius \cite{arrhenius1889} but not before 1940 it became a thorough theoretical foundation when Kramers published his celebrated paper on ``Brownian Motion in a Field of Force and the Diffusion Model of Chemical Reactions'' \cite{Kramers1940}.
He described the escape from a metastable state as a noise assisted reaction and derived the well known Kramers reaction rate.\\
Building up on Kramers results Doering and Gadoua \cite{Doering1992} were the first ones to investigate the case when the potential defining the metastable state in an escape problem is not constant but subject to fluctuations. The discovered an effect that they called resonant activation that describes a local minimum in mean first passage times emerging from the interplay of the timescales of barrier crossing and barrier fluctuations. \\

Although it seems to be a reasonable consequence of the present state of research the problem of reaction rates over fluctuating barriers has so far not been addressed. As of now, it will be topic of this thesis. \\
To follow an educative approach the chapter \ref{Short_Introduction_to_Stochastic_Processes} gives a short introduction to stochastic processes as far as it is helpful to supplement first, the following examples from diffusion controlled reaction theory in section \ref{K_s} and \ref{The_Debye_Reaction_Rate} and second, the derivations made in section \ref{Reaction_Rates_over_Fluctuating_Barriers}. \\
Chapter \ref{Reaction_Rates_over_Fluctuating_Barriers} gives an analytical treatment of a system consisting of a spherical sink surrounded by a metastable step shaped potential barrier that is embedded by a bath of Brownian particles and derives an expression for the rate of encounters of these particles with the sink. \\
Chapter \ref{numeric_model} gives a short resume of the numerical methods used in chapter \ref{results} that evaluates a simple example of the system described in chapter \ref{Reaction_Rates_over_Fluctuating_Barriers}. Finally chapter \ref{conclusion} sums up the results and gives an outlook on further work. \\

The keen reader may skip chapter \ref{Short_Introduction_to_Stochastic_Processes} and \ref{numeric_model} and proceed directly with chapter \ref{Reaction_Rates_over_Fluctuating_Barriers} and \ref{results} to consult chapters \ref{Short_Introduction_to_Stochastic_Processes} and \ref{numeric_model} only for reference. \\




 Maybe something more on PNIPA yolk shell nano particles \dots
