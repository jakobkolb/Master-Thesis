
\section{Fundamentals and Methods}
\label{Short_Introduction_to_Stochastic_Processes}
This section will introduce some basic concepts and fundamentals of stochastic processes and reaction rate theory as they concern the problem under study. For a broader context please refer to standard textbooks \cite{VanKampen1992} or review papers on the topic \cite{Calef1983a, Bressloff2013}. The goal here is to give a framework for the treatment of composite Markov processes in discrete and continuous space and to present the reference case for diffusion controlled reactions rates in the Debye-Smoluchowski interpretation \cite{Smoluchowski1917a, Debye1942}. \\
Therefore it will take the well-trodden trail from the Chapman Kolmogorow equation via the Kramers Moyal expansion to the Kolmogorow forward or Fokker-Planck equation. Here it will take a step to the side to calculate the drift and diffusion coefficients from the Langevin equation in the 'overdamped' case of a Brownian particle before we show the derivation of the Master equation again from the Chapman-Kolmogorow equation.\\
Thereafter the methods introduced before will be used to illustrate the treatment of multivariate Markov processes in discrete and continuous space. \\
Finally it gives a rigorous derivation of the diffusion controlled reaction rate from the Fokker-Planck description of a spherical sink embedded in a bath of Brownian particles.
\subsection{The Fokker Planck Equation}
\label{The_Fokker_Planck_Equation}
By definition a stochastic process is said to have the \emph{Markov property} if for any $n$ successive time steps its conditional probability density function is governed by the following relation:
\begin{equation}
    P(x_{n},t_{n}|x_{1},t_{1};\cdots;x_{n-1},t_{n-1}) = P(x_{n},t_{n}|x_{n-1},t_{n-1}), \quad t_{n}>t_{n-1}> \cdots >t_{1},
    \label{}
\end{equation}
i.e. the conditional probability at $t_n$ is only determined by the value of $x_{n-1}$ at $t_{n-1}$ and not influenced by any knowledge of the process at earlier times.\\
Hence, the entire realization of the process is determined by the initial distribution $P(x_1,t_1)$ and the two step transition probability $P(x_{n},t_{n}|x_{n-1},t_{n-1})$ and every multi step probability distribution function can be expressed as a hierarchy of these two. \\
For instance for $ t_n > t_{n-1} > \cdots > t_1$ one has:
\begin{align}
    P(x_1,t_1;x_2,t_2;\cdots;x_n,t_n) &= P(x_n,t_n|x_{n-1},t_{n-1})P(x_{n-1},t_{n-1}|x_{n-2},t_{n-2}) \cdots \nonumber \\
                                      & \cdots P(x_2,t_2|x_1,t_1)P(x_1,t_1)
    \label{hierarchy}
\end{align}
For only three time steps $t_3>t_2>t_1$ the integration of the three step joint probability distribution over the intermediate step leads to:
\begin{equation}
    P(x_3,t_3;x_1,t_1) = P(x_1,t_1)\int P(x_3,t_3|x_2,t_2) P(x_2,t_2|x_1,t_1) {\rm d} x_2,
\end{equation}
and division by $P(x_1,t_1)$ results in the well known \emph{Chapman Kolmogorow} equation:
\begin{equation}
    P(x_3,t_3|x_1,t_1) = \int P(x_3,t_3|x_2,t_2) P(x_2,t_2|x_1,t_1) {\rm d} x_2.
    \label{CKeq}
\end{equation}
An equivalent formulation of the Chapman Kolmogorow equation is the \emph{Kramers Moyal expansion} \cite{Kramers1940, Moyal1949}. To derive it the expression for the transition probabilities \eqref{CKeq} is multiplied with the initial probability distribution $P(x_1,t_1)$ and integrated over $x_1$ which leads to
\begin{equation}
    P(x_3,t_3) = \int P(x_3,t_3|x_2,t_2)P(x_2,t_2) {\rm d} x_2.
    \label{CKeq3}
\end{equation}
The integrand may be written in terms of $\Delta x = x_3 - x_2$ and then be expanded for $\Delta x \ll 1$
\begin{align}
    P(x_3,t_3|x_2,t_2)P(x_2,t_2) &= P( (x_3 - \Delta x) + \Delta x, t_3|x_3 - \Delta x, t_2)P( (x_3 - \Delta x), t_2) \nonumber \\
    &= \sum_{n=0}^{\infty} \frac{(-1)_{n}}{n!}\frac{\partial^{n}}{\partial x_3^{n}}\left\{P(x_3 + \Delta x, t_3|x_3,t_2) P(x_3,t_2)\right\}
\end{align}
This is again plugged into \eqref{CKeq3}. Integration over $\Delta x$ and substitution of $\Delta t = t_3 - t_2$ then yields:
\begin{equation}
    P(x_3,t_2 + \Delta t) = \sum_{n=0}^{\infty} \frac{(-1)_{n}}{n!}\frac{\partial^{n}}{\partial x_3^{n}}\left\{ M_{n}(x_3,t,\Delta t) P(x_3,t_2) \right\}
    \label{KME1}
\end{equation}
where $M_n$ are the so called \textit{jump moments} defined by
\begin{equation}
    M_n(x,t,\Delta t) = \int (\Delta x)^{n} P(x + \Delta x, t + \Delta t | x, t) {\rm d} (\Delta x).
    \label{Jump_moments}
\end{equation}
Note that from normalization of $P(x+\Delta x, t+ \Delta t|x,t)$ it follows that the lowest of the jump moments $ M_0(x,t,\Delta t) $ is equal to one. \\
This formulation still gives describes the time evolution of the probability distribution in terms of discrete time steps. To derive a formulation in a continuous time variable one subtracts the first therm of the sum on the right hand side of equation \eqref{KME1}, divides by $\Delta t$ and takes the limit of $\Delta t \rightarrow 0$ to obtain \\
\begin{equation}
    \frac{\partial P(x,t)}{\partial t} = \sum_{n = 1}^{\infty}\frac{(-1)^{n}}{n!}\frac{\partial^n}{\partial x^n} \left\{ \lim_{\Delta t \rightarrow 0} M_n(x,t,\Delta t) P(x,t) \right\}.
    \label{KME2}
\end{equation}
As we will see later it is also reasonable to assume that for short time differences the jump moments go linear with $\Delta t$:
\begin{equation}
    M_{n}(x,t,\Delta t) \sim \Delta t + \mathcal{O}(\Delta t^{2}).
\end{equation}
Bearing this in mind it makes sense to introduce so called \emph{kinetic coefficients} of the form
\begin{align}
    K^{(n)}(x,t) &= \lim_{\Delta t \rightarrow 0} \frac{1}{ \Delta t } M_n(x,t,\Delta t) \nonumber \\
    &=\lim_{\Delta t \rightarrow 0} \frac{1}{\Delta t} \int (\Delta x)^m P(x+\Delta x,t+\Delta t|x,t) {\rm d}(\Delta x) .
    \label{kinetic_coefficients}
\end{align}
(But note, that this is only a matter of notation and does not require the small $\Delta t$ behaviour of $M_n$ mentioned  before!) \\
Substituting these coefficients back into equation \eqref{KME2} results in the desired formulation of the Kramers Moyal expansion \cite{Moyal1949}:
\begin{equation}
    \frac{\partial P(x,t)}{\partial t} = \sum_{n = 1}^{\infty}\frac{(-1)^{n}}{n!}\frac{\partial^n}{\partial x^n} \left\{ K^{(n)}(x,t) P(x,t) \right\}.
    \label{Kramers Moyal expansion}
\end{equation}

So far nothing has be assumed, other than the Markov property and the existence of the Taylor series. However in many application the examination of the jump moments reveals, that it is a suitable approximation to truncate the expansion for $n>2$. In this case, one obtains the following form, known as the \textit{Fokker Planck equation}:
\begin{equation}
    \frac{\partial P(x,t)}{\partial t} = - \frac{\partial}{\partial x} \left[K^{(1)}(x,t)P(x,t) \right] + \frac{1}{2}\frac{\partial^2}{\partial x^2}\left[ K^{(2)}(x,t)P(x,t) \right] 
    \label{FPE}
\end{equation}
where $K^{(1)}$ and $K^{(2)}$ are independent of $t$ if the process is stationary. \\
\subsection{Brownian Motion}
\label{Brownian_Motion}
Brownian motion is the oldest example of a Markov process that is known in physics \cite{Einstein1905,Smoluchowski1906}. It emerges from a heavy particle in a solution of lighter particles, that collide with each other in a random fashion. Consequently, the velocity of the heavier particle undergoes a series of supposedly uncorrelated jumps. When the velocity $v$ has a certain direction, there will be on average more collisions from this side, than from the other. Therefore the probability of a change in velocity $\Delta v$ depends on its current value, but not on the velocity at earlier times. As a consequence, the velocity of the heavier particle can be considered to be a Markov process. When the whole system is in equilibrium the process is stationary and its autocorrelation time is the time in which an initial velocity of the heavy particle is damped out. \\
Now in the \textit{overdamped limit} the correlation time of the velocity is much smaller then the time between two observations of the heavy particle. In this case the observation of the particle gives a series $x(t_1), x(t_2), \cdots , x(t_n)$ of subsequent particle positions. Each displacement $x(t_{n}) - x(t_{n-1})$ does not depend on the previous history of the process, i.e. it is independent of $x_{n-2}, \cdots , x_{1}$. Hence not only the velocity, but also the position of the particle itself is a Markov process (at least on a coarse grained timescale). 
\par
In the following we will start with the \emph{Langevin equation} \cite{Langevin1908} for the velocity of a particle in a fluid and calculate the corresponding kinetic coefficients to obtain a Fokker-Planck equation for the distribution of its position.\\ 
This Langevin equation is given by:
\begin{equation}
    m \frac{{\rm d}^2 x}{{\rm d}t^2} = -\gamma \frac{ {\rm d}x}{{\rm d}t} + f(x) + \varepsilon(t)
    \label{Langewin equation}
\end{equation}
where $\varepsilon(t)$ is a random process describing the collision interaction of the particle and the solute. From the central limit theorem it follows that its distribution must be Gaussian. Also it is assumed first that the system is in thermal equilibrium, i.e. its velocities are governed by a Boltzmann distribution and therefore for the second moment it holds:
\begin{equation}
    \left< \dot{x}^{2} \right> = \frac{K_B T}{m}
    \label{2nd_moment_of_velocities}
\end{equation}
and second that the spacial variable $x(t)$ and the random force $\varepsilon (t)$ are not correlated. \\
Given these assumptions it can be shown that the autocorrelation of the random force is given by:
\begin{equation}
    \left< \varepsilon(t) \varepsilon(t') \right> = 2 K_B T \gamma \delta(|t-t'|)
    \label{ff_autocorrelation}
\end{equation}
and that the correlation time of the velocities is equal to
\begin{align}
    \left< \dot{x}(t) \dot{x}(t') \right> &= \frac{K_B T}{m} \exp \left\{-\frac{\gamma}{m}|t-t'|\right\} \nonumber \\
    &=\frac{K_B T}{m} \exp\left\{\frac{\tau}{\tau_0} \right\} .
    \label{vv_autocorrelation}
\end{align}
In the so called overdamped limit this auto correlation time $ \tau_0 = \dfrac{m}{\gamma}$ is very small, such that the Langevin equation can be approximated by
\begin{equation}
    \gamma \frac{ {\rm d}x}{{\rm d}t} = f(x) + \varepsilon(t).
    \label{BD1}
\end{equation}
As described in the introductory part of the section this process must be observed on a coarse grained timescale to be considered Markovian. Therefore one integrates equation \eqref{BD1} over one time step $\Delta t$ to describe it in discrete time. Doing so results in:
\begin{equation}
x(t + \Delta t) = x(t) + \frac{1}{\gamma} f(x) \Delta t + \frac{1}{\gamma} \int\limits_{t}^{t+\Delta t} \varepsilon(t) {\rm d} t.
    \label{od1}
\end{equation}
The last term on the right hand side can be expressed in terms of a effective random force of the form:
\begin{equation}
    \varepsilon'(t) \Delta t = \frac{1}{\gamma}\int\limits_{t}^{t + \Delta t} \varepsilon(t) {\rm d} t
    \label{eff_rdf}
\end{equation}
such that equation \eqref{od1} reads:
\begin{equation}
    x(t + \Delta t) = x(t) + \frac{1}{\gamma} f(x) \Delta t + \varepsilon'(t) \Delta t.
    \label{od2}
\end{equation}
This effective random force must again be Gaussian distributed and from equation \eqref{ff_autocorrelation} it follows, that its autocorrelation is given by:
\begin{equation}
    \left< \varepsilon'(t) \varepsilon'(t') \right> = \frac{2 K_B T \gamma}{ \Delta t}
    \label{ff_eff_autocorrelation}
\end{equation}
and its distribution is therefore equal to:
\begin{equation}
    P(\varepsilon ' ) = \sqrt{\frac{\Delta t}{4 \pi D \gamma^{2}}} \exp \left[ - \frac{\varepsilon ^{\prime 2} \Delta t}{4 D \gamma^{2}} \right].
    \label{eps dist}
\end{equation}
where the diffusion constant $D$ is given by the \emph{Einstein-Smoluchowski relation}:
\begin{equation}
    D = \frac{K_B T}{\gamma}
    \label{ESR}
\end{equation}
From the distribution of the random force one can compute the transitions probability $P(x+\Delta x, t+ \Delta t| x, t)$ for the Brownian particle as the estimate over its translocations:
\begin{equation}
    P(x+\Delta x,t+\Delta t|x,t)  = \left< \delta \left(  \Delta x - (x(t-\Delta t) - x(t)) \right)\right>.
\end{equation}
Here we use equation \eqref{od2} and \eqref{eps dist} to write this as
\begin{equation}
     P(x+\Delta x,t+\Delta t|x,t) = \int {\rm d}\varepsilon ' \delta \left(  \Delta x - \left( \frac{1}{\gamma} f(x) \Delta t + \varepsilon'(t) \Delta t \right) \right)  \sqrt{\frac{\Delta t}{4 \pi D \gamma^{2}}} \exp \left[ - \frac{\varepsilon  ^{\prime 2} \Delta t}{4 D \gamma^{2}} \right] \nonumber\\
 \end{equation}
 which finally evaluates to 
 \begin{equation}
      P(x+\Delta x,t+\Delta t|x,t) = \sqrt{\frac{1}{4 \pi D \Delta t}} \exp \left[ \frac{-\left(\Delta x - f(x) \frac{\Delta t}{\gamma} \right)^2}{4 D \Delta t} \right]
    \label{BM_transition_probability}
\end{equation}
Now it is straight forward to calculate the jump moments from this transition probability according to equation \eqref{Jump_moments} and as it was already anticipated it turns out to be true that the first and second moment are linear in $\Delta t$ in leading order: 
\begin{equation}
    M_1(x,t,\Delta t) = f(x)\frac{\Delta t}{\gamma}, \qquad M_2(x,t,\Delta t) = 2 D \Delta t + \left(f(x)\frac{\Delta t}{\gamma} \right)^{2}
    \label{BM_jump_moments}
\end{equation}
such the kinetic coefficients \eqref{kinetic_coefficients} are equal to
\begin{equation}
    K^{(1)}(x,t) = \frac{f(x)}{\gamma}, \qquad K^{(2)}(x,t) = 2 D.
    \label{BD_kinetic_coefficients}
\end{equation}
It can be shown that all higher jump moments are of order $\mathcal{O}(\Delta t ^{2})$ such that it is indeed valid to truncate the Kramers Moyal expansion after the second therm. Therefore the time evolution of the probability distribution of a Brownian particle is fully described by the following expression:
\begin{equation}
    \frac{\partial P(x,t)}{\partial t} = - \frac{\partial}{\partial x} \left[f(x)P(x,t) \right] + D\frac{\partial^2}{\partial x^2}\left[P(x,t) \right] 
    \label{FPE2}
\end{equation}

\subsection{The Master-Equation}
\label{The_Master_Equation}
The master equation is one more equivalent formulation of the Chapman Kolmogorow equation. The Chapman Kolmogorow equation usually is of not much help since it is essentially a property of the solution for the transition probabilities. The master equation however is its formulation in terms of a differential equation and is far more useful especially for the description in a discrete state space. \\
In order to derive it one has to reason first about the short time behaviour of the transition probabilities. From the Chapman Kolmogorow equation for equal time arguments it is obvious that
\begin{equation}
    P(x_2,t|x_1,t) = \delta(x_1-x_2)
    \label{leading_order}
\end{equation}
which is the zero order therm of the following formulation of the short time transition probability $P(x_2,t+\Delta t|x_1,t)$:
\begin{equation}
    P(x_2,t+\Delta t|x_1,t) = W(x_2|x_1)\Delta t + \left[ 1 - \Delta t \int {\rm d} x W(x_2|x_1) \right] \delta(x_2-x_1) + O(\Delta t ^{2})
    \label{master_assumption}
\end{equation}
For the better understanding of this expression imagine the following: At time $t$ the system was in state $x_1$. In the subsequent time interval $\Delta t$ it might have made a transition to the state $x_2$.
Here the probability of the transition is expressed in terms of the (non negative) {\it transition rate} i.e. the transition probability per unit time from state $x_1$ to $x_2$: $W(x_2|x_1)$. So the first term on the right hand side of equation \eqref{master_assumption} gives the transition probability from state $x_1$ to another state $x_2 \ne x_1$ whereas the second therm on the right hand side is equal to one minus the probability to move to any other state i.e. the probability for the system to rest in state $x_1$ during the time $\Delta t$.\\
To maintain a readable form it is common to introduce the notation
\begin{equation}
    T_\tau (x_2|x_1) = P(x_2,t+\tau|x_1,t)
\end{equation}
and to omit the absolute time dependence, since the process is assumed to be stationary. \\
Chapman-Kolmogorow equation in this formulation reads:
\begin{equation}
    T_{\tau + \tau'}(x_3|x_2) = \int T_{\tau'}(x_3|x_2)T_{\tau}(x_2|x_1){\rm d} x_2.
    \label{K2}
\end{equation}
Now the insertion of equation \eqref{master_assumption} on the right hand side leads to:
\begin{equation*}
    T_{\tau+\tau'}(x_3|x_1) = \int \left\{ \left[1 - \tau' \int {\rm d} z W(z|x_3) \right] \delta(x_3 - x_2) + \tau' W(x_3|x_2) \right\} T_{\tau}(x_2|x_1){\rm d} x_2
\end{equation*}
and regrouping the terms and dividing by $\tau ' $ results in:
\begin{align*}
    \frac{1}{\tau'} T_{\tau+\tau'}(x_3|x_1) &= \frac{1}{\tau'}  \int T_{\tau}(x_2|x_1) \delta(x_3 - x_2){\rm d} x_2\\
    &- \int \left\{ W(z|x_2)  T_{\tau}(x_2|x_1)\delta(x_3 - x_2) \right\}{\rm d} z {\rm d} x_2 \\
    &+ \int \left\{ W(x_3|x_2) T_{\tau}(x_2|x_1) \right\}{\rm d} x_2.
\end{align*}
The integrals in the first and the second therm on the right hand side can be evaluated and the fist therm can be moved to the left hand side:
\begin{align*}
    \frac{1}{\tau}\left\{  T_{\tau+\tau'}(x_3|x_1) - T_{\tau}(x_3|x_1)\right\} &= \int \left\{ W(x_3|x_2) T_{\tau}(x_2|x_1) \right\}{\rm d} x_2  \nonumber \\
    &- \int \left\{ W(z|x_3)  T_{\tau}(x_3|x_1) \right\}{\rm d} z.
\end{align*}
Finally one renames $z$ to $x_2$ and takes the limit of $\tau' \rightarrow 0$ to obtain the well known formulation of the master equation in continuous space:
\begin{equation}
    \frac{\partial}{\partial \tau}T_{\tau}(x_3|x_1) = \int \left\{ W(x_3|x_2) T_{\tau}(x_2|x_1) - W(x_2|x_3) T_{\tau}(x_3|x_1) \right\}{\rm d} x_2
    \label{continuous_space_master_equation}
\end{equation}where the $W(x_i|x_j)$ are properties of the specific process.
This equation describes the time development of the transition probabilities given an initial condition $(x_1,t_1)$. A more intuitive form follows from multiplying and integrating over a distribution of initial conditions $P(x_1,t_1)$ and its spatial coordinate $x_1$.
\begin{equation}
    \frac{\partial P(x,t)}{\partial t} = \int \left\{ W(x|x') P(x',t) - W(x'|x)P(x,t) \right\} {\rm } x'
\end{equation}
In this form the meaning becomes particularly clear. The master equation is a \textit{gain loss equaition} for the probabilities of each state $x$. The first term on the right hand side describes the gain of probability of state $x$ due to transitions from other states $x'$, whereas the second term on the right hand side describes the loss of probability of state $x$ due to transitions to other states.\\
For a discrete state space the integral on the right hand side is replaced by the sum over all possible states and the master equation has the form of a system of coupled ordinary differential equations:
\begin{equation}
    \frac{{\rm d} P_n(t)}{{\rm d} t} = \sum_{n'} W_{n n'}P_{n'}(t) - W_{n'n}P_{n}(t).
    \label{discrete_space_master_equation}
\end{equation}
Or in a more compact form with the following \emph{transition rate matrix} $\mathbb{W}$:
\begin{equation}
    \mathbb{W}_{n n'} = W_{n n'} - \delta_{n n'}\sum\limits_{m n} W_{m n}
    \label{transition_rate_matrix}
\end{equation}
resulting in 
\begin{equation}
    \frac{{\rm d} P_n(t)}{{\rm d} t} = \sum_{n'} \mathbb{W}_{n n'}P_{n'}(t).
    \label{ME3}
\end{equation}
The transition rate matrix satisfies the following conditions:
\begin{align*}
    &0 \ge \mathbb{W}_{n,n'} \quad \mbox{ for all } n \ne n' \\
    &0 \ge -\mathbb{W}_{n,n} \ge \infty \\
    &\sum_{n'} \mathbb{W}_{n,n'} = 0
    \label{Transitions_rate_matrix}
\end{align*}
In general it is not symmetric and can thus not be diagonalized. \\
From equation \eqref{discrete_space_master_equation} one immediately sees, that for a steady state solution this implies, that the loss of probability from one state is compensated by the gain of probability by transitions from other states.\\
\begin{equation}
    \sum\limits_{n'} W_{n n'}P_{n'} = \sum\limits_{n'}W_{n' n} P_n
    \label{equilibrium}
\end{equation}
For stationary time reversible Markov processes this criterion can even be tightened to a property called \textit{detailed equilibrium}.
This property requires, that the total exchange of probability between two states to each other must be equal i.e.
\begin{equation}
     W_{n n'}p_{n'} = W_{n'n}p_{n}
    \label{detailed_balance}
\end{equation}
It can be proven to be true for a wide range of physical and chemical processes \cite{Boltzmann1872,Einstein1917,Wegscheider1911} and is also closely related to the Onsager reciprocal relations \cite{Onsager1931,Wigner1954}.
It also implies a certain symmetry of the transition rate matrix that can be used to show that for this class of matrices it is possible to find a symmetric representation such that it can be diagonalized via a suitable orthogonal transformation.
\subsection{Composite Markov Processes}
\label{Multivariate_Markov_Processes}
It is straight forward to continue to Markov processes whose sample space is a direct product of continuous and discrete variables i.e. $\Omega = \mathbb{R}^{3} \times [1,\cdots, N]$. The Chapman Kolmogorow equation then reads
\begin{equation}
    P(\vec{x}_3,n_3,t_3|\vec{x}_1,n_1,t_1) = \sum_{n_2} \int P(\vec{x}_3,n_3,t_1|\vec{x}_2,n_2,t_2)P(\vec{x}_2,n_2,t_2|\vec{x}_1,n_1,t_1) {\rm d} \vec{x}_2.
    \label{MCK}
\end{equation}
In this case the variable $\vec{x}$ can be treated by means of the Kramers Moyal expansion as discussed in section \ref{The_Fokker_Planck_Equation} whereas the variable $n$ can be treated by the approach of the Master equation as described in section \ref{The_Master_Equation}. Assuming that the driving process for the evolution of the continuous variable is Gaussian and that the Kramers Moyal expansion in moments of its transition probabilities can therefore be truncated after the second therm one can derive the following expression:
\begin{align}
    \frac{\partial}{\partial t } P(\vec{x},n,t) =   &- \vec{ \nabla } \left[\vec{K}^{(1)}(\vec{x},n,t)P(\vec{x},n,t) \right] + \frac{1}{2}\vec{\nabla}^{2}\left[ K^{(2)}(\vec{x},n,t)P(\vec{x},n,t) \right] \nonumber \\
                                                    &+ \sum_{n'} \left\{ W_{nn'}P(\vec{x},n',t) - W_{n'n}P(\vec{x},n,t)\right\}.
    \label{composite_mp}
\end{align}
This gives the time evolution of the composite Markov process. If the underlying process for the continuous variable is again Brownian motion in an external potential then it follows from \eqref{BD_kinetic_coefficients} that the previous expression can be refined to
\begin{align}
    \frac{\partial}{\partial t } P(\vec{x},n,t) =   &- \vec{ \nabla } \left[\frac{1}{\gamma}\vec{f}(\vec{x},n,t)P(\vec{x},n,t) \right] +\vec{\nabla}^{2}\left[ D(\vec{x},n)P(\vec{x},n,t) \right] \nonumber \\
                                                    &+ \sum_{n'} \left\{ W_{nn'}P(\vec{x},n',t) - W_{n'n}P(\vec{x},n,t)\right\}
    \label{fpmeq1}
\end{align}
This can be interpreted as the motion of Brownian particles with different internal states such as spatial or electronic confirmations or binding states to some surface. In each of these internal states the may therefore have different friction and diffusion constants, be under the influence of forces and may even be subject to different boundary conditions. \\
It is convenient to write this equation in a vector notation for the variable $n$. Then $\vect{p}$ indicates the vector $(p(0),p(1),\cdots,p(N))$ on the $N$ dimensional space over the closed interval $[0,1]$ (the variables $\vect{x}$ and $t$ have been omitted). \\ 
Also the diagonal Fokker-Planck operator for the drift and diffusion terms is introduced as
\begin{equation}
    \mathbb{F} = {\rm diag} \left[- \vec{ \nabla } \frac{1}{\gamma}\vec{f}(\vec{x},n,t) + \vec{\nabla}^{2} D(\vec{x},n) \right]
    \label{fpo}
\end{equation}
and the transition rates are again written in terms of $\mathbb{W}$ as introduced in equation \eqref{transition_rate_matrix} such that equation \eqref{fpmeq1} has the following more compact form
\begin{equation}
    \frac{\partial}{\partial t} \vect{p}(\vec{x},t) = \left\{ \mathbb{F} + \mathbb{W} \right\} \vect{p}(\vec{x},t).
    \label{fpmeq2}
\end{equation}
This kind of description has been used to explain various transport properties for instance on polymer chains or on reactive surfaces. \cite{Friedman1968,Caceres1990}.
\section{The Smoluchowski Reaction Rate}
For educational reasons and since it will referred to later, at this point the solution to the original Smoluchowski \cite{Smoluchowski1917a} problem will be given.
The problem involves a perfect spherical sink of radius $R_s$ embedded in an initially homogeneous distribution of Brownian particles. The aim is to calculate the time dependent and stationary absorption rate of particles into the sink.
The boundary and initial conditions are therefore
\begin{align}
    \rho(r > R_s, t = 0) &= \rho_o, \\
    \rho(r=R_s,t) &= 0, \\
    \lim_{r \rightarrow \infty} \rho(r, t) &= \rho_o.
    \label{BC}
\end{align}
In the following, the corresponding Fokker Planck Equation in terms of particle densities
\begin{equation}
        \frac{\partial \rho(\vec{r},t)}{\partial t} = - \vec \nabla \left[ \vec f(\vec{r})\rho(\vec{r},t) \right] + D\vec \nabla ^2 \left[\rho(\vec{r},t) \right] 
    \label{FPE3}
\end{equation}
will be solved without external force, i.e. $\vec f(r) = 0$ and subject to the given boundary and initial conditions.
With the substitution $r \cdot \rho(r,t) = u(r,t)$ and the assumption, that the problem is spherically symmetric the derivatives in the Fokker Planck equation simplify to
\begin{equation}
    \frac{\partial u(r,t)}{\partial t} = D \frac{\partial ^2 u(r,t)}{\partial r^2}
    \label{Simplified FPE}
\end{equation}
Laplace transform of the equation yields:
\begin{align}
    \int_0^\infty e^{-st}\frac{\partial u(r,t)}{\partial t} \rm{d} t &= D \frac{\partial ^2 }{\partial r^2} \int_0^\infty e^{-st} u(r,t) \rm{d} t \\
    \left[e^{-st} u(r,t) \right]_0^\infty + s \int_0^\infty e^{-st} u(r,t) \rm{d} t &= D \frac{\partial^2}{\partial r^2} \tilde{u}(r,s)\\
    u(r,0) + s \tilde{u}(r,s) &= D \frac{\rm{d}^2}{\rm{d} r^2} \tilde{u}(r,s).
\end{align}
This is an ordinary 2nd degree inhomogeneous differential equation with constant coefficients.
For the standard ansatz $\tilde{u}(r,s) = \exp(\lambda(s) r)$ for the homogeneous solution we get the following characteristic polynomial:
\begin{equation}
    \lambda(s) ^2 - \frac{s}{D} = 0
    \label{characteristic_polynomial}
\end{equation}
resulting in the following homogeneous solution:
\begin{equation}
    \tilde{u}_h(r,s) = C_1 e^{ - \sqrt{\frac{s}{D}} \cdot r } + C_2 e^{ \sqrt{\frac{s}{D}} \cdot r }
    \label{u_h}
\end{equation}
We find the inhomogeneous solution using a polynomial ansatz of the form $\tilde{u}_i = C_3 r + C_4$ leading to the following relation:
\begin{align}
    s(C_3 r + C_4)  &= -u(r,0)\\
                    &= - r \rho_o \\
    \Rightarrow C_3 &= \frac{r}{s}\rho_o \\
                C_4 &= 0
\end{align}
Now the entire solution has to be fitted to the boundary conditions as in (\ref{BC}). The solution in Laplace space then reads:
\begin{equation}
\tilde{u}(r,s) = \rho_o \left( \frac{r}{s} + \frac{R_s}{s} e^{ \sqrt{\frac{s}{D}}(R_s - r) } \right) 
\end{equation}
The inverse Laplace transform
\begin{align}
    u(r,t)  &= \frac{1}{2 \pi i} \int\limits_{\gamma - i \infty}^{\gamma + i \infty}  e^{st} \tilde{u}(r,s){\rm d}t \\
    &= \frac{\rho_o}{ 2 \pi i} \left\{  \int\limits_{\gamma - i \infty}^{\gamma + i \infty} \frac{r}{s}  {\rm d}t +  \int\limits_{\gamma - i \infty}^{\gamma + i \infty}\frac{R_s}{s} e^{ \sqrt{\frac{s}{D}}(R_s - r) }  {\rm d}t \right\}
    \label{inverse laplace}
\end{align}
is done using the residue theorem for the first integral:
\begin{align}
    \oint_{ \gamma } {\rm d}z f(z) &= 2 \pi i \sum_{k = 1}^{n}I(\gamma, a_k) {\rm Res}(f,a_k) \\
    { \rm Res}(f,y_o) &= \frac{1}{(m-1)!} \lim_{z\rightarrow z_o} \frac{{ \rm d} ^{m-1}}{{\rm d} z^{m-1}} \left[ (z - z_o)^{m}f(z) \right]
    \label{residue theorem}
\end{align}
and the following identity for the second:
\begin{equation}
    \mathcal{L}\left[ {\rm erfc\left( \frac{a}{2\sqrt{t}} \right)} \right] = \frac{1}{s}e^{a\sqrt{s}}
    \label{L(erfc)}
\end{equation}
resulting in the following time dependent solution for $u(r,t)$ resp. the particle density $\rho(r,t)$:
\begin{align}
    u(r,t) &= \rho_o \left\{ r - R_s {\rm erfc} \left( \frac{r - R_s}{\sqrt{4 D t}} \right) \right\} \\
    \rho(r,t) &= \rho_o \left\{ 1 - \frac{R_s}{r} + {\rm erf} \left( \frac{r - R_s}{\sqrt{4Dt}} \right) \right\}.
    \label{u(r,t)}
\end{align}
In the limit $t \rightarrow \infty$ this results in the steady state density profile:
\begin{equation}
    \rho(r) =  \rho_o \left( 1 - \frac{R_s}{r} \right)
    \label{steady_state_density}
\end{equation}
The reaction rate can be defined as the total flux of particles through the boundary $\Omega$ of the sink:
\begin{equation}
    K = \int_\Omega \vec{J} {\rm d}\vec{A} 
    \label{reaction rate}
\end{equation}
Using the differential continuity equation:
\begin{align}
    \frac{\partial \rho(\vec{r},t)}{\partial t}&= \vec{\nabla} \vec{J}(\vec{r},t) \\
    &= \vec{\nabla} \left\{ \rho(\vec{r},t) \nabla \vec{U}(\vec{r}) + D \vec{\nabla} \rho(\vec{r},t) \right\}
    \label{contiuity equation}
\end{align}
and the spherical symmetry of the solution one can derive the time dependent reaction rate of the Brownian particles with the spherical sink of radius $R_s$ as follows:
\begin{align}
    K(t) &= \int_\Omega D  \vec{\nabla} \rho(\vec{r},t) \\
    &= 4 \pi D R_s^2 \left. \vec{\nabla} \rho(\vec{r},t) \right|_{r = R_s}\\
    &= 4 \pi D R_s \rho_o \left( 1 + \frac{R_s}{\sqrt{4Dt}} \right)
    \label{ideal reaction rate}
\end{align}
Again in the limit of $t \rightarrow \infty$ this results in the steady state absorption rate:
\begin{equation}
    K = 4 \pi D R_s \rho_o
    \label{steady state ideal rate}
\end{equation}

\section{The Debye Reaction Rate}
\label{The_Debye_Reaction_Rate}
The problem of diffusion controlled reaction rates as outlined in the previous chapter was extended to interaction between the substrate and the absorbing particles. For the resulting Fokker-Planck Equation of the substrate particles
\begin{equation}
    \frac{\partial \rho(r,t)}{\partial t} = \vec \nabla \left[ \frac{\rho(r,t)\vec \nabla U(r)}{\gamma} + D \nabla \rho(r,t) \right]
    \label{fpe_debye}
\end{equation}
it is possible to obtain a steady state solution for the particle flux through the surface of the sink of radius $R_s$. Therefore one omits the lhs. and integrates the rhs. from $R_s$ to an arbitrary $r$. With $J$ being the particle flux
\begin{equation}
    \vec J(r,t) =  \frac{\rho(r,t)\vec \nabla U(r)}{\gamma} + D \nabla \rho(r,t)
    \label{flux}
\end{equation}
This results in
\begin{align}
    0 &= \int_{R_s}^{r} \vec \nabla \vec J(r) \nonumber \\
    \vec J(R_s) &=  \rho(r)\frac{\vec \nabla U(r)}{\gamma} + D \vec \nabla \rho(r).
\end{align}
Now since the absorption rate is given by the flux through the sink surface i.e. the flux $\vec J (R_s)$ integrated over the surface of the sink the rate $K$ is determined by the following equation:
\begin{equation}
    \frac{K}{4\pi D r^{2}} = \rho(r)\frac{\rm d}{\rm d r} \frac{U(r)}{K_B T} + \frac{\rm d}{\rm d r} \rho(r).
    \label{K}
\end{equation}
Equation \eqref{K} can be solved in terms of a general solution of the homogeneous differential equation
\begin{equation}
    \frac{\rm d}{\rm d r} \rho(r) = -\rho(r) \frac{\rm d}{\rm d r} \frac{U(r)}{K_B T}
    \label{homogeneous_equation}
\end{equation}
that has the form
\begin{equation}
    \rho(r) = C \exp \left[ - \frac{U(r)}{K_B T} \right].
    \label{homogeneous_solution}
\end{equation}
Using this the solution the inhomogeneous equation can be obtained by the method of variation of the constant. Therefore we take the constant $C$ in \eqref{homogeneous_solution} to be $r$ dependant and substitute $\rho(r)$ in eq. \eqref{K}:
\begin{align}
    \frac{K}{4 \pi D r^2} &= C(r) \exp \left[ - \frac{U(r)}{K_B T} \right] \frac{\rm d}{\rm d r} \frac{U(r)}{K_B T} + \frac{\rm d}{\rm d r} C(r) \exp \left[ - \frac{U(r)}{K_B T} \right] \nonumber\\
    \frac{K}{ 4 \pi D r^2} &= \exp \left[ -\frac{U(r)}{K_B T} \right] \frac{\rm d }{\rm d r} C(r)
    \label{equation_C(r)}
\end{align}
Now integrating this equation from $R_s$ to an arbitrary $r>R_s$ yields:
\begin{equation}
    C(r) = C(R_s) + \frac{K}{4 \pi D}\int_{R_s}^{r} \frac{\exp \left[ \frac{U(r')}{K_B T}\right]}{r'^2} \rm d r'
    \label{solution_C(r)}
\end{equation}
Using the boundary condition that $\rho(R_s)=0$  we can set the integration constant $C(R_s) = 0$ to obtain the result for the density profile of the substrate particles in the potential of the absorbing particle:
\begin{equation}
    \rho(r) = \frac{K}{4 \pi D}\exp \left[ -\frac{U(r)}{K_B T} \right] \int_{R_s}^{r} \frac{\exp \left[ \frac{U(r')}{K_B T}\right]}{r'^2} \rm d r'.
    \label{rho_debye}
\end{equation}
The rate $K$ can then be calculated by the use of the boundary condition that for $r \rightarrow \infty$ $\rho(r) \rightarrow \rho_o$ together with the assumption, that the interaction $U(r)$ only has a finite range, i.e. that it vanishes for $r \rightarrow \infty$:
\begin{equation}
    \rho_o = \lim_{r\rightarrow \infty} \rho(r) = \frac{K}{4 \pi D}\int_{R_s}^{\infty} \frac{\exp \left[ \frac{U(r')}{K_B T}\right]}{r'^2} \rm d r'
    \label{lim_rho_debye_infty}
\end{equation}
which is equivalent to 
\begin{equation}
    K = 4 \pi D \rho_o \left\{\int_{R_s}^{\infty} \frac{\exp \left[ \frac{U(r')}{K_B T}\right]}{r'^2} \rm d r' \right\}^{-1}
    \label{K_Debye}
\end{equation}
In the case of $U(r) \equiv 0$ this simplifies to the result obtained by Smoluchowski (comp. eq. \eqref{steady state ideal rate}.

\newpage
\section{Reaction Rates over Fluctuating Barriers}
\label{Reaction_Rates_over_Fluctuating_Barriers}
\subsection{Model Description}
\label{Model_Description}
The system under consideration consists of a spherical sink of radius $R_s$ that is surrounded by a potential barrier. The barrier is assumed to be of boxcar shape with limiting radius $a,b>R_s$. It fluctuates between states $n$ with different hight $U_n \in [U_0, \cdots U_N]$ subject to a transition rate matrix $\mathbb{W}$. The system is embedded in a reservoir of Brownian particles. It is desired to find the rate at which the particles are absorbed by the sink, given that the transition rate matrix of the potential fluctuations satisfies the detailed balance property.\\
The appropriate boundary conditions are, that the probability density function of the Brownian particles is zero at $R_s$ and takes a constant finite value for $|\vec{r}| \rightarrow \infty$. Due to the fact that the system is not spatially bounded it is not possible to normalize the joint pdf $\vect{p}(\vec{r},n,t)$ of the position of the Brownian particles and the state of the potential barrier in the sense that 
\begin{equation}
    \sum_n \int_{\mathbb{R}^{3}} p_n(\vec{r},t) {\rm d}V = 1.
    \label{pdfNormalization}
\end{equation}
Instead it is appropriate to normalize the distribution to the particle density $\vect{\rho}(\vec{r},t)$ as commonly done in statistical physics
\begin{equation}
    \sum_n \int_V \rho_n(\vec{r},t) {\rm d}V = N
    \label{densNormalisation}
\end{equation}
where $N$ is the total number of particles enclosed in the volume $V$. The time evolution of the joint pdf can be described by eq. \eqref{fpmeq2} derived in the previous section
\begin{equation}
    \frac{\partial}{\partial t}\vect{\rho}(\vec{r},n,t) = \left\{ \mathbb{F} + \mathbb{W} \right\} \vect{\rho}(\vec{r},n,t).
    \label{fpmeq3}
\end{equation}
Using the spherical symmetry of the system, the Fokker-Planck operator can be written as
\begin{equation}
    \mathbb{F} = {\rm diag}\left[ \vec{\nabla}\frac{1}{\gamma}\left( \vec{\nabla} U_n(r) \right)+ D \vec{\nabla}^{2} \right].
    \label{fpo2}
\end{equation}
It becomes obvious from this equation, that the state of the potential might as well be seen as a property of the Brownian particles. One could for instance imagine the barrier as a constant electric potential. Then the particles are fluctuating between differently charged states. For the assumption of noninteracting particles to be still valid, the solution has to be dilute and the Debye screening length has to be small. \\
\subsection{Fit Conditions}
\label{Fit_Conditions}
For the boundary at $|\vec{r}| \rightarrow \infty$ far away from the influence of the potential and the sink it is reasonable to assume that the particle density distribution is a stationary solution $\vect{\rho}^{(eq)}$ the sole Master equation. From the assumption of detailed balance it follows that it has to satisfy
\begin{align}
    &\mathbb{W} \vect{\rho}^{(eq)} = 0 \nonumber \\
    &\mathbb{W}_{n'n} \rho_n^{(eq)} = \mathbb{W}_{nn'}\rho_{n'}^{(eq)}.
    \label{detailed_balance2}
\end{align}
It can be shown that this $\vect{\rho}^{(eq)}$ is not degenerate and that all its entries are positive. \\
The next issue to investigate is the behaviour of the steady state solution for the particle density distribution at the jump discontinuities of the potential 
\begin{equation}
  U_n(r) = \left\{ \begin{array}{l l} 
        0 &: R_s < r \le a \\
        U_n &: a<r \le b \\
        0 &: b < r \le R_d
    \end{array} \right.
    \label{step_potential}
\end{equation}
Therefore we first integrate from the boundary of the sink to some arbitrary $r > R_s$
\begin{equation*}
    \int_{R_s}^{r}\mathbb{F}\vect{\rho}(r'){\rm d} r' = - \mathbb{W} \int_{R_s}^{r} \vect{\rho}(r') {\rm d} r' .
\end{equation*}
Note that the lower boundary of the Integral on the lhs is proportional to the particle flux through the sink boundary whereas the upper bound is proportional to the particle flux through the surface of a sphere with radius $r$ at a certain state of the potential. The Integral on the rhs is proportional to the particle flux from and to the this state of the potential in the observed volume due to transitions from and to other states of the potential.\\
    \begin{equation}
        \frac{1}{\gamma}\rho_n(r) \vec{\nabla} U_n(r) + D \vec{\nabla} \rho_n(r) = J_n(R_s) - \left\{ \mathbb{W} \int_{R_s}^{r} \vect{\rho}(r') {\rm d} r' \right\}_{n}
    \end{equation}
Now we integrate over a small vicinity of the jump discontinuity with width $\varepsilon$ and take the limit of $\varepsilon \rightarrow 0$
    \begin{equation}
        \int_{a-\varepsilon}^{a + \varepsilon} \frac{U_n}{\gamma D}\delta(a-r) + \int_{a-\varepsilon}^{a + \varepsilon} \frac{1}{\rho_n(r)}\nabla \rho_n(r) = \underbrace{\int_{a - \varepsilon}^{a + \varepsilon}\frac{J_n(r)}{\rho_n(r) D} - \int_{a - \varepsilon} ^{a+ \varepsilon} \left\{ \mathbb{W} \vect{\rho}(r)\right\}_{n}}_\text{$O(\varepsilon)$}
    \end{equation}
Since the terms on the rhs. of the equation scale only with $\varepsilon$ we end up with a result previously known for systems in thermal equilibrium:
\begin{equation}
    \vect{\rho}^{(I)}(a) = \underbrace{ {\rm diag}\left[\exp\left\{\frac{U_n}{K_B T} \right\}\right]}_\text{\large$\mathbb{U}$} \vect{\rho}^{(II)}(a)
    \label{dens_fit}
\end{equation}
Here $\vect{\rho}^{(I)}$ and $\vect{\rho}^{(II)}$ refers to the particle density at different side of the jump discontinuity.
Analogously we find relations for the derivatives of the particle density at the jump discontinuity:
\begin{equation}
    \vec{\nabla}\vect{\rho}^{(I)}(a) = \vec{\nabla}\vect{\rho}^{(II)}(a)
    \label{ddens_fit}
\end{equation}
\subsection{Expansion in Eigenfunctions of $\mathbb{W}$}
\label{Expansion_in_Eigenfunctions}
The assumption of the detailed balance property that implies the existence of an equilibrium distribution $\vect{\rho}^{(eq)}$ allows for the definition of an operator $\mathbb{T}$
\begin{equation}
    \mathbb{T} = \delta_{n,n'} [\rho_n^{(eq)}]^{\frac{1}{2}}.
    \label{symmetrisation_transform}
\end{equation}
This is a similarity transform that symmetrises $\mathbb{W}$
\begin{equation}
    \mathbb{T}^{-1}\mathbb{W}\mathbb{T} = \mathbb{S}.
    \label{symm_rate_matrix}
\end{equation}
Using property \eqref{detailed_balance2} it follows that
\begin{align}
    \mathbb{S}_{il} &= \mathbb{T}^{-1}_{ij} \mathbb{W}_{jk} \mathbb{T}_{kl} = \sum_j \delta_{ij} [\rho^{(eq)}_i]^{-\frac{1}{2}} \mathbb{W}_{jk} \mathbb{T}_{kl} \\ \nonumber
    &= [\rho^{(eq)}_{i}]^{\frac{1}{2}} \sum_{k} \mathbb{W}_{ik} \delta_{kl} [\rho^{(eq)}_l]^{-\frac{1}{2}} = \mathbb{W}_{il}^{\frac{1}{2}} \left( \mathbb{W}_{il} \frac{\rho^{(eq)}_i}{\rho^{(eq)}_l} \right)^{\frac{1}{2}} \\ \nonumber
    &= \left(\mathbb{W}_{il} \mathbb{W}_{li}\right)^{\frac{1}{2}} \\ \nonumber
    S_{ii} &= W_{ii}.
\end{align}
The resulting matrix can then be diagonalized by an orthogonal matrix $\mathbb{D}$
\begin{equation}
    \mathbb{D}^{\dagger} \mathbb{S} \mathbb{D} = -{\rm diag}\left[ \lambda_i \right].
    \label{orthogonal_transform}
\end{equation}
It can be shown that $\lambda_i > 0$ for $i>1$ and $\lambda_1 = 0$ with the corresponding eigenvector
\begin{equation}
    D_{i1} = \rho^{(eq)}(i)^{\frac{1}{2}}.
\end{equation}
We now write equation \eqref{fpmeq3} in terms of eigenfunctions of $\mathbb{W}$
\begin{equation}
    \tilde{\vect{\rho}}(\vec{r},t) = \mathbb{A}^{-1} \vect{\rho}(\vec{r},t)
    \label{eigenfunctions}
\end{equation}
where $\mathbb{A}$ denotes the transformation $\mathbb{T}\mathbb{D}$ and use the fact that the potential term vanishes everywhere but at the jump discontinuities. Therefore equation \eqref{fpmeq3} reads
\begin{equation}
    \frac{\partial }{\partial t} \tilde{\vect{\rho}}^{(j)}(\vec{r},t) = {\rm diag} \left[D \vec{\nabla}^{2} - \lambda_i  \right] \tilde{\vect{\rho}}^{(j)}(\vec{r},t)
    \label{fpmeq4}
\end{equation}
For the steady state case where the lhs. vanishes it it straight forward to write down the solution to this equation
\begin{align}
    \tilde{\rho}_{1}^{(j)}(r) &= c_{1,1}^{(j)} + c_{1,2}^{(j)} \frac{1}{r} \nonumber \\
    \tilde{\rho}_{i \ne 1}^{(j)}(r) &= c_{i,1}^{j}\frac{1}{r} \exp\left[-r\sqrt{\frac{\lambda_i}{D}}\right] + c_{i,2}^{j}\frac{1}{r} \exp\left[r\sqrt{\frac{\lambda_i}{D}}\right] 
    \label{fp_ind_sol}
\end{align}
Note that the solution corresponding to the first eigenvalue $\lambda_1 = 0$ equals the one derived in \eqref{steady_state_density} for the ungated problem. Together with the fitting conditions obtained in \eqref{dens_fit} they would result in the solution for a constant boxcar shaped potential barrier. \\
The other solutions corresponding to the nonzero eigenvalues of the transition rate matrix $\lambda_i<0$ describe deviations from this solution due to the metastability of the potential barrier. Note that they exponentially decay in space with a \textit{decay length}
\begin{equation}
    r_d^{(i)} = \sqrt{\frac{\lambda_i}{D}}
    \label{decay_length}
\end{equation}
that is unique for each state of the potential and which is closely related to its decay rate in time.
\subsection{Treatment of Boundary and Fit Conditions}
\label{Treatment_of_Boundary_and_Fit_Conditions}
Now we have to find an expression, that allows for the calculation of the coefficients $c_{i,k}^{(j)}$ from the transformed boundary and fit conditions
\begin{align}
    \mathbb{A}^{-1}\vect{\rho}^{(I)}(R_s) &= \vect{\tilde{\rho}}^{(I)}(R_s) = 0 \nonumber \\
    \vect{\tilde{\rho}}(r \rightarrow \infty) &= \mathbb{A}^{-1} \vect{\rho}^{(eq)} = (1,0,\cdots,0)^{T}
\end{align}
and
\begin{align}
    \vect{\tilde{\rho}}^{(I)}(a) &= \mathbb{A}^{-1}\mathbb{U}\mathbb{A} \vect{\tilde{\rho}}^{(II)}(a), \\ \nonumber
    \vect{\tilde{\rho} '}^{(I)}(a) &= \vect{\tilde{\rho} '}^{(II)}(a), \\ \nonumber
    \vect{\tilde{\rho}}^{(III)}(b) &= \mathbb{A}^{-1}\mathbb{U}\mathbb{A} \vect{\tilde{\rho}}^{(II)}(b), \\ \nonumber
    \vect{\tilde{\rho} '}^{(III)}(b) &= \vect{\tilde{\rho} '}^{(II)}(b).
\end{align}
Therefore it is useful to write the solution of equation \eqref{fpmeq4} as
\begin{equation}
    \tilde{\vect{\rho}}^{(j)} = \underbrace{ \left( \begin{array}{cllllllll}
       1   & \frac{1}{r}   & 0                 & 0                 & 0              & 0             & 0 & \cdots &\\
       0   & 0             &\frac{1}{r} e^{-r \alpha_2}   &\frac{1}{r} e^{r \alpha_2 }   & 0              & 0             & 0 & \cdots &\\
       0   & 0             & 0                 & 0                 &\frac{1}{r} e^{-r\alpha_3} &\frac{1}{r} e^{r\alpha_3} & 0 & \cdots &\\
       \vdots  &&&&&&&\ddots &\\
       \vdots  &&&&&&&&\ddots
   \end{array} \right)}_\text{\large$\hat{\rho}(r)$}
   \underbrace{\left(\begin{array}{c}  
       c_{1,1}^{(j)} \\ 
       c_{1,2}^{(j)} \\ 
       c_{2,1}^{(j)} \\ 
       c_{2,2}^{(j)}  \\ 
       \vdots 
   \end{array} \right)}_\text{\large$\mathbb{C}$}.
\end{equation}
Here $\alpha_i = (\lambda_i / D)^{1/2}$. Using this notation, the boundary and fit conditions can be but in one $6 N$ dimensional system of linear equations
\begin{equation}
    \left( \begin{array}{lll}
        \hat{\rho}(R_s) & 0 & 0 \\
        \hat{\rho}(a)   & -\tilde{\mathbb{U}}\hat{\rho}(a) & 0 \\
        \hat{\rho}'(a) & -\hat{\rho}'(a) & 0 \\
        0 &  -\tilde{\mathbb{U}}\hat{\rho}(b) & \hat{\rho}(b) \\
        0 &  -\hat{\rho}'(a) & - \hat{\rho}'(a) \\
        0 & 0 & \hat{\rho}(r\rightarrow \infty)
    \end{array}\right) \left( \begin{array}{c} c_{1,1}^{1} \\ \vdots \\ \vdots \\ \vdots \\ c_{N,2}^{3} \end{array} \right) = 
    \left( \begin{array}{c} 0 \\ \vdots \\ 0 \\ 1 \\ 0 \\ \vdots \end{array} \right) \begin{array}{c} \vdots \\ \vdots \\ \vdots \\ i = 5N+1 \\ \vdots \\ \vdots \end{array}
    \label{lgs}
\end{equation}
where dashes denote derivatives with respect to $r$. Solving this system of linear equations yields the somehow lengthy expressions for the coefficients $c_{i,k}^{(j)}$. \\ We obtain the actual density profile by applying the transformation $\mathbb{A}$ to return to the actual particle densities for the different states of the potential barrier.
\subsection{Calculation of Rates}
\label{Calculation_of_Rates}
The rate of the particles absorbed by the sink is calculated by taking the surface integral over the flux through the surface of the sphere with radius $R_s$
\begin{align}
    K   &= \int_{\partial \Omega_{R_s}} \vec{J} {\rm d} \vec{A}\nonumber\\
    &= \int_{\partial \Omega_{R_s}} D \vec{\nabla} \sum_{n=1}^{N} \rho_n^{(1)}(r)\nonumber \\
    &= 4 \pi D R_s^{2} \sum_{n=1}^{N} \left\{ \mathbb{A} \left. \frac{\partial}{ \partial r}\right|_{R_s} \tilde{\vect{\rho}} \right\}_n
    \label{Rate}
\end{align}
        So in the previous section 
\begin{itemize}
    \item We found fitting conditions for steady state density profiles at jump discontinuities of potential barriers,
    \item we propose a treatment for the fluctuations of the potential barrier in terms of eigenfunctions of its transition rate matrix assuming it satisfies a detailed balance property,
    \item we introduced a persistence length of the influence of the potential fluctuations on the particle density profile,
    \item and we provide a scheme for the calculation of the integration constants of the particle density functions.
\end{itemize}
Now that we have found a general way do derive an analytic solution we will investigate a simple example to reveal the qualitative implications of the previous work. 
\newpage

