\section{Analytic Considerations}

This part will introduce the basic equations that are relevant for the handling of Brownian motion as a stochastic process (and its realisations in a computational model) as well as the derivation of an equivalent Focker-Planck equation to derive an analytic expression for the wanted density profiles and reaction rates.

\subsection{The Focker Planck Equation for Brownian Particles}
Brownian motion is a markovian process, i.e. each time step in the random motion of particles does only depend on their preceding position. This implies, that the conditional distribution of their coordinates obeys the following relation:
\begin{equation}
    P(x,t|y,u;y,v) = P(x,t|y,v), \quad t>u>v
    \label{}
\end{equation}
This relation implies, that for a Markov process every multi step probability distribution can be expressed as a hierarchy of a initial distribution and the two step transition probabilities. For $ t_1 < t_2 < \cdots < t_n$:
\begin{align}
    P(x_1,t_1;x_2,t_2;\cdots;x_n,t_n) &= P(x_n,t_n|x_{n-1},t_{n-1})P(x_{n-1},t_{n-1}|x_{n-2},t_{n-2}) \cdots \nonumber \\
                                      & \cdots P(x_2,t_2|x_1,t_1)P(x_1,t_1)
    \label{hierarchy}
\end{align}
So the entire realization of the process is determined by the initial distribution and the two step transition probability. \\
Integrating the three step joint probability distribution over the intermediate step leads to the Chapman Kolmogorov equation:
\begin{equation}
    P(x,t|y,v) = \int P(x,t|z,u) P(z,u|y,v).
    \label{Chapman Kolmogorov equation}
\end{equation}
From this one can derive the Kramers Moyal expansion for $P(x,t)$:
\begin{equation}
\frac{\partial P(x,t)}{\partial t} = \sum_{m = 1}^{\infty}\frac{(-1)^{m}}{m!}\frac{\partial^m}{\partial x^m} \left[ a^{(m)}(x,t) P(x,t) \right]
    \label{Kramers Moyal expansion}
\end{equation}
with the {\it jump moments} of the transition probability $W(x,\Delta x,t, \Delta t) = P(x+\Delta x,t+\Delta t | x, t)$:
\begin{equation}
    a^{(m)}(x,t) = \int {\rm d} W(x,r,t, \Delta t) r^m .
    \label{jump moments}
\end{equation}
If the expansion is truncated after the second term, the result gives the well known Focker Planck Equation:
\begin{equation}
    \frac{\partial P(x,t)}{\partial t} = - \frac{\partial}{\partial x} \left[a^{(1)}P(x,t) \right] + \frac{\partial^2}{\partial x^2}\left[ a^{(2)}P(x,t) \right] 
    \label{FPE}
\end{equation}
These {\it jump moments} can be calculated from the Langewin equation, describing the 
Brownian motion of a Particle in solution:
\begin{equation}
    m \frac{{\rm d}^2 x}{{\rm d}t^2} = -\gamma \frac{ {\rm d}x}{{\rm d}t} + f(x) + \varepsilon(t)
    \label{Langewin equation}
\end{equation}
in which $\varepsilon(t)$ is a Gaussian distributed random process describing the 
collision interaction of the particle and the solute. 
In the overdamped limit this expression can be discretized in time and transforms to:
\begin{equation}
        x(t + \Delta t) = x(t) + \frac{1}{\gamma}f(x,t) \Delta t + \frac{1}{\gamma} \varepsilon'(t) \Delta t.
    \label{overdamped limit}
\end{equation}
From the distribution of the random force:
\begin{equation}
    P(\varepsilon ' ) = \sqrt{\frac{\Delta t}{4 \pi D \gamma^{2}}} \exp \left\{ \frac{\gamma^{2} \Delta t}{4 D \gamma^{2}} \right\}
    \label{eps dist}
\end{equation}
one can compute the transitions probability for the Brownian particle as:
\begin{align}
    W(x,\Delta x,t, \Delta t)  &= \left< \delta \left(  \Delta x - (x(t-\Delta t) - x(t)) \right)\right> \\
                        &= \int \rm{d}\varepsilon ' \delta \left(  \Delta x - (x(t-\Delta t) - x(t)) \right)  \sqrt{\frac{\Delta t}{4 \pi D \gamma^{2}}} \exp \left[ \frac{\gamma^{2} \Delta t}{4 D \gamma^{2}} \right] \\
                        &= \sqrt{\frac{1}{4 \pi D \Delta t}} \exp \left[ \frac{-\left(\Delta x - f(x) \frac{\Delta t}{\gamma} \right)^2}{4 D \Delta t} \right]
    \label{W}
\end{align}
For this Gaussian transition probability the coefficients of the Kramers Moyal Expansion vanish after the second term, such that the resulting Focker Planck equation holds the full analytic solution for the time evolution of the distribution of particles.
\begin{equation}
    \frac{\partial P(x,t)}{\partial t} = - \frac{\partial}{\partial x} \left[f(x)P(x,t) \right] + D\frac{\partial^2}{\partial x^2}\left[P(x,t) \right] 
    \label{FPE2}
\end{equation}

\subsection{Brownian Particles diffusing around a spherical Sink}

The problem that shall be approached implies a spherical sink of radius $R_s$ at the origin, that absorbes every particle, that crosses its boarder. Further the particle density at infinity is assumed to be constant, isotrop and homogeneous. The density for $t_o = 0$ is constant for all $r > R_s$. 
This leads to the following conditions:
\begin{align}
    \rho(r > R_s, t = 0) &= \rho_o, \\
    \rho(r=R_s,t) &= 0, \\
    \lim_{r \rightarrow \infty} \rho(r, t) &= \rho_o.
    \label{BC}
\end{align}
In the following, the given Focker Planck Equation in terms of particle densities:

\begin{equation}
        \frac{\partial \rho(x,t)}{\partial t} = - \vec \nabla \left[ \vec f(x)\rho(x,t) \right] + D\vec \nabla ^2 \left[\rho(x,t) \right] 
    \label{FPE3}
\end{equation}

will be solved without external force, i.e. $\vec f(r) = 0$ and subject to the given boundary and initial conditions.
With the substitution $r \cdot \rho(r,t) = u(r,t)$ and the assumption, that the problem is spherically symmetric the derivatives in the Focker Planck equation simplify to
\begin{equation}
    \frac{\partial u(r,t)}{\partial t} = D \frac{\partial ^2 u(r,t)}{\partial r^2}
    \label{Simplified FPE}
\end{equation}
Laplace transform of the equation yields:
\begin{align}
    \int_0^\infty e^{-st}\frac{\partial u(r,t)}{\partial t} \rm{d} t &= D \frac{\partial ^2 }{\partial r^2} \int_0^\infty e^{-st} u(r,t) \rm{d} t \\
    \left[e^{-st} u(r,t) \right]_0^\infty + s \int_0^\infty e^{-st} u(r,t) \rm{d} t &= D \frac{\partial^2}{\partial r^2} \tilde{u}(r,s)\\
    u(r,0) + s \tilde{u}(r,s) &= D \frac{\rm{d}^2}{\rm{d} r^2} \tilde{u}(r,s).
\end{align}
This is an ordinary 2nd degree inhomogeneous differential equation with constant coefficients.
For the standard ansatz $\tilde{u}(r,s) = \exp(\lambda(s) r)$ for the homogeneous solution we get the following characteristic polynomial:
\begin{equation}
    \lambda(s) ^2 - \frac{s}{D} = 0
    \label{}
\end{equation}
resulting in the following homogeneous solution:
\begin{equation}
    \tilde{u}_h(r,s) = C_1 e^{ - \sqrt{\frac{s}{D}} \cdot r } + C_2 e^{ \sqrt{\frac{s}{D}} \cdot r }
    \label{u_h}
\end{equation}
We find the inhomogeneous solution using a polynomial ansatz of the form $\tilde{u}_i = C_3 r + C_4$ leading to the following relation:
\begin{align}
    s(C_3 r + C_4)  &= -u(r,0)\\
                    &= - r \rho_o \\
    \Rightarrow C_3 &= \frac{r}{s}\rho_o \\
                C_4 &= 0
\end{align}
Now the entire solution has to be fitted to the boundary conditions as in (\ref{BC}). The solution in Laplace space then reads:
\begin{equation}
\tilde{u}(r,s) = \rho_o \left( \frac{r}{s} + \frac{R_s}{s} e^{ \sqrt{\frac{s}{D}}(R_s - r) } \right) 
\end{equation}
The inverse Laplace transform
\begin{align}
    u(r,t)  &= \frac{1}{2 \pi i} \int\limits_{\gamma - i \infty}^{\gamma + i \infty}  e^{st} \tilde{u}(r,s){\rm d}t \\
    &= \frac{\rho_o}{ 2 \pi i} \left\{  \int\limits_{\gamma - i \infty}^{\gamma + i \infty} \frac{r}{s}  {\rm d}t +  \int\limits_{\gamma - i \infty}^{\gamma + i \infty}\frac{R_s}{s} e^{ \sqrt{\frac{s}{D}}(R_s - r) }  {\rm d}t \right\}
    \label{inverse laplace}
\end{align}
is done using the residue theorem for the first integral:
\begin{align}
    \oint_{ \gamma } {\rm d}z f(z) &= 2 \pi i \sum_{k = 1}^{n}I(\gamma, a_k) {\rm Res}(f,a_k) \\
    { \rm Res}(f,y_o) &= \frac{1}{(m-1)!} \lim_{z\rightarrow z_o} \frac{{ \rm d} ^{m-1}}{{\rm d} z^{m-1}} \left[ (z - z_o)^{m}f(z) \right]
    \label{residue theorem}
\end{align}
and the following identity for the second:
\begin{equation}
    \mathcal{L}\left[ {\rm erfc\left( \frac{a}{2\sqrt{t}} \right)} \right] = \frac{1}{s}e^{a\sqrt{s}}
    \label{L(erfc)}
\end{equation}
resulting in the following time dependent solution for $u(r,t)$ resp. the particle density $\rho(r,t)$:
\begin{align}
    u(r,t) &= \rho_o \left\{ r - R_s {\rm erfc} \left( \frac{r - R_s}{\sqrt{4 D t}} \right) \right\} \\
    \rho(r,t) &= \rho_o \left\{ 1 - \frac{R_s}{r} + {\rm erf} \left( \frac{r - R_s}{\sqrt{4Dt}} \right) \right\}.
    \label{u(r,t)}
\end{align}
In the limit $t \leftarrow \infty$ this results in the steady state density profile:
\begin{equation}
    \rho(r) =  \rho_o \left( 1 - \frac{R_s}{r} \right)
    \label{steady state density}
\end{equation}
The reaction rate can be defined as the total flux of particles through the boundary $\Omega$ of the sink:
\begin{equation}
    K = \int_\Omega \vec{J} {\rm d}\vec{A} 
    \label{reaction rate}
\end{equation}
Using the differential continuity equation:
\begin{align}
    \frac{\partial \rho(\vec{r},t)}{\partial t}&= \vec{\nabla} \vec{J}(\vec{r},t) \\
    &= \vec{\nabla} \left\{ \rho(\vec{r},t) \nabla \vec{U}(\vec{r}) + D \vec{\nabla} \rho(\vec{r},t) \right\}
    \label{contiuity equation}
\end{align}
and the spherical symmetry of the solution one can derive the time dependent reaction rate of the Brownian particles with the spherical sink of radius $R_s$ as follows:
\begin{align}
    K(t) &= \int_\Omega D  \vec{\nabla} \rho(\vec{r},t) \\
    &= 4 \pi D R_s^2 \left. \vec{\nabla} \rho(\vec{r},t) \right|_{r = R_s}\\
    &= 4 \pi D R_s \rho_o \left( 1 + \frac{R_s}{\sqrt{4Dt}} \right)
    \label{ideal reaction rate}
\end{align}
Again in the limit of $t \rightarrow \infty$ this results in the steady state absorption rate:
\begin{equation}
    K = 4 \pi D R_s \rho_o
    \label{steady state ideal rate}
\end{equation}

\subsection{Steady State solution for Ideal Sink with Potential Barrier}

The next section approaches the problem of Brownian particles diffusing around an ideal sink with a spherically symmetric potential barrier. The barrier is assumably smooth i.e. continuous and continuously differentiable. The boundary conditions are as in \eqref{BC}. The Focker Planck Equation thus is the following:
\begin{equation}
    0=\vec{\nabla}\left( \frac{1}{\gamma} \rho(\vec{r}) \vec{\nabla} U(\vec{r}) + D \vec{\nabla} \rho(\vec{r}) \right)
\end{equation}
where $\gamma$ is the friction constant. The solution for this problem was originally derived by Debye in 1949.
From the spherical symmetry of the system, the Einstein Smoluchowski relation for friction and diffusion constant and the Gauss integration theorem follows:
\begin{equation}
    \frac{K}{4 \pi D r^{2}} = \rho(r)\frac{{\rm d}}{{\rm d}r} \frac{U(r)}{K_B T} + \frac{{\rm d}}{{\rm d}r}\rho(r) 
\end{equation}
where $K$ denotes the total flux through the sink boundary that is by definition equal to the desired rate constant.
The homogeneous Solution can be found by an exponential ansatz:
\begin{equation}
    \rho_h(r) = C e^{-\frac{U(r)}{K_B T}}.
\end{equation}
The particular solution can be found via variation of the integration constant:
\begin{equation}
    C(r) = C(R_s) + \frac{K}{4 \pi D} \int_{R_s}^{r} \exp \left( \frac{U(r')}{K_B T} \right)\frac{r'^{2}} {{\rm d}}r.
\end{equation}
To guarantee $\rho(R_s) = 0$, $C(R_s)$ has to be zero and for the solution holds:
\begin{equation}
    \rho(r) = \frac{K}{4 \pi D}\exp\left( -\frac{U(r)}{K_B T} \right) \int_{R_s}^{r} \exp\left( \frac{U(r')}{K_B T} \right)\frac{  {\rm d} r' }{r'^{2}}
    \label{Debye Solution}
\end{equation}
The rate constant $K$ is obtained by normalization of the solution:
\begin{equation}
    K = 4 \pi D \left\{  \int_{R_s}^{r} \exp\left( \frac{U(r')}{K_B T} \right)\frac{  {\rm d} r' }{r'^{2}} \right\}^{-1}
\end{equation}

\subsection{Debye Solution for Boxcar Potential}

The Debye solution for a spherical sink with a potential barrier only holds for smooth potential functions. In case of an unsteady potential the solution has to be obtained differently. The following section will consider the special case of a boxcar like potential barrier.
\begin{equation}
    U(r) = \left\{ \begin{array}{l l} 
        0 &: R_s < r \le a \\
        U_o &: a<r \le b \\
        0 &: b < r \le R_d
    \end{array} \right.
    \label{Boxcar_Potential}
\end{equation}
To find a solution of the Focker Planck equation for this Potential, we integrate over a small area containing the saltus of the Potential. 
\begin{align}
    0 &= \vec{\nabla} \left\{ \frac{1}{\gamma} \rho(\vec{r}) \vec{\nabla} U(\vec{r}) + D \vec{\nabla} \rho(\vec{r}) \right\} \\
    J(r) &=  \frac{1}{\gamma} \rho(\vec{r}) \vec{\nabla} U(\vec{r}) + D \vec{\nabla} \rho(\vec{r}) \\
    \int_{a - \varepsilon}^{a + \varepsilon} {\rm d} r \frac{J(r)}{D\rho(r)} &= \int_{a - \varepsilon}^{a + \varepsilon} {\rm d} r \delta(r - a) \frac{U_o}{K_B T} + \frac{1}{\rho(r)} \nabla \rho(r) \\
    -\frac{U_o}{K_B T} &= \lim_{\varepsilon \rightarrow 0} \left[ \ln (\rho(r) ) \right]_{a - \varepsilon}^{a + \varepsilon} - \lim_{\varepsilon \rightarrow 0}\int_{a - \varepsilon}^{a + \varepsilon} {\rm d} r \frac{J(r)}{D\rho(r)} \\
    \rho_{+}(a) &= \rho_{-}(a) e^{\frac{U_o}{K_B T}}
    \label{Boxcar_BC}
\end{align}

This way we obtain boundary conditions to fit the known solution for the force free Focker Planck equation \eqref{steady state density} at the jump discontinuity.\\
We solve the emerging system of linear equations and obtain the following solution for steady state density profile and absorption rate:
\begin{align}
    \rho(r) &= \rho_0 \cdot \left\{ \begin{array}{ l l}
                                     \left(1 - \frac{R_s}{r} \right) & for \quad R_s < r \le a \\
                                     \left( 1 - \frac{R_s}{a} \right) e^{-\frac{U_o}{K_B T}} + \frac{R_s}{a} - \frac{R_s}{r} & for \quad a < r \le b \\
                                     R_s \left(e^{\frac{U_0}{K_B T}} - 1\right) \left( \frac{1}{a} - \frac{1}{b} \right) + 1 - \frac{R_s}{r}  & for \quad b < r 
                            \end{array} \right. \\
    K &= 4 \pi R_s D \rho_0
    \label{Boxcar_solution}
\end{align}
\subsection{Treatment of fluctuating Boxcar Potential Barrier}

The following section will treat the analytic solution of a boxcar potential barrier with fluctuating hight. The equations for the system are Focker Planck equations that are coupled by the transition rate matrix for the different states of the barrier.
\begin{equation}
    \frac{\partial \vect{\rho}}{\partial t} = \nabla \vect{\rho} \nabla \hat{U} f(\vec{r}) + D\nabla^2\vect{\rho} + \hat{W} \vect{\rho} 
    \label{Rection_Diffusion}
\end{equation}
The Matrix $\hat{U}$ has diagonal elements $U_{ii}$ equal to the different hight of the potential barrier in different states and is zero elsewhere. $f(\vec{r})$ describes the radial shape of the potential. For the rate matrix holds: $W_{ii} = -\sum_{i \ne j} W_{ij}$. The Matrix $\hat{W}$ is generally not symmetric. But one can show that it always has an eigenvector satisfying $\hat{W} \vect{\rho}_{eq} = 0$ and that for this eigenvector the following relation, also known as {\it detailed balance} is valid:
\begin{equation}
    W_{ij} \vect{\rho}^{(eq)}(i) = W_{ji} \vect{\rho}^{(eq)}(j).
    \label{detailled_ballance}
\end{equation}
Therefore the similarity transformation
\begin{equation}
    \hat{T}^{-1}\hat{W}\hat{T} = \hat{S}
\end{equation}
where
\begin{equation}
    T_{ij} =\delta_{ij} \vect{\rho}^{(eq)}(i)^{-\frac{1}{2}}
    \label{similarity_transform}
\end{equation}
symmetrizes $\hat{W}$ such that
\begin{align}
    S_{il} &= T^{-1}_{ij} W_{jk} T_{kl} \\ \nonumber
    &= \sum_j \delta_{ij} \rho^{(eq)}(i)^{\frac{1}{2}} W_{jk} T_{kl} \\ \nonumber
    &= \rho^{(eq)}(i)^{\frac{1}{2}} \sum_{k} W_{ik} \delta_{kl} \rho^{(eq)}(l)^{-\frac{1}{2}} \\ \nonumber
    &= W_{il}^{\frac{1}{2}} \left( W_{il} \frac{\rho^{(eq)}(i)}{\rho^{(eq)}(l)} \right)^{\frac{1}{2}} \\ \nonumber
    &= \left(W_{il} W_{li}\right)^{\frac{1}{2}} \\ \nonumber
    S_{ii} &= W_{ii} 
\end{align}
The resulting matrix can then be diagonalized by an orthogonal matrix $\hat{D}$:
\begin{equation}
    \hat{D}^{\dagger} \hat{S} \hat{D} = -\hat{d}.
    \label{orthogonal_transform}
\end{equation}
It can be shown that $d_i \le 0$ and $d_1 = 0$ with the corresponding eigenvector:
\begin{equation}
    D_{i1} = \rho^{(eq)}(i)^{\frac{1}{2}}.
\end{equation}
Now we will treat the potential term in \eqref{Rection_Diffusion}. Since $f(r)$ is constant except for the $r = a, r = b$ where it jumps from 0 to 1 and from 1 to 0 respectively, the barrier can be treated analogously to \eqref{Boxcar_BC}. Integration over an infinitesimally small interval containing the jump discontinuity of the potential results in fitting conditions for each component of  $\vect{\rho}$:
\begin{align}
    \vect{\rho}^{(I)}(a) &= \mathrm{diag}[\exp(\frac{U_i}{K_B T})] \vect{\rho}^{(II)}(a), \nonumber \\ \nonumber
    \vect{\rho '}^{(I)}(a) &= \vect{\rho '}^{(II)}(a), \\ \nonumber
    \vect{\rho}^{(III)}(b) &= \mathrm{diag}[\exp(\frac{U_i}{K_B T})] \vect{\rho}^{(II)}(b), \\ 
    \vect{\rho '}^{(III)}(b) &= \vect{\rho '}^{(II)}(b).
    \label{fitting_conditions}
\end{align}
other boundary conditions are:
\begin{align}
    \vect{\rho}^{(I)}(R_s) &= 0, \nonumber \\ 
    \vect{\rho}^{(III)}(r \rightarrow \infty) &= \vect{\rho}^{(eq)}.
    \label{fp_bc}
\end{align}
Here I, II and III denote the different regions for I: $ R_s < r \le a$, II: $a < r \le b$ and III: $ b < r$.
Now, to find a solution for the density profile, eq. \eqref{Rection_Diffusion} is transformed into new, independent coordinates via the transformations \eqref{similarity_transform} and \eqref{orthogonal_transform}. Also the drift term from the potential is spared, since it is treated via fitting conditions only. For the steady state case this results in
\begin{equation}
    0 = D \nabla^{2} \vect{\tilde{\rho}} - \hat{d} \vect{\tilde{\rho}}
    \label{independent_rd}
\end{equation}
The solution then reads:
\begin{align}
    \tilde{\rho}_{1}^{(j)}(r) &= c_{1,1}^{(j)} + c_{1,2}^{(j)} \frac{1}{r} \nonumber \\
    \tilde{\rho}_{i \ne 1}^{(j)}(r) &= c_{i,1}^{j} \exp\left[-r\sqrt{\frac{d_i}{D}}\right] + c_{i,2}^{j} \exp\left[r\sqrt{\frac{d_i}{D}}\right] 
    \label{fp_ind_sol}
\end{align}
Now an efficient formulation for the calculation of the coefficients $c_{i,k}^{(j)}$ must be found. Therefore the fitting and boundary conditions from \eqref{fitting_conditions} and \eqref{fp_bc} are transformed to the new coordinates:
\begin{align}
    \hat{A}^{-1}\vect{\rho}^{(I)}(R_s) &= \vect{\tilde{\rho}}^{(I)}(R_s) = 0 \nonumber \\
    \vect{\tilde{\rho}}(r \rightarrow \infty) &= \hat{A}^{-1} \vect{\rho}^{(eq)} = (1,0,\cdots,0)
\end{align}
and
\begin{align}
    \vect{\tilde{\rho}}^{(I)}(a) &= \hat{A}^{-1}\mathrm{diag}[\exp(\frac{U_i}{K_B T})]\hat{A} \vect{\tilde{\rho}}^{(II)}(a), \\ \nonumber
    \vect{\tilde{\rho} '}^{(I)}(a) &= \vect{\tilde{\rho} '}^{(II)}(a), \\ \nonumber
    \vect{\tilde{\rho}}^{(III)}(b) &= \hat{A}^{-1}\mathrm{diag}[\exp(\frac{U_i}{K_B T})]\hat{A} \vect{\tilde{\rho}}^{(II)}(b), \\ \nonumber
    \vect{\tilde{\rho} '}^{(III)}(b) &= \vect{\tilde{\rho} '}^{(II)}(b)
\end{align}
where $\hat{A} = \hat{D}\hat{T}$.
The resulting system of linear equations is then used to determine the coefficients $c_{i,k}^{j}$. The absorption rate can still be calculated as
\begin{equation}
    K = 4 \pi R_s^2 \left. \frac{\partial}{\partial r} \right|_{R_s} \sum_{i} \rho^{I}_{i}
\end{equation}
where $\rho^{I}_{i}$ are the entries of the reverse transformed of $\vect{\tilde{\rho}}$ i.e. $\vect{\rho} = \hat{A} \vect{\tilde{\rho}} = \hat{D}\hat{T} \vect{\tilde{\rho}}$.
