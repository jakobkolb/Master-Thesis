\chapter{Numeric Model}
\label{numeric_model}
The analytic methods that were derived so far only hold for step shaped potentials. Since step potentials are only an approximation to what typically appears in nature it is necessary do be able to deal with more realistic smooth potentials. This can be done conveniently using numeric simulations. Therefore this section will introduce two possible approaches to the composite Markov process of Brownian particles in the vicinity of a fluctuating external potential.
\section{Brownian Dynamics}
\label{BDsim}
The term \textit{Brownian Dynamics Simulation} refers to the integration of the overdamped Langewin equation \eqref{BD1} i.e. the motion of particles with respect to a Gaussian random force. Therefore the equation for the diffusive coordinate $r_n$ of the Brownian particles is discretized in time, such that the actual discrete equation of Motion has the Form
\begin{equation}
    \vec r_m(t + \Delta t) = \vec r_m(t) - \frac{\vec \nabla U_m(\vec r)}{\gamma}\Delta t + \sqrt{2 D \Delta t} R(t)
    \label{discrete_eqm}
\end{equation}
where $R(t)$ is a Gaussian random process with zero average and $\sigma = 1$. $n$ denotes the reactive coordinate of the Brownian particle. This coordinate is updated in each time step with respect to the appropriate Master equation. This is done via a probabilistic scheme. For each time step we draw a random number from an uniform distribution on the interval $[0,1]$ and if the particle is in state $j$ we check which of the bis on the bins on 
\begin{equation}
    [W_{1,j}, \cdots , W_{j-1,j},1 - \sum_{i \ne j} W_{i,j}, W_{j+1,j}, \cdots , W_{M,j}]
    \label{num_meq}
\end{equation}
it hits. The particle does not change its state, if the random number is in bin number $j$ and changes its state to $m = k$ if the random number hits bin number $k$.\\
This is probably the most trivial numerical solution to the given problem, but since the substrate particles do not interact it is still sufficient to evaluate the situation.

\subsection{Boundary Conditions}
Since we are looking for a steady state solution we must make sure that the number of particles is conserved. Therefore the flux of particles out of the simulation domain through the surface of the sink must equal the flux of particles into the simulation domain through its outer boundary.
Also, it proves appropriate to use spherical simulation domain of Radius $R_b$ with the sink of radius $R_s$ in its center to preserve the spherical symmetry of the anticipated solution.
Keeping these necessities in mind the boundary conditions at the sink surface and at the outer domain boundary are implemented as follows: \\
\begin{itemize}
    \item particles are reflected at the outer simulation boundary:
        \begin{lstlisting}
        r = SQRT(DOT_PRODUCT(par(:),par(:)))
        IF(r>Rd) THEN
            rnew = 2*Rd - r
        ENDIF
        \end{lstlisting}
        where par(:) are the particles x,y and z coordinate, r is therefore the particles radial position after an integration step of the integration of the eqm.
    \item If the trajectory that connects the position of a particles before (A) and after (B) an integration step crosses the boundary of the sink the particle is set to the outer boundary of the simulation domain:
        \begin{lstlisting}
        A   = parold(:)
        B   = parnew(:)
        AB  = parold(:) - parnew(:)

        px = A + DOT_PRODUCT(A,AB)*AB/DOT_PRODUCT(AB,AB)

        IF( DOT_PRODUCT( px-A),(px-B)) < 0 )THEN
            r = SQRT(DOT_PRODUCT(px,px))
        ELSEIF( DOT_PRODUCT( px-A),(px-B)) >= 0 )THEN
            r = SQRT(DOT_PRODUCT(B,B))
        ENDIF
        
        IF(r<Rs) THEN
            rnew = Rs + Rd - r
        ENDIF
        \end{lstlisting}
        This fragment of code calculates the closest point px to the sink center on the line containing A and B. Then it checks, if this point px is between A and B. Based on this, it updates the radial position of the particle to set it to the boundary of the simulation domain, if it crossed the boundary of the sink.
\end{itemize}
\subsection{Fit Conditions}
The potential $U_m(r)$ is set to be a modified Gaussian:
\begin{align}
    U_m(r) &= U_m \cdot \exp \left[-\left( \frac{r-\alpha}{\beta} \right)^{2n}\right] \nonumber \\
    \alpha &= a + \frac{b-a}{2} \nonumber \\
    \beta  &= \frac{b-a}{2}
    \label{mod_gauss}
\end{align}
This is a regular Gaussian bell for $n=1$ and converges to a step potential for $n\rightarrow \infty$. \\
Since for large $n$ the shape of the potential can not be resolved by the trajectories of the Brownian particles (unless the time step is set very small) it is necessary to return to the fit conditions for a step potential as derived earlier.
The fit conditions at the jump discontinuities of the potential barrier as given in \eqref{dens_fit} are implemented as follows. If a particle in state $m$ crosses the boundary of the potential barrier from a lower to a higher level, it has a probability $P_r = P_{\Delta U_m}$ to be reflected and a probability $P_p = 1 - P_{\Delta U_m}$ to pass where $P_{\Delta U_m}$ is given by an Arrhenius factor
\begin{equation}
    P_{ \Delta U_m} = \exp \left[\frac{\Delta U_m}{K_B T}  \right].
    \label{arrhenius_factor}
\end{equation}
If a particle crosses the boundary of the potential barrier from a higher to a lower level it does so with probability $P_p = 1$.
\subsection{Density Profile}
To calculate the radial density profile of the substrate particles we bin their radii at each time step and normalize the resulting histogram to its volume per bin. This is averaged over each time step after a certain equilibration time of the simulation.
\subsection{Absorption Rate}
To calculate the absorption rate, we simply count the number of particles that crosses the sink surface and is set to the simulation boundary. This number is then normalized to the time per time step and averaged over each time step after a certain equilibration time of the simulation.
\section{Method of Lines}
\label{method_of_lines}
The method of lines \cite{pregla1989, saucez2001} is a technique for solving partial differential equations that are well posed as an initial value problem in at least one dimension. It uses spatial discretization of the derivatives in all but other dimensions and then treats the resulting semi discrete problem as a system of coupled ordinary differential equations. This has the advantage, that it is possible to use highly optimized methods that have been developed for numeric integration of ordinary differential equations for the treatment of partial differential equations.\\
Since the time dependent description of the system: 
\begin{align}
    \frac{\partial}{\partial t } \rho_n(r,t) =   &- \vec{ \nabla } \left[\frac{1}{\gamma}\vec{f}(\vec{x},n,t)\rho_n(r,t) \right] +\vec{\nabla}^{2}\left[ D\rho_n(r,t) \right] \nonumber \\
    &+ \sum_{n'} \left\{ W_{nn'}\rho_{n'}(r,t) - W_{n'n}\rho_n(r,t)\right\}.
    \label{fpmeqmol}
\end{align}
together with arbitrarily chosen initial conditions:
\begin{equation}
    \rho_n(r,t_0) = \rho_n^{(0)}(r)
    \label{rho0mol}
\end{equation}
and the boundary conditions \eqref{bcrs} and \eqref{bcinf} does fulfill the requirement of being a initial value problem in one dimension (the time dimension in this case) it is possible to treat it using the method of lines to obtain a solution for the density profiles.\\
The resulting reaction rates can then be calculated using equation \eqref{Rate}.\\
Since the solution of this type of Fokker-Planck system is unique \cite{soize1994} and a global attractor \cite{Efendiev2000} the choice of the initial condition does only influence the time that the system needs to converge to the steady state.\\
For technical details of the implementation of the method please refer to the documentation of {\tt Mathematica} 9\textsuperscript{\textregistered}. The options used with {\tt NDsolve} to obtain the results presented in this thesis are the following:\par
{\tt    MaxSteps $\rightarrow$ Infinity,  \\
        MaxStepFraction $\rightarrow$ 0.002, \\
        AccuracyGoal $\rightarrow$ 15,  \\
        StartingStepSize $\rightarrow$ 0.001, \\
        WorkingPrecision $\rightarrow$ MachinePrecision, \\
        Method $\rightarrow$ \{``MethodOfLines'', ``SpatialDiscretization'' $\rightarrow$ \{``TensorProductGrid'', ``MinPoints'' $\rightarrow$ 10000\}\}}

\section{Comparison of Models}
\comment{You suggested to put some numeric results here, but I think this will screw the structure of the thesis, since the features of the solution are explainend subsequentially in section \ref{results} and putting them here without detailed explanation would be one big spoiler imho. But I will definately put some in the appendix comparing Analytic results with MOL with BD for the examples given in figure \ref{rsd} and \ref{asd}}
The two computational models described in this chapter both target the same problem but with a fundamentally different approach. Brownian Dynamics simulations integrate the underlying stochastic differential equations whereas the Method of Lines solves the equivalent partial differential equations. Therefore the Brownian Dynamics approach allows for the calculation of microscopic quantities such as locally resolved mean square displacement of particles whilst the Method of Lines approach only gives access to macroscopic quantities such as fluxes and density profiles. The advantage of the Method of Lines lies in its efficiency. With Brownian dynamics Simulations the errors in all calculated quantities scale with $N^{-1/2}$ with $N$ being the number of simulated particles. If the particles do not interact, the time $T$ for the simulation scales linear with $N$ and therefore the precision of the results will also go with $T^{-1/2}$. Although the documentation of Mathematica does not tell much about the actual routines in use for the implementation of the Method of lines they are obviously a lot more efficient that this when it comes to derive solutions up to a required precision of accuracy. \\
For this reason the method of lines will be used to calculate numeric results for macroscopic quantities and Brownian Dynamic simulations will only be used if it is necessary to explicitly calculate microscopic quantities.

