
\chapter{Analytic Evaluation of the Model for Step Barriers}
\label{Reaction_Rates_over_Fluctuating_Barriers}
This chapter focusses on the analytical treatment of the reaction rate of Brownian particles over a fluctuating step shaped potential barrier with a spherical sink. The reaction rates are calculated from the density profiles of Brownian particles that move subject to the fluctuating barrier. The derivation of the solution for the particle densities is done in three steps.
\begin{itemize}
    \item In the first step one calculates the boundary and fit conditions of the density profiles at the sink surface, at infinity and at the boundaries of the potential barrier \ref{Fit_Conditions}. These calculations are an original result of this thesis. 
    \item In the second step one derives a general solution for the density profiles between the jump discontinuities of the fluctuating potential \ref{Expansion_in_Eigenfunctions}. This derivation was guided by methods previously used in literature for i.e. the treatment of stochastically gated reactions \cite{Szabo1982}. 
    \item In the third step the general solution for the density profiles between the jump discontinuities of the barrier are combined via the previously derived fit conditions. This again is an original result of this thesis.
\end{itemize}
Similar to the previous numeric approach the starting point for the following analytic considerations is the description of the system in terms of a composite Markov process as outlined in section \ref{Multivariate_Markov_Processes} in terms of a continuous diffusion coordinate and a discrete coordinate for the barrier fluctuations.\\
Due to the fact that the system is not spatially bounded it is not possible to normalize the joint PDF $p_m(\vec{r},t)$ of the position of the Brownian particles and the state of the potential barrier in the sense that 
\begin{equation}
    \sum_{m=0} \int_{\mathbb{R}^{3}} p_m(\vec{r},t) {\rm d}V = 1.
    \label{pdfNormalization}
\end{equation}
Instead it is appropriate to normalize the distribution to the particle density $\rho_m(\vec{r},t)$ as commonly done in statistical physics
\begin{equation}
    \sum\limits_{m=0}^{M} \int_V \rho_m(\vec{r},t) {\rm d}V = N
    \label{densNormalization}
\end{equation}
where $N$ is the total number of particles enclosed in the volume $V$. The time evolution of the joint PDF can be described by equation \eqref{fpmeq2} derived in the previous section
\begin{equation}
    \frac{\partial}{\partial t}\vect{\rho}(\vec{r},t) = \left\{ \mathbb{F} + \mathbb{W} \right\} \vect{\rho}(\vec{r},t).
    \label{fpmeq3}
\end{equation}
where $\vect{\rho}(\vec{r},t)$ denotes the vector of $\rho_m(\vec{r},t)$ for all $m \in [0,N]$. \\
Using the spherical symmetry of the system, the Fokker-Planck operator can be written as
\begin{equation}
    \mathbb{F} = {\rm diag}\left[ \vec{\nabla}\frac{1}{\gamma}\left( \vec{\nabla} U_m(r) \right)+ D \vec{\nabla}^{2} \right].
    \label{fpo2}
\end{equation}
It becomes obvious from this equation, that the state of the potential might as well be seen as a property of the Brownian particles. Therefore the index $m$ will further be denoted as the state of the particles. One could for instance imagine the barrier as a constant electric potential as it is done in the derivation of the Debye reaction rate \cite{Debye1942}. Then the particles are fluctuating between differently charged states. For the assumption of noninteracting particles to be still valid, the solution has to be dilute and the Debye screening length has to be small. \\
\section{Boundary and Fit Conditions}
\label{Fit_Conditions}
Since the boundary of the sink is absorbing the density there vanishes:
\begin{equation}
    \vect{\rho}(R_s) = 0.
    \label{bcrs}
\end{equation}
For the boundary at $|\vec{r}| \rightarrow \infty$ far away from the influence of the potential and the sink it is reasonable to assume that the particle density distribution is a stationary solution $\vect{\rho}^{(eq)}$ to equation \eqref{fpmeq3} without the Fokker-Planck term:
\begin{equation}
    \mathbb{W} \vect{\rho}^{(eq)} = 0 \nonumber
\end{equation}
such that
\begin{equation}
    \lim_{r \rightarrow \infty}\vect{\rho}(r) = \vect{\rho}^{(eq)}.
    \label{bcinf}
\end{equation}
From the assumption of detailed balance it follows that it has to satisfy
\begin{align}
    &\mathbb{W}_{m'm} \rho_m^{(eq)} = \mathbb{W}_{mm'}\rho_{m'}^{(eq)}.
    \label{detailed_balance2}
\end{align}
It can be shown that this $\vect{\rho}^{(eq)}$ is not degenerate and that all its entries are positive. For a thorough proof see for instance \cite{VanKampen1992} or \cite{Oppenheim1977}. \\
The next issue to investigate is the behavior of the steady state solution for the particle density distribution at the jump discontinuities of the potential 
\begin{equation}
  U_m(r) = \left\{ \begin{array}{l l} 
        0 &: R_s < r \le a \\
        U_m &: a<r \le b \\
        0 &: b < r \le R_d.
    \end{array} \right.
    \label{step_potential}
\end{equation}
Therefore one first integrates equation \eqref{fpmeq3} over a closed volume bounded by the surfaces of the sink and a sphere of radius $r>R_s$:
\begin{equation*}
    \int_{R_s}^{r} \frac{\partial \vect{\rho}(r,t)}{\partial t}{\rm d}V =  \int_{R_s}^{r}\mathbb{F}\vect{\rho}(r',t){\rm d} V + \int_{R_s}^{r} \mathbb{W}\vect{\rho}(r',t) {\rm d} V .
\end{equation*}
The integrand of the first term in the right hand side can be written in terms of the particle flux in each state $\vect{j}(r,t) = (\vec{j}_{0}(r), \cdots , \vec{j}_{M}(r))$. Also, one can use Gauss' integral theorem to transform the volume into a surface integral. For the second term on the right hand side one can use the linearity of the integral to obtain the following expression:
\begin{equation}
    \int_{R_s}^{r} \frac{\partial \vect{\rho}(r',t)}{\partial t} {\rm d} V = \int_{R_s}^{r} \vect{j}(r') {\rm d} \vec{A} + \mathbb{W} \int_{R_s}^{r} \vect{\rho}(r') {\rm d} V.
    \label{ce0}
\end{equation}
Next, the integrals on the left hand side and in the second term on the right hand side can be evaluated. According to the normalization \eqref{densNormalization} one thereby obtains a vector $\vect{N}(r,t)$ whose entries $N_m(r,t)$ are given by the number of particles in state $m$ that are enclosed in the volume bounded by the radii $R_s$ and $r$:
\begin{equation}
    \frac{\partial}{\partial t} \vect{N}(r,t) = \int_{R_s}^{r} \vect{j}(r',t) {\rm d} \vec{A} + \mathbb{W} \vect{N}(r,t).
    \label{integral_ce}
\end{equation}
This is an integral form of the continuity equation. The change in the particle numbers in each state inside the volume under consideration is equal to the spatial flux through its boundaries and the reaction flux from and to other states. \\
In steady state this particle number has to be constant like every other macroscopic variable of the system. Therefore the left hand side of equation \eqref{ce0} vanishes. One can further use the spherical symmetry of the problem to write the expression for each state separately as
\begin{equation}
    \frac{1}{\gamma} \rho_m(r) \frac{\partial}{\partial r} U_m(r) + D\frac{\partial}{\partial r} \rho_m(r) =\frac{1}{4 \pi r^2} \left\{ 4 \pi R_s^2 |\vec{j}_m(R_s)| - \sum_{m'=0}^{M} \mathbb{W}_{mm'} N_{m'}(r) \right\}.
    \label{ce1}
\end{equation}
Without loss of generality one now considers the inner boundary of the potential such that the derivative of the potential $\frac{\partial}{\partial r}U_m(r)$ is equal to a delta function $\frac{\partial}{\partial r}U_m(r) = U_m \delta (r-a)$. In addition, the particle number $N_m$ can be written in its integral representation again and the particle flux through the sink surface can be expressed as the reaction rate of this particle species $4 \pi R_s^{2} |\vec{j}_m(R_s)| = K_m$ such that the former expression is equivalent to
\begin{equation}
    \frac{U_m}{\gamma D} \delta (r-a) + \frac{1}{\rho_m(r)}\frac{\partial}{\partial r} \rho_m(r) = \frac{K_m}{4 \pi D r^2 \rho_m(r)} - \sum_{m'=0}^{M} \frac{\mathbb{W}_{mm'}}{D r^{2} \rho_{m}(r)}\int_{R_s}^{r} \rho_{m'}(r')r'^2{\rm d} r'.
    \label{ce2}
\end{equation}
Now one integrates over a small vicinity of the jump discontinuity with width $2 \varepsilon$ and uses the Einstein Smoluchowski relation to write the product of friction and diffusion constant as $\gamma D = K_B T$
\begin{align}
    \int_{a-\varepsilon}^{a + \varepsilon} \frac{U_m}{K_B T}\delta(r-a) {\rm d}r &+ \int_{a-\varepsilon}^{a + \varepsilon} \frac{1}{\rho_m(r)}\frac{\partial}{\partial r} \rho_m(r){\rm d} r = \nonumber \\
    & \underbrace{\int_{a-\varepsilon}^{a+\varepsilon}\frac{K_m}{r^2 \rho_m(r)}{\rm d}r}_{\textit{\normalsize I}_1} - \underbrace{\int_{a-\varepsilon}^{a+\varepsilon}\sum_{m'=0}^{M} \frac{\mathbb{W}_{mm'}}{Dr^2 \rho_m(r)}\int_{R_s}^{r}\rho_{m'}(r')r'^2{\rm d}r'{\rm d}r}_{\textit{\normalsize I}_2}.
    \label{ce3}
    \end{align}
    The aim is now to evaluate this expression on the limit of $\varepsilon \rightarrow 0$ to obtain a relation for the behavior of the particle density at the jump discontinuity. Since the particle density is not necessarily a steady function at that point it is useful to denote it with $\rho^{(I)}_m(r)$ for $r<a$ and with $\rho^{(II)}_m(r)$ for $r>a$. \\
While the left hand side of this expression can be easily evaluated as
\begin{equation}
    \frac{U_m}{K_B T} + \ln \left\{\frac{\rho^{(II)}_m(a+\varepsilon)}{\rho^{(I)}_m(r-\varepsilon)}\right\}
    \label{ce4}
\end{equation}
the right hand side needs a closer examination. First consider the integral $I_1$. The integration domain can be split into two parts
\begin{equation}
    I_1 = \int_{a-\varepsilon}^{a}\frac{K_m}{r^2 \rho_m(r)}{\rm d}r + \int_{a}^{a+\varepsilon}\frac{K_m}{r^2 \rho_m(r)}{\rm d}r
    \label{ce5}
\end{equation}
such that it is possible to use the mean value theorem of integration to express the integrals in terms of $\rho^{(I)}$ and $\rho^{(II)}$. 
One finds a $\xi \in [a-\varepsilon,a]$ and a $\xi' \in [a, a+\varepsilon]$ such that \eqref{ce5} is equal to
\begin{equation}
    I_1 = K_m \varepsilon \left\{ \frac{1}{\xi^{2} \rho^{(I)}_{m}(\xi)} + \frac{1}{\xi'^{2}\rho^{(II)}_m(\xi')} \right\}
    \label{ce6}
\end{equation}
Since the particle density is positive everywhere except at the absorbing boundary of the sink (due to the assumed ergodicity of the system) this expression evaluates to zero in the limit of $\varepsilon \rightarrow 0$. \\
The integral $I_2$ can be written as
\begin{equation}
    I_2 = \sum_{m'}^M\frac{\mathbb{W}_{mm'}}{D} \int_{a-\varepsilon}^{a+\varepsilon} {\rm d} r \int_{R_s}^{r} {\rm d} r' \frac{r'^2 \rho_{m'}(r')}{r^2\rho_m(r)}
    \label{ce7}
\end{equation}
and like before the integration domain can be split such that the integrands can be expressed in ether $\rho^{(I)}$ or $\rho^{(II)}$:
\begin{align}
    I_2 = \sum_{m'}^M\frac{\mathbb{W}_{mm'}}{D} & \left\{ \int_{a-\varepsilon}^{a} {\rm d} r \int_{R_s}^{r} {\rm d} r' \frac{r'^2 \rho^{(I)}_{m'}(r')}{r^2\rho^{(I)}_m(r)} \right. \nonumber \\ 
    &\left. + \int_{a}^{a+\varepsilon} {\rm d} r \int_{R_s}^{a} {\rm d} r' \frac{r'^2 \rho^{(I)}_{m'}(r')}{r^2\rho^{(II)}_m(r)} + \int_{a}^{a+\varepsilon} {\rm d} r \int_{a}^{r} {\rm d} r' \frac{r'^2 \rho^{(II)}_{m'}(r')}{r^2\rho^{(II)}_m(r)} \right\}
    \label{ce8}
\end{align}
Note that $r$ and $r'$ are positive and the particle density only vanishes for $r=R_s$. Therefore it is obvious from this representation of $I_2$ that the integrands are well behaved such that the expression scales with the size of the integration domain which is essentially linear in $\varepsilon$. Consequently, in the limit of $\varepsilon \rightarrow 0$ the integral $I_2$ evaluates to zero.
\par
To sum up: this calculation showed that in the limit of an infinitely small integration domain only the left hand side of equation \eqref{ce3} remains. This remaining expression as evaluated in equation \eqref{ce4} can finally be combined for all particle species:
\begin{equation}
    \vect{\rho}^{(I)}(a) = \underbrace{ {\rm diag}\left[\exp\left\{\frac{U_n}{K_B T} \right\}\right]}_\text{\large$\mathbb{U}_a$} \vect{\rho}^{(II)}(a).
    \label{dens_fit}
\end{equation}
Analogous considerations lead to an expression for the boundary of the barrier at $r=b$:
\begin{equation}
     \vect{\rho}^{(II)}(b) = \underbrace{ {\rm diag}\left[\exp\left\{-\frac{U_n}{K_B T} \right\}\right]}_\text{\large$\mathbb{U}_b$} \vect{\rho}^{(III)}(b).
    \label{dens_fit_b}
\end{equation}
A similar result is well known for exclusively diffusive systems in thermal equilibrium but in the case under consideration it also holds for a reaction diffusion system in a steady state. \\
Fit condition for the first derivative of the particle densities can easily be deduced from equation \eqref{ce0}. Therefore the integration domain is again set to be of width $2 \varepsilon$ and symmetric around the jump discontinuity of the potential:
\begin{equation}
    0 = \int_{a-\varepsilon}^{a+\varepsilon} \vect{j}(r') {\rm d} \vec{A} + \mathbb{W} \int_{a-\varepsilon}^{a+\varepsilon} \vect{\rho}(r') {\rm d} V.
    \label{ce9}
\end{equation}
This can be evaluated for each state $m$ separately by splitting the integration domain and using the mean value theorem of integration for the second integral. The second integral can be evaluated trivially due to the spherical symmetry of the integrand to obtain:
\begin{equation}
    (a+\varepsilon)^2 |\vec{j}^{(II)}_m(a+\varepsilon)| - (a+\varepsilon)^2 |\vec{j}^{(I)}_m(a+\varepsilon)| = - \varepsilon \sum_{m'}^{M} \mathbb{W}_{mm'} \left( \rho^{(II)}_{m'}(\xi') + \rho^{(I)}_{m'}(\xi) \right)
    \label{ce10}
\end{equation}
with $\xi \in [a-\varepsilon,a]$ and $\xi' \in [a,a+\varepsilon]$. In the limit of $\varepsilon \rightarrow 0$ the right hand side vanishes and one ends up with 
\begin{equation}
    |\vec{j}^{(I)}_m(a)| =  |\vec{j}^{(II)}_m(a)|
    \label{ce11}
\end{equation}
which can again be written for all particle species as
\begin{equation}
    \vec{\nabla}\vect{\rho}^{(I)}(a) = \vec{\nabla}\vect{\rho}^{(II)}(a).
    \label{ddens_fit}
\end{equation}
As before, the same reasoning leads to an expression for the outer boundary of the potential barrier at $r=b$:
\begin{equation}
    \vec{\nabla}\vect{\rho}^{(II)}(b) = \vec{\nabla}\vect{\rho}^{(III)}(b).
    \label{ddens_fit_b}
\end{equation}
The previous calculations have shown that in steady state the density of Brownian particles at a jump discontinuity of their driving potential shows the same behavior as it would in thermal equilibrium. \\
There is one more thing to add concerning boundary conditions in this system. In many applications the ``sink'', as it is called here, is not perfect, in other words particles have to overcome a certain activation energy $U_r$ to react with it. This means that particles at the sink surface have a certain probability to react with the sink that is given by an Arrhenius factor:
\begin{equation}
    P_r = \exp\left[- \frac{U_r}{K_B T} \right]
    \label{reaction_arrhenius_factor}
\end{equation}
In the case under study this activation energy is taken to be equal for all particles, but it is common to take it as a fluctuating property of the particles to describe gating functionalities of the sink. \cite{Szabo1982} \\
As a result the substrate reacts with the sink with a finite rate, proportional to the concentration of particle at the sink surface.
\begin{equation}
    K_m = P_r \cdot \rho_m(R_s)
    \label{sink_reaction_rate}
\end{equation}
It is obvious from this equation that the particle density at the sink boundary is not zero anymore as in equation \eqref{bcrs} but takes a finite value. Therefore the boundary condition at $r=R_s$ has to be modified.
Taking into account that the reaction rate $K_r$ is equal to the total flux of particles through the sink surface it can be written as
\begin{align}
    P_r \cdot \rho_m(R_s) &= \int \vec{j}_m(R_s) {\rm d} \vec{A} \nonumber \\
    &= 4 \pi D \left.\frac{\partial}{\partial r} \rho_m(r)\right|_{R_s}
    \label{nbcrs}
\end{align}
Or again in the compact form for all particle states:
\begin{equation}
    P_r \vect{\rho}(R_s) = 4 \pi D |\vec{\nabla} \vect{\rho}(R_s)|
    \label{nbcrsall}
\end{equation}
\section{Expansion in Eigenfunctions of $\mathbb{W}$}
\label{Expansion_in_Eigenfunctions}
Now that the behavior of the system due to the influence of the potential is known (see eq. \eqref{dens_fit} and \eqref{ddens_fit}), it remains to find the solution to the density profile in between these singular points. Therefore one has to further investigate the possibilities that arise from the properties of the transition matrix $\mathbb{W}$. This will be done in the following. The goal is here to find an orthogonal representation of $\mathbb{W}$ and thereby also of equation \eqref{fpmeq3} such that it can be integrated independently for each component.\\

The assumption of the detailed balance property \eqref{detailed_balance} implies the existence of an equilibrium distribution $\vect{\rho}^{(eq)}$. This allows for the definition of the following orthogonal operator $\mathbb{T}$:
\begin{equation}
    \mathbb{T} = \delta_{m,m'} [\rho_m^{(eq)}]^{\frac{1}{2}}.
    \label{symmetrisation_transform}
\end{equation}
This operator happens to be a similarity transform that symmetrizes $\mathbb{W}$. The symmetric form of $\mathbb{W}$ will be denoted by $\mathbb{S}$ in the following and is defined as:
\begin{equation}
    \mathbb{T}^{-1}\mathbb{W}\mathbb{T} = \mathbb{S}.
    \label{symm_rate_matrix}
\end{equation}
The element wise calculation of $\mathbb{S}$ using property \eqref{detailed_balance2} also makes clear why it is symmetric:
\begin{align}
    \mathbb{S}_{il} &= \mathbb{T}^{-1}_{ij} \mathbb{W}_{jk} \mathbb{T}_{kl} = \sum_j \delta_{ij} [\rho^{(eq)}_i]^{-\frac{1}{2}} \mathbb{W}_{jk} \mathbb{T}_{kl} \\ \nonumber
    &= [\rho^{(eq)}_{i}]^{\frac{1}{2}} \sum_{k} \mathbb{W}_{ik} \delta_{kl} [\rho^{(eq)}_l]^{-\frac{1}{2}} = \mathbb{W}_{il}^{\frac{1}{2}} \left( \mathbb{W}_{il} \frac{\rho^{(eq)}_i}{\rho^{(eq)}_l} \right)^{\frac{1}{2}} \\ \nonumber
    &= \left(\mathbb{W}_{il} \mathbb{W}_{li}\right)^{\frac{1}{2}} \\ \nonumber
    \mathbb{S}_{ii} &= \mathbb{W}_{ii}.
\end{align}
The resulting symmetric matrix can then be diagonalized by an orthogonal transformation $\mathbb{D}$:
\begin{equation}
    \mathbb{D}^{\dagger} \mathbb{S} \mathbb{D} = -{\rm diag}\left[ \lambda_i \right].
    \label{orthogonal_transform}
\end{equation}
It can be shown that $\lambda_i > 0$ for $i>1$ and $\lambda_1 = 0$ with the corresponding eigenvector
\begin{equation}
    \mathbb{D}_{i1} = \rho^{(eq)\frac{1}{2}}_{i}.
\end{equation}
For a thorough proof see \cite{Oppenheim1977}.\\
The combined transformation of \eqref{symm_rate_matrix} and \eqref{orthogonal_transform} that orthogonalizes $\mathbb{W}$ will be referred to as $\mathbb{A}$:
\begin{equation}
    \mathbb{A}^{-1}\mathbb{W}\mathbb{A} = -{\rm diag} \left[ \lambda_i \right].
    \label{diag_transform}
\end{equation}
With this the original particle density $\vect{\rho}$ can be written as a superposition of eigenvectors of $\mathbb{W}$, i.e. as a superposition of column vectors of $\mathbb{A}$. Since like the densities themselves the coefficients of their decomposition are not necessarily continuous at the jump discontinuities of the potential it is useful to equally mark them with an upper index:
\begin{equation}
    \vect{\rho}^{(k)}(r) = \sum_i \mathbb{A}_{ij}\tilde{\rho}_{i}^{(k)}(r)
    \label{decomposition}
\end{equation}
where the summands of the decomposition can be called \emph{eigenfunctions} of $\mathbb{W}$. \\
Since the Fokker-Planck operator of the system \eqref{fpo2} can be written as 
\begin{equation}
    \mathbb{F} = \unity \cdot D \vec{\nabla}^{2}
    \label{fpunity}
\end{equation}
for $r \ne a, b$ it obviously commutes with $\mathbb{A}$. Consequently, it is straightforward to deduce equations for the coefficients $\tilde{\vect{\rho}}^{(k)}(r)$ of the decomposition in \eqref{decomposition} from \eqref{fpmeq3}:
\begin{equation}
    \frac{\partial }{\partial t} \tilde{\vect{\rho}}^{(k)}(r,t) = {\rm diag} \left[D \vec{\nabla}^{2} - \lambda_i  \right] \tilde{\vect{\rho}}^{(k)}(r,t).
    \label{fpmeq4}
\end{equation}
This is the desired orthogonal representation of the reaction diffusion problem under study. From this one can calculate steady state solution for the coefficients vector $\tilde{\vect{\rho}}^{(k)}$. Given this solution the density profiles can be obtained by plugging these coefficients into the decomposition \eqref{decomposition}.
\par
It is easy to check that for the steady state case the desired solution reads:
\begin{align}
    \tilde{\rho}_{1}^{(k)}(r) &= c_{1,1}^{(k)} + c_{1,2}^{(k)} \frac{1}{r} \nonumber \\
    \tilde{\rho}_{i \ne 1}^{(k)}(r) &= c_{i,1}^{(k)}\frac{1}{r} \exp\left[-r\sqrt{\frac{\lambda_i}{D}}\right] + c_{i,2}^{(k)}\frac{1}{r} \exp\left[r\sqrt{\frac{\lambda_i}{D}}\right] 
    \label{fp_ind_sol}
\end{align}
where it is important do distinguish between the case of $i=1$ where $\lambda_i = 0$ and $i>1$ where the eigenvalues are positive.\\

Note that the solution corresponding to the first eigenvalue $\lambda_1 = 0$ equals the one derived in \eqref{steady_state_density} for the ungated problem. Together with the fitting conditions obtained in \eqref{dens_fit} this would result in the steady state solution for a constant boxcar shaped potential barrier that could also be computed from \eqref{rho_debye}. So far the calculations are consistent with preexisting results. \\
The other solutions that correspond to the nonzero eigenvalues of the transition rate matrix $\lambda_i>0$ describe deviations from this solution due to the metastability of the potential barrier. Note that they exponentially decay in space with a \textit{decay length}
\begin{equation}
    \boxed{r_d^{(i)} = \sqrt{\frac{D}{\lambda_i}}}
    \label{decay_length}
\end{equation}
that is unique for each state of the potential. \\
It is clear that the coefficients $c^{(k)}_{i,j}$ can be calculated from the boundary and fit conditions obtained earlier. It is now necessary to find a systematic way to do this.
\section{Treatment of Boundary and Fit Conditions}
\label{Treatment_of_Boundary_and_Fit_Conditions}
To make use of the boundary and fit conditions from section \ref{Fit_Conditions} they have to be brought to the same basis in that the density profiles were calculated previously. \\
For the boundary conditions at $r=R_s$ and $r \rightarrow \infty$ given in equations \eqref{bcrs} and \eqref{detailed_balance2} the transformation reads:
\begin{align}
    \mathbb{A}^{-1}\vect{\rho}^{(I)}(R_s) &= \vect{\tilde{\rho}}^{(I)}(R_s) = 0, \nonumber \\
    \vect{\tilde{\rho}}(r \rightarrow \infty) &= \mathbb{A}^{-1} \vect{\rho}^{(eq)} = (1,0,\cdots,0)^{T}.
\end{align}
Note that for $r\rightarrow \infty$ only the first coefficient $\tilde{\rho}_1$ is nonzero since it corresponds to the eigenvalue $\lambda_1=0$ which has the equilibrium particle density as associated eigenvector.\\
For the fit conditions at $r=a,b$ the transformation reads:
\begin{align}
    \vect{\tilde{\rho}}^{(I)}(a) &= \mathbb{A}^{-1}\mathbb{U}_a\mathbb{A} \vect{\tilde{\rho}}^{(II)}(a), \\ \nonumber
    \vect{\tilde{\rho} '}^{(I)}(a) &= \vect{\tilde{\rho} '}^{(II)}(a), \\ \nonumber
    \vect{\tilde{\rho}}^{(II)}(b) &= \mathbb{A}^{-1}\mathbb{U}_b\mathbb{A} \vect{\tilde{\rho}}^{(III)}(b), \\ \nonumber
    \vect{\tilde{\rho} '}^{(III)}(b) &= \vect{\tilde{\rho} '}^{(II)}(b).
\end{align}
Now we have to find an expression that allows for the calculation of the coefficients $c_{i,k}^{(j)}$ from these transformed boundary and fit conditions.\\
Therefore it is useful to write the solution of equation \eqref{fpmeq4} as the product of an $r$ dependent part and a vector of the corresponding coefficients:
\begin{equation}
    \tilde{\vect{\rho}}^{(k)} = \underbrace{ \left( \begin{array}{cllllllll}
       1   & \frac{1}{r}   & 0                 & 0                 & 0              & 0             & 0 & \cdots &\\
       0   & 0             &\frac{1}{r} e^{-r / r_d^{(2)}}   &\frac{1}{r} e^{r / r_d^{(2)} }   & 0              & 0             & 0 & \cdots &\\
       0   & 0             & 0                 & 0                 &\frac{1}{r} e^{-r / r_d^{(3)}} &\frac{1}{r} e^{r/r_d^{(3)}} & 0 & \cdots &\\
       \vdots  &&&&&&&\ddots &\\
       \vdots  &&&&&&&&\ddots
   \end{array} \right)}_\text{\large$\hat{\rho}(r)$}
   \underbrace{\left(\begin{array}{c}  
       c_{1,1}^{(k)} \\ 
       c_{1,2}^{(k)} \\ 
       c_{2,1}^{(k)} \\ 
       c_{2,2}^{(k)}  \\ 
       \vdots 
   \end{array} \right)}_\text{\large$\vect{c}^{(k)}$}.
\end{equation}
Using this notation, the boundary and fit conditions read:
\begin{align}
    \hat{\rho}(R_s) \cdot \vect{c}^{(I)} &= 0, \nonumber \\
    \hat{\rho}(a)\left( \vect{c}^{(I)} - \mathbb{A}^{-1} \mathbb{U}_a \mathbb{A} \vect{c}^{(II)} \right) &= 0, \nonumber \\
    \hat{\rho}'(a)\left( \vect{c}^{(I)} - \vect{c}^{(II)} \right) &= 0, \nonumber \\
    \hat{\rho}(b)\left( \vect{c}^{(II)} - \mathbb{A}^{-1} \mathbb{U}_b \mathbb{A} \vect{c}^{(III)} \right) &= 0, \nonumber \\
    \hat{\rho}'(b)\left( \vect{c}^{(II)} - \vect{c}^{(III)} \right) &= 0, \nonumber \\
    \hat{\rho}(\infty) \cdot \vect{c}^{(III)} &=(1,0, \cdots ,0)^{T}. \nonumber 
\end{align}
These conditions can be put in one $6 N$ dimensional system of linear equations:
\begin{equation}
    \left( \begin{array}{ccc}
        \hat{\rho}(R_s) & 0 & 0 \\
        \hat{\rho}(a)   & -\tilde{\mathbb{U}}\hat{\rho}(a) & 0 \\
        \hat{\rho}'(a) & -\hat{\rho}'(a) & 0 \\
        0 &  -\tilde{\mathbb{U}}\hat{\rho}(b) & \hat{\rho}(b) \\
        0 &  -\hat{\rho}'(a) & - \hat{\rho}'(a) \\
        0 & 0 & \hat{\rho}(r\rightarrow \infty)
    \end{array}\right) \left( \begin{array}{c} c_{1,1}^{1} \\ \vdots \\ \vdots \\ \vdots \\ c_{N,2}^{3} \end{array} \right) = 
    \left( \begin{array}{c} 0 \\ \vdots \\ 0 \\ 1 \\ 0 \\ \vdots \end{array} \right) \begin{array}{c} \vdots \\ \vdots \\ \vdots \\ i = 5N+1 \\ \vdots \\ \vdots \end{array}
    \label{lgs}
\end{equation}
where dashes denote derivatives with respect to $r$. In some cases it is possible to solve this system of linear equations analytically to derive expressions for $c_{i,k}^{(j)}$. \\ 
This will be done in the following sections for several examples. As stated before, the actual density profiles can then be calculated by plugging in these expressions into the decomposition \eqref{decomposition}. The last step is then to calculate the total rate of Brownian particles interacting with the sink from their density at the sink boundary.
\section{Calculation of Rates}
\label{Calculation_of_Rates}
The rate of the particles absorbed by the sink is calculated via integration over the flux through the surface of the sphere with radius $R_s$:
\begin{align}
    K   &= \int_{\partial \Omega_{R_s}} \vec{J} {\rm d} \vec{A}\nonumber\\
    &= \int_{\partial \Omega_{R_s}} D \vec{\nabla} \sum_{n=1}^{N} \rho_n^{(1)}(r)\nonumber \\
    &= 4 \pi D R_s^{2} \sum_{n=1}^{N} \left\{ \mathbb{A} \left. \frac{\partial}{ \partial r}\right|_{R_s} \tilde{\vect{\rho}} \right\}_n.
    \label{Rate}
\end{align}

\section{Summary}
In the previous section 
\begin{itemize}
    \item fitting conditions for steady state density profiles at jump discontinuities of potential barriers were calculated,
    \item a treatment for the fluctuations of the potential barrier in terms of eigenfunctions of its transition rate matrix was proposed assuming it satisfies a detailed balance property,
    \item  the persistence length of the influence of the potential fluctuations on the particle density profile was introduced,
    \item and a scheme for the calculation of the integration constants of the particle density functions was provided.
\end{itemize}
This allows an analytic treatment for the problem of diffusion controlled reaction rates over piecewise constant barriers fluctuating between arbitrarily many states of different heights. The next section evaluates the results from these methods with an example and compares them to numeric results obtained by the methods outlined in section \ref{numeric_model}.
\newpage

