\chapter{Summary and Conclusion}
\label{conclusion}

In the introductory part in chapter \ref{intro} of this thesis, the topic of reaction rates over fluctuating  barriers was motivated. First, Several examples from current theoretical research were given, ranging from transition rate theory over fluctuating barriers and adsorption in multicomponent systems with static interaction, to adsorption to reactive sinks that exhibit fluctuating surface properties. Second, the introduction describes a variety of applications, from phase transitions in hydrogels and hydrodynamic fluctuations in cavity ligand binding, to pH controlled effects in protein folding. \\
A minimal model for diffusion controlled reaction rates over fluctuating barriers was introduced to provide a feasible approach to the problem. This model consisted of a spherical sink surrounded by a step shaped barrier that reactants have to overcome to react/adsorb. The barrier fluctuates between states of different height as a Markov process.\\

Chapter \ref{Short_Introduction_to_Stochastic_Processes} gave a selection of textbook knowledge and historic examples on the topic of stochastic processes and diffusion controlled reaction rates to provide the reader with the set of tools that was used in the latter to analyze the minimal model setup for diffusion controlled reaction rates that was motivated before. \\ 

Chapter \ref{numeric_model} outlines two different numeric models for the description of the minimal model. First, Brownian dynamics simulations for the treatment of the underlying stochastic differential equations in discrete time. Second the numerical method of lines that serves well for the integration of the partial differential equations that describe the time evolution of the system in terms of macroscopic properties such as particle density functions.\\
It was stressed that although the results from both methods are in good agreement both have their advantages and disadvantages. Where the method of lines is highly efficient in terms of performance accuracy and precision, it does not give any insight on microscopic properties. Brownian dynamics simulations offer detailed insight in the microscopic behavior of the system, but suffer from the fact, that the accuracy of the result scales only with $1/\sqrt{T}$ with $T$ being the computation time.  \begin{itemize}
    \item Initial results for a two state repulsive fluctuating barrier show interesting signatures in particle density profiles that qualitatively depend on the switching rate of the potential barrier.
\end{itemize}

A thorough analytic treatment of the model system was given in section \ref{Reaction_Rates_over_Fluctuating_Barriers}. Starting with the description of the system as a composite Markov process the analytic solution to the problem is derived in three steps. First, the derivation of boundary conditions at the surface of the sink and for $r \rightarrow \infty$ and fit conditions at the jump discontinuities of the potential barrier. Second, a solution for the particle density profiles in terms of an expansion in eigenfunctions of the transition rate matrix describing the process of the barrier fluctuations. Third, an algebraic scheme for the derivation of the integration constants of this solution. \\
This solution assumes statistic independence of the barrier fluctuations from the substrate diffusion and vice versa everywhere but at the jump discontinuities of the barrier. It also assumes that the barrier is an ideal step potential and that its fluctuations obey the detailed balance property. It also introduced a quantity dubbed decay length \eqref{decay_length}, useful to describe the nonlocality of the influence of the barrier fluctuations on the particle density profiles.\\

Section \ref{results} finally presented the results of a detailed study of the model system with a two state barrier and symmetric barrier fluctuations where one of the two states of the barrier was chosen to be $U(r) \equiv 0$. The choice of this particular setup keeps the free parameters of the system to a minimum while still showing very complex behavior.
The analytic solution from the previous chapter reproduced the numeric results that have already been presented.  The signatures that emerged in the particle density profiles were analyzed in terms of spacial and reactive fluxes. This analysis revealed that 
\begin{itemize}
    \item the complex signatures in particle density profiles are caused by circular fluxes around the boundaries of the potential barrier that are of different spatial extend depending on the decay length of the system. 
\end{itemize}
This dependency indicates that the decay length does indeed describe the nonlocality of the influence of the barrier fluctuations. It was then shown that
\begin{itemize}
    \item the spatial overlay of these circular fluxes from both boundaries of the barrier leads to \emph{resonant activation} that has previously been observed in escape problems over fluctuating barriers.
\end{itemize}
This essentially means that the rate of particles crossing the barrier and reacting with the sink depends on the rate of the barrier fluctuations and that it is maximized for a certain resonant value of the barrier fluctuation rate. \\
Further, the dependency of the reaction rate on the free parameters of the system was analyzed, including the rate of the barrier fluctuations, the barrier spacing, width and the barrier height in its active state. For each of these parameters the appropriate limits were investigated. \\
In the limit of slow barrier fluctuations [long decay length] in section \ref{lim_long_rd}, the reaction rate is equal to the average over the reaction rate over the different configurations of the barrier. This has has already been speculated in the introduction. In the limit of fast barrier fluctuations the reaction rate over a step shaped barrier is NOT equal to the reaction rate over a potential of mean force. 
A more detailed analysis of the situation revealed that, in the fast switching limit, there is a fundamental qualitative difference between smooth and step shaped barriers. This difference basically comes from the order of the limits from finite to infinite switching rates, and from a smooth to a step shaped barrier.
If the limit from finite to infinitely large switching rates is taken first, the resulting rate is that over a potential of mean force. If instead the transition from a smooth to a step shaped barrier is taken first, the resulting rate is given by the expression in equation \eqref{kla}. For large but finite switching rates the applicability of one or the other description was shown to depend on the height of the barrier in the active state and the ratio of the decay length and the width of the area in which the particles are subject to a force from the barrier. \\
The study of the dependence of the reaction rate on the barrier spacing in section \ref{barrier_spacing} revealed a power law dependency of the resonant switching rate on the barrier spacing and a saturation of the resulting maximum reaction rate in the limit of large barrier spacing. It also revealed two effects that are in sharp contrast to what Debye theory:
\begin{itemize}
    \item Due to resonant activation the rate over a repulsive fluctuating barrier can exceed the Smoluchowski reaction rate for a sink without any barrier, and
    \item the rate over a repulsive fluctuating barrier can even exceed the rate over an attractive fluctuating barrier of equivalent height.
\end{itemize}
The analysis of the dependency of the reaction rate on the height of the barrier in its active state in section \ref{u1_dependence} showed that most features of finitely high barriers are persistent in the limit of infinitely high barriers. The only exception was the slow switching limit in the case of an attractive barrier. In section \ref{lim_long_rd} the slow switching limit has been evaluated under the assumption that the relaxation time of the particle density profile is negligible relative to the lifetime of each barrier state. Now in the case of an attractive barrier the relaxation time is given by the time that it takes for the barrier to gather enough particles to raise the density on its inside to a level that makes escape from the barrier equally likely as the absorption into it. Since the density is not bounded from above, this obviously never happens for an ideally attractive barrier. Therefore, the density profiles never relax and thus the timescale separation breaks down. This results in a slow switching reaction rate, that is qualitatively different for finitely and infinitely attractive barriers. \\
The analysis of the limit of small barrier spacing in section \ref{1dLimit} reproduced results that were previously known from the study of a so called ``rated sphere'' i.e. the study of a spherical sink that fluctuates between states with different surface reactivity. It showed (as suggested by Debye rate theory \ref{The_Debye_Reaction_Rate}) that an infinitely attractive and a finitely high barrier have no effect if they are infinitely thin. For an infinitely high and infinitely thin barrier (where the limit of infinite height is taken first) findings of Szabo et al. \cite{Szabo1982} have been reproduced. Namely, that in the fast switching limit the barrier has no effect, whereas in the slow switching limit the reaction rate is given by the average of the rates in each configuration of the barrier. \\
Finally section \ref{mapping} tried to bridge the gap between theory and possible realizations through experiments. Therefore it checks the validity of two assumptions that are commonly made. First, the validity of the description of the barrier by a potential of mean force and Debye theory and second, the derivation of effective adsorption rates from kinetic rates and surface rates by simple summation of their inverse (see equation \eqref{Deff}). It became evident that the applicability of these assumptions strongly depends on the validity of the time scale separation between barrier fluctuations and diffusive relaxation.To make this clear:
\begin{itemize}
    \item It is shown that in general, Debye theory is NOT applicable to calculate reaction rates over fluctuating barriers, and
\end{itemize}
for system that are governed by a diffusion limited rate $K_D$ and a surface rate of the sink $K_A$
\begin{itemize}
    \item it is shown that if the kinetic rate $K_D$ is influenced by a fluctuating barrier, it is NOT possible to calculate the effective reaction rate the commonly used formula:
        \begin{equation}
            K_{eff} = \frac{K_A K_D}{K_A + K_D}. \nonumber
        \end{equation}
\end{itemize}
\section{Outlook}
For future research it would be interesting to identify experimental setups that actually exhibit the effects that are outlined in this thesis, like for instance resonant activation and an increase of reaction rates due to a fluctuating repulsive potential barrier. It would also be intriguing to find an analytic treatment for potential barriers that are not step shaped. 
