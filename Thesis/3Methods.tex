\chapter{Fundamentals and Methods}
\label{Short_Introduction_to_Stochastic_Processes}
This section introduces some of the basic concepts and fundamentals of stochastic processes and reaction rate theory as 
far as they concern the problem under study. For a broader context please refer to standard textbooks \cite{VanKampen1992, doob1953stochastic, ross1996stochastic} 
or review papers on the topic \cite{Calef1983a, Bressloff2013}. The goal here is to give a framework for the treatment of 
composite Markov processes in discrete and continuous space and to present the reference case for diffusion controlled 
reactions rates in the Debye-Smoluchowski interpretation \cite{Smoluchowski1917a, Debye1942}. \\
Therefore it takes the well-trodden trail from the Chapman Kolmogorov equation via the Kramers Moyal expansion to the 
Kolmogorov forward or Fokker-Planck equation. Here it takes a step to the side to calculate the drift and diffusion 
coefficients from the Langevin equation in the 'overdamped' case of a Brownian particle before it shows the derivation 
of the Master equation again from the Chapman-Kolmogorov equation.\\
Thereafter the methods introduced before are used to illustrate the treatment of multivariate Markov processes in discrete 
and continuous space. \\
Finally it gives a rigorous derivation of the diffusion controlled reaction rate from the Fokker-Planck description of 
a spherical sink embedded in a bath of Brownian particles, as historically done by Smoluchowski and Debye.
\section{The Fokker-Planck Equation}
\label{The_Fokker_Planck_Equation}
By definition a stochastic process is said to have the \emph{Markov property} if for any $n$ successive time steps its conditional probability density function is governed by the following relation:
\begin{equation}
    P(x_{n},t_{n}|x_{1},t_{1};\cdots;x_{n-1},t_{n-1}) = P(x_{n},t_{n}|x_{n-1},t_{n-1}), \quad t_{n}>t_{n-1}> \cdots >t_{1},
    \label{}
\end{equation}
i.e. the conditional probability to be at $x_n$ at $t_n$ is only determined by the value of $x_{n-1}$ at $t_{n-1}$ and not influenced by any knowledge of the process at earlier times.\\
Hence, the entire realization of the process is determined by the initial distribution $P(x_1,t_1)$ and the two step transition probability $P(x_{n},t_{n}|x_{n-1},t_{n-1})$ and every multi step probability distribution function can be expressed as a hierarchy of these two. \\
For instance for $ t_n > t_{n-1} > \cdots > t_1$ one has:
\begin{align}
    P(x_1,t_1;x_2,t_2;\cdots;x_n,t_n) &= P(x_n,t_n|x_{n-1},t_{n-1})P(x_{n-1},t_{n-1}|x_{n-2},t_{n-2}) \cdots \nonumber \\
                                      & \cdots P(x_2,t_2|x_1,t_1)P(x_1,t_1).
    \label{hierarchy}
\end{align}
For only three time steps $t_3>t_2>t_1$ the integration of the three step joint probability distribution over the intermediate step leads to:
\begin{equation}
    P(x_3,t_3;x_1,t_1) = P(x_1,t_1)\int P(x_3,t_3|x_2,t_2) P(x_2,t_2|x_1,t_1) {\rm d} x_2,
\end{equation}
and division by $P(x_1,t_1)$ results in the well known \emph{Chapman Kolmogorov} equation:
\begin{equation}
    P(x_3,t_3|x_1,t_1) = \int P(x_3,t_3|x_2,t_2) P(x_2,t_2|x_1,t_1) {\rm d} x_2.
    \label{CKeq}
\end{equation}
An equivalent formulation of the Chapman Kolmogorov equation is the \emph{Kramers Moyal expansion} \cite{Kramers1940, Moyal1949}. To derive it the expression for the transition probabilities \eqref{CKeq} is multiplied with the initial probability distribution $P(x_1,t_1)$ and integrated over $x_1$ which leads to
\begin{equation}
    P(x_3,t_3) = \int P(x_3,t_3|x_2,t_2)P(x_2,t_2) {\rm d} x_2.
    \label{CKeq3}
\end{equation}
The integrand may be written in terms of $\Delta x = x_3 - x_2$ and then be expanded for $\Delta x \ll 1$
\begin{align}
    P(x_3,t_3|x_2,t_2)P(x_2,t_2) &= P( (x_3 - \Delta x) + \Delta x, t_3|x_3 - \Delta x, t_2)P( (x_3 - \Delta x), t_2) \nonumber \\
    &= \sum_{n=0}^{\infty} \frac{(-1)^{n}}{n!}\frac{\partial^{n}}{\partial x_3^{n}}\left\{P(x_3 + \Delta x, t_3|x_3,t_2) P(x_3,t_2)\right\}.
\end{align}
This is again plugged into \eqref{CKeq3}. Integration over $\Delta x$ and substitution of $\Delta t = t_3 - t_2$ then yields:
\begin{equation}
    P(x_3,t_2 + \Delta t) = \sum_{n=0}^{\infty} \frac{(-1)^{n}}{n!}\frac{\partial^{n}}{\partial x_3^{n}}\left\{ M_{n}(x_3,t,\Delta t) P(x_3,t_2) \right\}
    \label{KME1}
\end{equation}
where $M_n$ are the so called \textit{jump moments} defined by
\begin{equation}
    M_n(x,t,\Delta t) = \int (\Delta x)^{n} P(x + \Delta x, t + \Delta t | x, t) {\rm d} (\Delta x).
    \label{Jump_moments}
\end{equation}
Note that from normalization of $P(x+\Delta x, t+ \Delta t|x,t)$ it follows that the lowest of the jump moments $ M_0(x,t,\Delta t) $ is equal to one. \\
This formulation still describes the time evolution of the probability distribution in terms of discrete time steps. To derive a formulation in a continuous time variable one subtracts the first term of the sum on the right hand side of equation \eqref{KME1}, divides by $\Delta t$ and takes the limit of $\Delta t \rightarrow 0$ to obtain \\
\begin{equation}
    \frac{\partial P(x,t)}{\partial t} = \sum_{n = 1}^{\infty}\frac{(-1)^{n}}{n!}\frac{\partial^n}{\partial x^n} \left\{ \lim_{\Delta t \rightarrow 0} M_n(x,t,\Delta t) P(x,t) \right\}.
    \label{KME2}
\end{equation}
As we will see later it is also reasonable to assume that for short time differences the jump moments go linear with $\Delta t$:
\begin{equation}
    M_{n}(x,t,\Delta t) \sim \Delta t + \mathcal{O}(\Delta t^{2}).
\end{equation}
Bearing this in mind it makes sense to introduce so called \emph{kinetic coefficients} of the form
\begin{align}
    K^{(n)}(x,t) &= \lim_{\Delta t \rightarrow 0} \frac{1}{ \Delta t } M_n(x,t,\Delta t) \nonumber \\
    &=\lim_{\Delta t \rightarrow 0} \frac{1}{\Delta t} \int (\Delta x)^n P(x+\Delta x,t+\Delta t|x,t) {\rm d}(\Delta x) .
    \label{kinetic_coefficients}
\end{align}
(But note that this is only a matter of notation and does not require the small $\Delta t$ behavior of $M_n$ mentioned  before!) \\
Substituting these coefficients back into equation \eqref{KME2} results in the desired formulation of the Kramers Moyal expansion \cite{Moyal1949}:
\begin{equation}
    \frac{\partial P(x,t)}{\partial t} = \sum_{n = 1}^{\infty}\frac{(-1)^{n}}{n!}\frac{\partial^n}{\partial x^n} \left\{ K^{(n)}(x,t) P(x,t) \right\}.
    \label{Kramers Moyal expansion}
\end{equation}

So far nothing has been assumed, other than the Markov property and the existence of the Taylor series. However in many application the examination of the jump moments reveals that it is a suitable approximation to truncate the expansion for $n>2$. In this case, one obtains the following form, known as the \textit{Fokker Planck equation}:
\begin{equation}
    \boxed{    \frac{\partial P(x,t)}{\partial t} = - \frac{\partial}{\partial x} \left[K^{(1)}(x,t)P(x,t) \right] + \frac{1}{2}\frac{\partial^2}{\partial x^2}\left[ K^{(2)}(x,t)P(x,t) \right] }
    \label{FPE}
\end{equation}
where $K^{(1)}$ and $K^{(2)}$ are independent of $t$ if the process is stationary. \\
\section{Brownian Motion}
\label{Brownian_Motion}
Brownian motion is the oldest example of a Markov process that is known in physics \cite{Einstein1905,Smoluchowski1906}. It emerges from the picture of a heavy particle in a solution of lighter particles, that collide with each other in a random fashion. Consequently, the velocity of the heavier particle undergoes a series of supposedly uncorrelated jumps. When its velocity $v$ has a certain direction, there will be on average more collisions from this side, than from the other. Therefore the probability of a change in velocity $\Delta v$ depends on its current value, but not on the velocity at earlier times. As a consequence, the velocity of the heavier particle can be treated as a Markov process. When the whole system is in equilibrium the process is stationary and its autocorrelation time is the time in which an initial velocity of the heavy particle is damped out. \\
Now in the \textit{overdamped limit} the correlation time of the velocity is much smaller then the time between two observations of the heavy particle. In this case the observation of the particle gives a series $x(t_1), x(t_2), \cdots , x(t_n)$ of subsequent particle positions. Each displacement $x(t_{n}) - x(t_{n-1})$ does not depend on the previous history of the process, i.e. it is independent of $x(t_{n-2}), \cdots , x(t_{1})$. Hence not only the velocity, but also the position of the particle itself is a Markov process (at least on a coarse grained timescale). 
\par
In the following we will start with the \emph{Langevin equation} \cite{Langevin1908} for the position of a particle in a fluid and calculate the corresponding kinetic coefficients to obtain a Fokker-Planck equation for the position probability distribution.\\ 
This Langevin equation is given by:
\begin{equation}
    m \frac{{\rm d}^2 x}{{\rm d}t^2} = -\gamma \frac{ {\rm d}x}{{\rm d}t} + f(x) + \varepsilon(t)
    \label{Langewin equation}
\end{equation}
where $m$ is the mass and $\gamma$ is the friction constant of the particle in the solute, $f(x)$ describes any external forces present and $\varepsilon(t)$ is a random process describing the collision interaction of the particle and the solute. From the central limit theorem it follows that $\varepsilon(t)$ must follow a Gaussian distribution. It is also assumed that the velocities in the system are locally equilibrated, i.e. they are governed by a Boltzmann distribution and their second moment will be given by:
\begin{equation}
    \left< \dot{x}^{2} \right> = \frac{K_B T}{m}
    \label{2nd_moment_of_velocities}
\end{equation}
with $K_B$ being the Boltzmann constant. Furthermore it is assumed that the spacial variable $x(t)$ and the random force $\varepsilon (t)$ are not correlated. \\
Given these assumptions it can be shown that the autocorrelation of the random force is given by:
\begin{equation}
    \left< \varepsilon(t) \varepsilon(t') \right> = 2 K_B T \gamma \delta(|t-t'|)
    \label{ff_autocorrelation}
\end{equation}
and that the autocorrelation function of the velocities is equal to
\begin{align}
    \left< \dot{x}(t) \dot{x}(t') \right> &= \frac{K_B T}{m} \exp \left\{-\frac{\gamma}{m}|t-t'|\right\} \nonumber \\
    &=\frac{K_B T}{m} \exp\left\{-\frac{\tau}{\tau_0} \right\}
    \label{vv_autocorrelation}
\end{align}
where a typical timescale for velocity relaxation, $\tau_0 = m/\gamma$ appears.
In the so called overdamped limit $ \tau_0 $ is very small, such that the Langevin equation can be approximated by
\begin{equation}
    \gamma \frac{ {\rm d}x}{{\rm d}t} = f(x) + \varepsilon(t).
    \label{BD1}
\end{equation}
As described in the introductory part of this section this process must be observed on a coarse grained timescale to be considered Markovian. Therefore one integrates equation \eqref{BD1} over one time step $\Delta t$ to describe it in discrete time. Doing so results in:
\begin{equation}
x(t + \Delta t) = x(t) + \frac{1}{\gamma} f(x) \Delta t + \frac{1}{\gamma} \int\limits_{t}^{t+\Delta t} \varepsilon(t) {\rm d} t.
    \label{od1}
\end{equation}
The last term on the right hand side can be expressed in terms of an effective random force of the form:
\begin{equation}
    \varepsilon'(t) \Delta t = \frac{1}{\gamma}\int\limits_{t}^{t + \Delta t} \varepsilon(t) {\rm d} t
    \label{eff_rdf}
\end{equation}
such that equation \eqref{od1} reads:
\begin{equation}
    x(t + \Delta t) = x(t) + \frac{1}{\gamma} f(x) \Delta t + \varepsilon'(t) \Delta t.
    \label{od2}
\end{equation}
This effective random force must again be Gaussian distributed and from equation \eqref{ff_autocorrelation} it follows, that its autocorrelation is given by:
\begin{equation}
    \left< \varepsilon'(t) \varepsilon'(t') \right> = \frac{2 K_B T \gamma}{ \Delta t}
    \label{ff_eff_autocorrelation}
\end{equation}
and its distribution is therefore equal to:
\begin{equation}
    P(\varepsilon ' ) = \sqrt{\frac{\Delta t}{4 \pi D \gamma^{2}}} \exp \left[ - \frac{\varepsilon ^{\prime 2} \Delta t}{4 D \gamma^{2}} \right]
    \label{eps dist}
\end{equation}
where the diffusion constant $D$ is given by the \emph{Einstein-Smoluchowski relation}:
\begin{equation}
    D = \frac{K_B T}{\gamma}.
    \label{ESR}
\end{equation}
From the distribution of the random force one can compute the transition probability $P(x+\Delta x, t+ \Delta t| x, t)$ for the Brownian particle as the estimate over its translocations:
\begin{equation}
    P(x+\Delta x,t+\Delta t|x,t)  = \left< \delta \left(  \Delta x - \left[x(t-\Delta t) - x(t)\right] \right)\right>.
\end{equation}
Here we use equation \eqref{od2} and \eqref{eps dist} to write this as
\begin{align}
     P(x+\Delta x,t+\Delta t|x,t) = \int & {\rm d}\varepsilon '  \delta  \left(  \Delta x - \left( \frac{1}{\gamma} f(x) \Delta t + \varepsilon'(t) \Delta t \right) \right) \nonumber \\ 
     & \times \sqrt{\frac{\Delta t}{4 \pi D \gamma^{2}}} \exp \left[ - \frac{\varepsilon  ^{\prime 2} \Delta t}{4 D \gamma^{2}} \right]
 \end{align}
 which finally evaluates to 
 \begin{equation}
      P(x+\Delta x,t+\Delta t|x,t) = \sqrt{\frac{1}{4 \pi D \Delta t}} \exp \left[ \frac{-\left(\Delta x - f(x) \frac{\Delta t}{\gamma} \right)^2}{4 D \Delta t} \right].
    \label{BM_transition_probability}
\end{equation}
Now it is straight forward to calculate the jump moments from this transition probability according to equation \eqref{Jump_moments} and as it was already anticipated it turns out to be true that the first and second moment are linear in $\Delta t$ in leading order: 
\begin{equation}
    M_1(x,t,\Delta t) = f(x)\frac{\Delta t}{\gamma}, \qquad M_2(x,t,\Delta t) = 2 D \Delta t + \left(f(x)\frac{\Delta t}{\gamma} \right)^{2}
    \label{BM_jump_moments}
\end{equation}
such that the kinetic coefficients \eqref{kinetic_coefficients} are equal to
\begin{equation}
    \boxed{K^{(1)}(x,t) = \frac{f(x)}{\gamma}, \qquad K^{(2)}(x,t) = 2 D.}
    \label{BD_kinetic_coefficients}
\end{equation}
It can be shown that all higher jump moments are of order $\mathcal{O}(\Delta t ^{2})$ such that it is indeed valid to truncate the Kramers Moyal expansion after the second term. Therefore the time evolution of the probability distribution of a Brownian particle is fully described by the following expression:
\begin{equation}
    \frac{\partial P(x,t)}{\partial t} = - \frac{\partial}{\partial x} \left[\frac{f(x)}{\gamma}P(x,t) \right] + D\frac{\partial^2}{\partial x^2}\left[P(x,t) \right] .
    \label{FPE2}
\end{equation}

\section{The Master Equation}
\label{The_Master_Equation}
The master equation is yet another equivalent formulation of the Chapman Kolmogorov equation \eqref{CKeq}. The Chapman Kolmogorov equation usually is of not much help since it is essentially a property of the solution for the transition probabilities. The master equation however is its formulation in terms of a differential equation and is far more useful especially for the description in a discrete state space. \\
In order to derive it one has to reason first about the short time behavior of the transition probabilities. From the Chapman Kolmogorov equation for equal time arguments it is obvious that
\begin{equation}
    P(x_2,t|x_1,t) = \delta(x_1-x_2)
    \label{leading_order}
\end{equation}
which is the zero order term of the following formulation of the short time transition probability $P(x_2,t+\Delta t|x_1,t)$:
\begin{equation}
    P(x_2,t+\Delta t|x_1,t) = W(x_2|x_1)\Delta t + \left[ 1 - \Delta t \int {\rm d} x_3 W(x_3|x_1) \right] \delta(x_2-x_1) + O(\Delta t ^{2}).
    \label{master_assumption}
\end{equation}
For a better understanding of this expression imagine the following: At time $t$ the system was in state $x_1$. In the subsequent time interval $\Delta t$ it might have made a transition to the state $x_2$.
Here the probability of the transition is expressed in terms of the (non negative) {\it transition rate}  $W(x_2|x_1)$ i.e. the transition probability per unit time from state $x_1$ to $x_2$. So the first term on the right hand side of equation \eqref{master_assumption} gives the transition probability from state $x_1$ to another state $x_2 \ne x_1$ whereas the second term on the right hand side is equal to one minus the probability to move to any other state i.e. the probability for the system to rest in state $x_1$ during the time $\Delta t$.\\
To maintain a readable form it is common to introduce the notation
\begin{equation}
    T_\tau (x_2|x_1) = P(x_2,t+\tau|x_1,t)
\end{equation}
and to omit the absolute time dependence, since the process is assumed to be stationary. \\
The Chapman-Kolmogorov equation in this formulation reads:
\begin{equation}
    T_{\tau + \tau'}(x_3|x_2) = \int T_{\tau'}(x_3|x_2)T_{\tau}(x_2|x_1){\rm d} x_2.
    \label{K2}
\end{equation}
Now the insertion of equation \eqref{master_assumption} on the right hand side leads to:
\begin{align*}
    T_{\tau+\tau'}(x_3|x_1) = \int  & \left\{ \left[1 - \tau' \int {\rm d} z W(z|x_3) \right] \delta(x_3 - x_2) \right. \\
    & \left. \vphantom{\int {\rm d} z}+ \tau' W(x_3|x_2) \right\} T_{\tau}(x_2|x_1){\rm d} x_2
\end{align*}
and regrouping the terms and dividing by $\tau ' $ results in:
\begin{align*}
    \frac{1}{\tau'} T_{\tau+\tau'}(x_3|x_1) &= \frac{1}{\tau'}  \int T_{\tau}(x_2|x_1) \delta(x_3 - x_2){\rm d} x_2\\
    &- \int \left\{ W(z|x_2)  T_{\tau}(x_2|x_1)\delta(x_3 - x_2) \right\}{\rm d} z {\rm d} x_2 \\
    &+ \int \left\{ W(x_3|x_2) T_{\tau}(x_2|x_1) \right\}{\rm d} x_2.
\end{align*}
The integrals in the first and the second term on the right hand side can be evaluated and the fist term can be moved to the left hand side:
\begin{align*}
    \frac{1}{\tau}\left\{  T_{\tau+\tau'}(x_3|x_1) - T_{\tau}(x_3|x_1)\right\} &= \int \left\{ W(x_3|x_2) T_{\tau}(x_2|x_1) \right\}{\rm d} x_2  \nonumber \\
    &- \int \left\{ W(z|x_3)  T_{\tau}(x_3|x_1) \right\}{\rm d} z.
\end{align*}
Finally one renames $z$ to $x_2$ and takes the limit of $\tau' \rightarrow 0$ to obtain the well known formulation of the master equation in continuous space:
\begin{equation}
    \frac{\partial}{\partial \tau}T_{\tau}(x_3|x_1) = \int \left\{ W(x_3|x_2) T_{\tau}(x_2|x_1) - W(x_2|x_3) T_{\tau}(x_3|x_1) \right\}{\rm d} x_2
    \label{continuous_space_master_equation}
\end{equation}where the $W(x_i|x_j)$ are properties of the specific process.
This equation describes the time development of the transition probabilities given an initial condition $(x_1,t_1)$. A more intuitive form follows from multiplying with a distribution of initial conditions $P(x_1,t_1)$ and integrating over its spatial coordinate $x_1$:
\begin{equation}
    \frac{\partial P(x,t)}{\partial t} = \int \left\{ W(x|x') P(x',t) - W(x'|x)P(x,t) \right\} {\rm } x'.
\end{equation}
In this form the meaning becomes particularly clear. The master equation is a \textit{gain loss equaition} for the probabilities of each state $x$. The first term on the right hand side describes the gain of probability of state $x$ due to transitions from other states $x'$, whereas the second term on the right hand side describes the loss of probability of state $x$ due to transitions to other states.\\
For a discrete state space the integral on the right hand side is replaced by the sum over all possible states and the master equation has the form of a system of coupled ordinary differential equations:
\begin{equation}
    \boxed{\frac{{\rm d} P_n(t)}{{\rm d} t} = \sum_{n'} W_{n n'}P_{n'}(t) - W_{n'n}P_{n}(t).}
    \label{discrete_space_master_equation}
\end{equation}
Or in a more compact form with the following \emph{transition rate matrix} $\mathbb{W}$:
\begin{equation}
    \mathbb{W}_{n n'} = W_{n n'} - \delta_{n n'}\sum\limits_{m} W_{m n}
    \label{transition_rate_matrix}
\end{equation}
resulting in 
\begin{equation}
    \frac{{\rm d} P_n(t)}{{\rm d} t} = \sum_{n'} \mathbb{W}_{n n'}P_{n'}(t).
    \label{ME3}
\end{equation}
The transition rate matrix satisfies the following conditions:
\begin{align*}
    &0 \le \mathbb{W}_{n,n'} \quad \mbox{ for all } n \ne n', \\
    &0 \le -\mathbb{W}_{n,n} \le \infty, \\
    &\sum_{n'} \mathbb{W}_{n,n'} = 0.
    \label{Transitions_rate_matrix}
\end{align*}
In general it is not symmetric and can thus not be diagonalized. \\
From equation \eqref{discrete_space_master_equation} one immediately sees that for a steady state solution the loss of probability from one state is compensated by the gain of probability from other states:\\
\begin{equation}
    \sum\limits_{n'} W_{n n'}P_{n'} = \sum\limits_{n'}W_{n' n} P_n.
    \label{equilibrium}
\end{equation}
For stationary time reversible Markov processes this criterion can even be tightened to a property called \textit{detailed equilibrium}.
This property requires, that the total exchange of probability between two states to each other must be equal, i.e.
\begin{equation}
   \boxed{ W_{n n'}p_{n'} = W_{n'n}p_{n}.}
    \label{detailed_balance}
\end{equation}
It can be proven to be true for a wide range of physical and chemical processes \cite{Boltzmann1872,Einstein1917,Wegscheider1911} and is also closely related to the Onsager reciprocal relations \cite{Onsager1931,Wigner1954}.
It further implies a certain symmetry of the transition rate matrix that can be used to show that for this class of matrices it is possible to find a symmetric representation such that it can be diagonalized via a suitable orthogonal transformation\cite{Oppenheim1977}.
\section{Composite Markov Processes}
\label{Multivariate_Markov_Processes}
It is straight forward to continue to Markov processes whose sample spaces are a direct products of continuous and discrete variables, i.e. $\Omega = \mathbb{R}^{3} \times [1,\cdots, N]$. The Chapman Kolmogorov equation then reads
\begin{equation}
    P(\vec{x}_3,n_3,t_3|\vec{x}_1,n_1,t_1) = \sum_{n_2} \int P(\vec{x}_3,n_3,t_1|\vec{x}_2,n_2,t_2)P(\vec{x}_2,n_2,t_2|\vec{x}_1,n_1,t_1) {\rm d} \vec{x}_2.
    \label{MCK}
\end{equation}
In this case the variable $\vec{x}$ can be treated by means of the Kramers Moyal expansion as discussed in section \ref{The_Fokker_Planck_Equation} whereas the variable $n$ can be treated by the approach of the Master equation as described in section \ref{The_Master_Equation}. Assuming that the driving process for the evolution of the continuous variable is Gaussian and that the Kramers Moyal expansion in moments of its transition probabilities can therefore be truncated after the second term one can derive the following expression:
\begin{align}
    \frac{\partial}{\partial t } P(\vec{x},n,t) =   &- \vec{ \nabla } \left[\vec{K}^{(1)}(\vec{x},n,t)P(\vec{x},n,t) \right] + \frac{1}{2}\vec{\nabla}^{2}\left[ K^{(2)}(\vec{x},n,t)P(\vec{x},n,t) \right] \nonumber \\
                                                    &+ \sum_{n'} \left\{ W_{nn'}P(\vec{x},n',t) - W_{n'n}P(\vec{x},n,t)\right\}.
    \label{composite_mp}
\end{align}
This gives the time evolution of the composite Markov process. If the underlying process for the continuous variable is again Brownian motion in an external potential then it follows from \eqref{BD_kinetic_coefficients} that the previous expression can be refined to
\begin{align}
    \frac{\partial}{\partial t } P(\vec{x},n,t) =   &- \vec{ \nabla } \left[\frac{1}{\gamma}\vec{f}(\vec{x},n,t)P(\vec{x},n,t) \right] +\vec{\nabla}^{2}\left[ D(\vec{x},n)P(\vec{x},n,t) \right] \nonumber \\
                                                    &+ \sum_{n'} \left\{ W_{nn'}P(\vec{x},n',t) - W_{n'n}P(\vec{x},n,t)\right\}.
    \label{fpmeq1}
\end{align}
This can be interpreted as the motion of Brownian particles with different internal states such as spatial or electronic conformations or binding states to some surface. In each of these internal states they may therefore have different friction and diffusion constants, be under the influence of different forces and may even be subject to different boundary conditions. \\
It is convenient to write this equation in a vector notation for the variable $n$. Then $\vect{p}$ indicates the vector $(p(0),p(1),\cdots,p(N))^{T} \in[0,1]^N$ (the variables $\vect{x}$ and $t$ have been omitted). \\ 
Also the diagonal Fokker-Planck operator for the drift and diffusion terms is introduced as
\begin{equation}
    \mathbb{F} = {\rm diag} \left[- \vec{ \nabla } \frac{1}{\gamma}\vec{f}(\vec{x},n,t) + \vec{\nabla}^{2} D(\vec{x},n) \right]
    \label{fpo}
\end{equation}
and the transition rates are again written in terms of $\mathbb{W}$ as introduced in equation \eqref{transition_rate_matrix} such that equation \eqref{fpmeq1} has the following more compact form
\begin{equation}
    \boxed{\frac{\partial}{\partial t} \vect{p}(\vec{x},t) = \left\{ \mathbb{F} + \mathbb{W} \right\} \vect{p}(\vec{x},t).}
    \label{fpmeq2}
\end{equation}
This kind of description has been used to explain various transport properties for instance on polymer chains or on reactive surfaces \cite{Friedman1968,Caceres1990}.
\section{The Smoluchowski Reaction Rate}
\label{K_s}
For educational reasons and since it will be referred to later, this section gives the solution to the original Smoluchowski problem. \\
The problem, which is essentially the coagulation of gold particles in solution, was outlined by Marian von Smoluchowski in a series of three talks on diffusion given in 1916 \cite{Smoluchowski1916} and finally published in 1917 \cite{Smoluchowski1917a}. It involves a perfect spherical sink of radius $R_s$ embedded in an initially homogeneous distribution of Brownian particles $\rho(\vec{r},t)$. The aim is to calculate the time dependent and stationary adsorption rate of particles into the sink. \\
\vspace{- .5 cm} \par

\begin{minipage}[t]{0.38 \textwidth}
    \begin{figure}[H]
        \caption{Sketch of the density profile $\rho(r)$ for freely diffusing Brownian particles in the vicinity of a perfectly adsorbing spherical sink of radius $R_s$ given according to equation \eqref{steady_state_density}. The density profile has a $1/r$ signature and saturates to the bulk density for $r \rightarrow \infty$.\label{fig:rho_smoluchowski}}
    \end{figure}
\end{minipage}\begin{minipage}[t]{0.62 \textwidth}
    \begin{figure}[H]
         \input{plots/Smoluchowski.pdf_tex}
    \end{figure}
\end{minipage}
\vspace{.3 cm}\\
Therefore one is looking for a time dependent solution to the corresponding Fokker-Planck Equation in terms of particle densities:
\begin{equation}
        \frac{\partial \rho(\vec{r},t)}{\partial t} = - \vec \nabla \left[ \vec f(\vec{r})\rho(\vec{r},t) \right] + D\vec \nabla ^2 \left[\rho(\vec{r},t) \right] 
    \label{FPE3}
\end{equation}
without external force $\vec{f}(\vec{r})=0$ and subject to the following boundary and initial conditions:
\begin{align}
    \rho(r > R_s, t = 0) &= \rho_o, \\
    \rho(r=R_s,t) &= 0, \\
    \lim_{r \rightarrow \infty} \rho(r, t) &= \rho_o.
    \label{BC}
\end{align}
The substitution $r \cdot \rho(r,t) = u(r,t)$ and the assumption of spherical symmetry reduce the equation to:
\begin{equation}
    \frac{\partial u(r,t)}{\partial t} = D \frac{\partial ^2 u(r,t)}{\partial r^2}.
    \label{Simplified FPE}
\end{equation}
This expression can now be transformed to an ordinary 2nd degree inhomogeneous differential equation in Laplace space:
\begin{align}
    \int_0^\infty e^{-st}\frac{\partial u(r,t)}{\partial t} \rm{d} t &= D \frac{\partial ^2 }{\partial r^2} \int_0^\infty e^{-st} u(r,t) \rm{d} t \\
    \left[e^{-st} u(r,t) \right]_0^\infty + s \int_0^\infty e^{-st} u(r,t) \rm{d} t &= D \frac{\partial^2}{\partial r^2} \tilde{u}(r,s)\\
    u(r,0) + s \tilde{u}(r,s) &= D \frac{\rm{d}^2}{\rm{d} r^2} \tilde{u}(r,s)
\end{align}
with transformed boundary conditions:
\begin{align}
    \tilde{u}(r=R_s,s) &= 0, \nonumber \\
    \lim\limits_{r\rightarrow \infty} \tilde{u}(r,s) &= \frac{r}{s}\rho_o.
    \label{transformed_BC}
\end{align}
According to the standard procedure one first solves the homogeneous equation. The use of the ansatz
\begin{equation}
    \tilde{u}_{h}(r,s) = e^{\lambda(s) \cdot r}
\end{equation}
results in the following characteristic polynomial:
\begin{equation}
    \lambda(s) ^2 - \frac{s}{D} = 0.
    \label{characteristic_polynomial}
\end{equation}
Calculation of the eigenvalues and linear combination of the independent solutions then lead to the following homogeneous solution:
\begin{equation}
    \tilde{u}_h(r,s) = C_1 e^{ - \sqrt{\frac{s}{D}} \cdot r } + C_2 e^{ \sqrt{\frac{s}{D}} \cdot r }.
    \label{u_h}
\end{equation}
The inhomogeneous solution is found using a polynomial ansatz of the form $\tilde{u}_i = C_3 r + C_4$:
\begin{align}
    s(C_3 r + C_4)  &= u(r,0)\nonumber\\
                    &= r \rho_o \nonumber\\
    \Rightarrow C_3 &= \frac{\rho_o}{s}, \nonumber\\
                C_4 &= 0
\end{align}
such that the entire solution reads:
\begin{align}
    \tilde{u}(r,t)&=\tilde{u}_{i}(r,t)+\tilde{u}_{h}(r,t) \nonumber \\
    &= \tilde{u}_h(r,s) = C_1 e^{ - \sqrt{\frac{s}{D}} \cdot r } + C_2 e^{ \sqrt{\frac{s}{D}} \cdot r } + \rho_o\frac{r}{s} .
\end{align}
This solution has to be fitted to the boundary conditions \eqref{transformed_BC} such that it reduces to:
\begin{equation}
    \tilde{u}(r,s) = \rho_o \left( \frac{r}{s} - \frac{R_s}{s} e^{ \sqrt{\frac{s}{D}}(R_s - r) } \right) .
\end{equation}
The inverse Laplace transform
\begin{align}
    u(r,t)  &= \frac{1}{2 \pi i} \int\limits_{\gamma - i \infty}^{\gamma + i \infty}  e^{st} \tilde{u}(r,s){\rm d}t \\
    &= \frac{\rho_o}{ 2 \pi i} \left\{  \int\limits_{\gamma - i \infty}^{\gamma + i \infty}e^{st} \frac{r}{s}   {\rm d}t -  \int\limits_{\gamma - i \infty}^{\gamma + i \infty}  e^{st} \frac{R_s}{s} e^{ \sqrt{\frac{s}{D}}(R_s - r) }  {\rm d}t \right\}
    \label{inverse laplace}
\end{align}
is done using the residue theorem for the first integral:
\begin{align}
    \oint_{ \gamma } {\rm d}z f(z) &= 2 \pi i \sum_{k = 1}^{n}I(\gamma, a_k) {\rm Res}(f,a_k) \\
    { \rm Res}(f,y_o) &= \frac{1}{(m-1)!} \lim_{z\rightarrow z_o} \frac{{ \rm d} ^{m-1}}{{\rm d} z^{m-1}} \left[ (z - z_o)^{m}f(z) \right]
    \label{residue theorem}
\end{align}
where $\gamma$ is a positively oriented simple closed curve and $I(\gamma,a_k)=1$ if $a_k$ is in the interior of $\gamma$ and $0$ if not. For the second Integral, one uses the following identity:
\begin{equation}
    \mathcal{L}\left[ {\rm erfc\left( \frac{a}{2\sqrt{t}} \right)} \right] = \frac{1}{s}e^{a\sqrt{s}}
    \label{L(erfc)}
\end{equation}
resulting in the following time dependent solution for $u(r,t)$ respectively the particle density $\rho(r,t)$:
\begin{align}
    u(r,t) &= \rho_o \left\{ r - R_s {\rm erfc} \left( \frac{r - R_s}{\sqrt{4 D t}} \right) \right\}, \\
    \rho(r,t) &= \rho_o \left\{ 1 - \frac{R_s}{r}{\rm erfc} \left( \frac{r - R_s}{\sqrt{4Dt}} \right) \right\}.
    \label{u(r,t)}
\end{align}
In the limit $t \rightarrow \infty$ this results in the steady state density profile that is illustrated in figure \ref{fig:rho_smoluchowski}:
\begin{equation}
    \rho(r) =  \rho_o \left( 1 - \frac{R_s}{r} \right).
    \label{steady_state_density}
\end{equation}
The reaction rate can be defined as the total flux of particles through the boundary $\Omega$ of the sink:
\begin{equation}
    K = \int_\Omega \vec{J} {\rm d}\vec{A} .
    \label{reaction rate}
\end{equation}
Using the differential continuity equation:
\begin{align}
    \frac{\partial \rho(\vec{r},t)}{\partial t}&= \vec{\nabla} \vec{j}(\vec{r},t) \nonumber \\
    &= \vec{\nabla} \left\{ \rho(\vec{r},t) \nabla \vec{U}(\vec{r}) + D \vec{\nabla} \rho(\vec{r},t) \right\}
    \label{contiuity_equation}
\end{align}
and the spherical symmetry of the solution one can derive the time dependent reaction rate of the Brownian particles as the flux through the surface $\Omega$ of the spherical sink of radius $R_s$ as follows:
\begin{align}
    K(t) &= \int_\Omega \vec j(r,t) {\rm d} \vec A \nonumber \\
    &= \int_\Omega D  \vec{\nabla} \rho(\vec{r},t) {\rm d} \vec{A} \nonumber \\
    &= 4 \pi D R_s^2 \left. \vec{\nabla} \rho(\vec{r},t) \right|_{r = R_s} \nonumber \\
    &= 4 \pi D R_s \rho_o \left( 1 + \frac{R_s}{\sqrt{4Dt}} \right).
    \label{ideal reaction rate}
\end{align}
Again in the limit of $t \rightarrow \infty$ this results in the steady state adsorption rate:
\begin{equation}
    \boxed{K = 4 \pi D R_s \rho_o.}
    \label{steady state ideal rate}
\end{equation}
Intuitively, one might have expected that the rate would scale with the surface of the sink. However, it goes only linearly with its radius.
\section{The Debye Reaction Rate}
\label{The_Debye_Reaction_Rate}
The problem of diffusion controlled reaction rates as outlined in the previous chapter was extended to interaction between the substrate and the absorbing particles by Peter Debye in 1942 \cite{Debye1942}. He longed to describe reaction rates between charged particles in ionic solutions. \\
\vspace{- .5 cm} \par

\begin{minipage}[t]{0.38 \textwidth}
    \begin{figure}[H]
        \caption{Sketch of the density profile $\rho(r)$ for a spherical sink surrounded by a step shaped repulsive potential (indicated in red) of height $U = 3 K_B T$ between $a$ and $b$ as given by the Debye formula \eqref{rho_debye}. The density profile has a $1/r$ signature with jump discontinuities at the boundaries of the potential barrier. \label{fig:rho_debye}}
    \end{figure}
\end{minipage}\begin{minipage}[t]{0.62 \textwidth}
    \begin{figure}[H]
         \input{plots/Debye.pdf_tex}
    \end{figure}
\end{minipage}
\vspace{.3 cm}\\


In his paper he assumed two species of Brownian particles with different diffusion constants $D_{1}$ and $D_{2}$ in solution, with one of them only present in very dilute concentration. Also, different particles are pairwise independent amongst their own species but interact with particles from the other species. Therefore the particles of the dilute species can be regarded as fixed targets and their diffusive behaviour is taken into account by taking the effective diffusion constant of the ``moving'' species to be the sum of the diffusion constants of both species. 
\begin{equation}
    D_{eff} = D_{1} + D_{2}.
\end{equation}
This effective diffusion constant can depend on the distance of the particles and is therefore denoted $D(r)$.
Again, one assumes spherical symmetry and consequently the problem can be described by a Fokker-Planck equation in terms of the density of the ``moving'' particles $\rho(r,t)$:
\begin{equation}
    \frac{\partial \rho(r,t)}{\partial t} = \vec \nabla \left[ \frac{\rho(r,t)\vec \nabla U(r)}{\gamma} + D(r) \nabla \rho(r,t) \right].
    \label{fpe_debye}
\end{equation}
From this expression it is possible to obtain a steady state solution for the particle flux through the surface of the sink of radius $R_s$. \\
Therefore one omits the left hand side of the equation and integrates the right hand side from $R_s$ to an arbitrary $r$:
\begin{align}
    0 &= \int_{R_s}^{r} \vec \nabla \vec j(r) {\rm d} r\nonumber \\
    \vec j(R_s) &=  \rho(r)\frac{\vec \nabla U(r)}{\gamma} + D(r) \vec \nabla \rho(r),
\end{align}
where $\vec{j}$ is again the local particle flux (comp. equation \eqref{contiuity_equation}).\\ Now since the adsorption rate is given by the flux through the sink surface, i.e. the flux $\vec j (R_s)$ integrated over the surface of the, sink the rate $K$ is determined by the following equation:
\begin{equation}
    \frac{K}{4\pi D(r) r^{2}} = \rho(r)\frac{\rm d}{ {\rm d}r} \frac{U(r)}{K_B T} + \frac{\rm d}{ {\rm d}r} \rho(r).
    \label{K}
\end{equation}
This is again an inhomogeneous ordinary differential that can be solved in terms of a homogeneous and an inhomogeneous solution. The homogeneous equation:
\begin{equation}
    \frac{\rm d}{{\rm d} r} \rho(r) = -\rho(r) \frac{\rm d}{ {\rm d} r} \frac{U(r)}{K_B T}
    \label{homogeneous_equation}
\end{equation}
obviously has a solution of the form:
\begin{equation}
    \rho_h(r) = C \exp \left[ - \frac{U(r)}{K_B T} \right].
    \label{homogeneous_solution}
\end{equation}
This expression is then used to find the solution of the inhomogeneous equation by the method of variation of the constant. Therefore one takes the constant $C$ in \eqref{homogeneous_solution} to be $r$ dependent and substitute $\rho_h(r)$ in equation \eqref{K}:
\begin{align}
    \frac{K}{4 \pi D(r) r^2} &= C(r) \exp \left[ - \frac{U(r)}{K_B T} \right] \frac{\rm d}{ {\rm d} r} \frac{U(r)}{K_B T} + \frac{\rm d}{ {\rm d} r} \left( C(r) \exp \left[ - \frac{U(r)}{K_B T} \right] \right) \nonumber\\
    \frac{K}{ 4 \pi D(r) r^2} &= \exp \left[ -\frac{U(r)}{K_B T} \right] \frac{\rm d }{ {\rm d} r} C(r).
    \label{equation_C(r)}
\end{align}
Then one integrates this equation from $R_s$ to an arbitrary $r>R_s$:
\begin{equation}
    C(r) = C(R_s) + K\int_{R_s}^{r} \frac{\exp \left[ \frac{U(r')}{K_B T}\right]}{4 \pi D(r') r'^2} {\rm d} r'.
    \label{solution_C(r)}
\end{equation}
Using the boundary condition $\rho(R_s)=0$ one can set the integration constant $C(R_s) = 0$. The resulting expression for $C(r)$ is then plugged into equation \eqref{homogeneous_solution}):
\begin{equation}
    \rho(r) = K\exp \left[ -\frac{U(r)}{K_B T} \right] \int_{R_s}^{r} \frac{\exp \left[ \frac{U(r')}{K_B T}\right]}{4 \pi D(r') r'^2} {\rm d} r'.
    \label{rho_debye}
\end{equation}
This is the spacial dependent density profile of the ``moving'' particles in the potential of the fixed target particles. For step repulsive shaped potentials it is depicted in figure \ref{fig:rho_debye}.\\
The rate $K$ can then be calculated using the boundary condition $r \rightarrow \infty$, $\rho(r) \rightarrow \rho_o$ together with the assumption, that the interaction $U(r)$ only has a finite range, i.e. that it vanishes for $r \rightarrow \infty$:
\begin{equation}
    \rho_o = \lim_{r\rightarrow \infty} \rho(r) = K\int_{R_s}^{\infty} \frac{\exp \left[ \frac{U(r)}{K_B T}\right]}{4 \pi D(r) r^2} \rm d r
    \label{lim_rho_debye_infty}
\end{equation}
which is equivalent to 
\begin{equation}
    \boxed{K = \rho_o \left\{\int_{R_s}^{\infty} \frac{\exp \left[ \frac{U(r)}{K_B T}\right]}{ 4 \pi D(r)r^2} \rm d r \right\}^{-1}.}
    \label{K_Debye}
\end{equation}
In the case of $U(r) \equiv 0$ this simplifies to the result obtained by Smoluchowski (compare equation \eqref{steady state ideal rate}. \par
\section{Summary}
The previous section gave a brief introduction on Markov processes and Brownian motion as well as two historic example applications for diffusion controlled reaction rates with either no or constant interaction between reactants. \\
In the following section these concepts will be extended to interactions that are fluctuating in time. Therefore the formalism for the description of composite Markov processes described in section \ref{Multivariate_Markov_Processes} will be used to describe the combined process of the Brownian motion of the particles around a spherical sink that is encapsulated in a fluctuating potential barrier.

\newpage

