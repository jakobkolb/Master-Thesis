%\documentclass[preprint,groupeaddress]{revtex4}
\documentclass[twocolumn,superscriptaddress]{revtex4}


\usepackage{amsmath} 
\usepackage{amssymb} 
 \usepackage{amsfonts}

\usepackage{graphicx} 

\usepackage{array}
\usepackage{multirow}

\newcommand{\vect}[1]{\boldsymbol{\mathbf{#1}}}



\begin{document}

\bibliographystyle{apsrev4-1} 

\title{Diffusion-Controlled Reactions over Fluctuating Barriers} 

\author{Jakob J. Kolb}
\affiliation{Department of Physics, Humboldt Universit{\"a}t zu Berlin, Newtonstr. 15, 12489 Berlin, Germany, Germany}
\author{Stefano Angioletti-Uberti}
\affiliation{Department of Physics, Humboldt Universit{\"a}t zu Berlin, Newtonstr. 15, 12489 Berlin, Germany, Germany}
\author{Joachim Dzubiella}
%\thanks{To whom correspondence should be addressed. E-mail: joachim.dzubiella@helmholtz-berlin.de}
\affiliation{Department of Physics, Humboldt Universit{\"a}t zu Berlin, Newtonstr. 15, 12489 Berlin, Germany, Germany}
\affiliation{Soft Matter and Functional Materials, Helmholtz-Center Berlin, Hahn-Meitner Platz 1, 14109 Berlin, Germany}



\begin{abstract}
Recent studies on tunable nano-reactors with a thermosensitive polymer shell have shown curious effects in reaction rates
right at the polymer critical solution temperature.
The shell is presumably stochastically fluctuating between states with different permeability for the substrate.
To investigate this effect a simplified system of diffusing particles in the vicinity of a spherical sink shielded by a metastable potential barrier is investigated. We derive an implicit solution for the resulting Fokker-Planck equation to obtain the diffusion-controlled reaction rate and verify these results with Brownian dynamics computer simulations. The system shows resonant activation as previously seen with thermally activated escape over fluctuating barriers.


\end{abstract}

\maketitle

\section{Introduction}

\section{Methods} We study the problem of diffusion limited reaction rates in the Smoluchowski Debye sense \cite{Smoluchowski1917a, Debye1942} in the presence of a fluctuating boxcar shaped potential barrier. The barrier switches between different states according to a discrete time reversible Markov process $\eta(t)$. Therefore the system evolution follows the following stochastic differential equation
\begin{equation}
    \frac{{\rm d} \vec{r}}{{\rm d} t} = \vec{\nabla}\frac{1}{\gamma}f(r)\eta(t) + \sqrt{2D}\vec{\varepsilon}(t)
    \label{SDE}
\end{equation}
where $\varepsilon(t)$ is white Gaussian noise with time correlation $\left< \varepsilon(t) \varepsilon(t') \right> = \delta(t-t')$ and $\eta(t) \in [U_0, U_1,\cdots U_n]$ and $f(r) = \Theta(r-a)-\Theta(r-b)$ define the hight and shape of the potential barrier. The intervals in $r$ with constant potential namely $R_s \le r < a$, $a \le r < b$ and $b\le r$ will be referred to as $(I)$, $(II)$ and $(III)$ in the following.\\
For the Brownian particles it is assumed that their total density equals one at $r \rightarrow \infty$ and that they vanish as their trajectory crosses the boundary of the sink at $r = R_s$. It is therefore appropriate to normalize the PDF of the system not to unity but to the number of particles per volume. \\
Now, equivalently to the SDE the system can be described in terms of a combined reaction-diffusion equation for the particle density function $\rho_n(\vec{r},t)$ of the discrete variable $n$ of the potential and the continuous variable $\vec{r}$ of the overdamped particles
\begin{equation}
    \frac{\partial}{\partial t}\vect{\rho}(\vec{r},t) = \left\{ \mathbb{F} + \mathbb{W} \right\} \vect{\rho}(\vec{r},t)
    \label{mfpe}
\end{equation}
with $\mathbb{F}$ being the Fokker-Planck operator
\begin{equation}
    \mathbb{F} = {\rm  diag}\left[\vec{\nabla} \frac{1}{\gamma} \left(\vec{\nabla} U_n f(r)\right) + D\vec{\nabla}^{2} \right].
    \label{FPO}
\end{equation}
$\mathbb{W}$ is the transition rate matrix of the Markov process for the barrier switching. $\vect{\rho}(\vec{r},t)=(\rho_0(\vec{r},t),\cdots,\rho_n(\vec{r},t))^{T}$ denotes the vector of particle density functions related to each state of the potential barrier. Since the underlying Markov process of $\mathbb{W}$ is time reversible the transition rate matrix satisfies detailed balance.This also implies that the particle density vector at infinity is equal to the equilibrium distribution $\vect{\rho}^{(eq)}$ of $\mathbb{W}$. \\
Therefore it is possible to find a similarity transform $\mathbb{T}_{ij}=[\rho_i^{(eq)}]^{1/2}\delta_{i,j}$ such that the resulting $\mathbb{T}^{-1}\mathbb{W}\mathbb{T} = \mathbb{S}$ is symmetric \cite{Oppenheim1977}. This symmetric matrix can then be diagonalized by an orthogonal transformation $\mathbb{D}$ resulting in $\mathbb{D}^{\dagger}\mathbb{S}\mathbb{D} = - {\rm diag}[\lambda_n]$. It can be shown \cite{VanKampen1992} that $\lambda_{i>0}>0$ and $\lambda_0=0$ with corresponding eigenvector $\mathbb{D}_{0,i}=[\rho_i^{(eq)}]^{1/2}$.  \\
Now we can give a steady state solution to eq. \eqref{mfpe} in terms of eigenfunctions of $\mathbb{W}$
\begin{align}
    \label{solution}
    \tilde{\rho}_{1}^{(j)}(r) &= c_{1,1}^{(j)} + c_{1,2}^{(j)} \frac{1}{r} \\
    \tilde{\rho}_{n \ne 1}^{(j)}(r) &= c_{n,1}^{(j)}\frac{1}{r} \exp\left[-r\sqrt{\frac{\lambda_n}{D}}\right] + c_{n,2}^{(j)}\frac{1}{r} \exp\left[r\sqrt{\frac{\lambda_n}{D}}\right]  \nonumber
\end{align}
separately for the regions $(I)$, $(II)$ and $(III)$ exploiting the fact that the Fokker-Planck operator $\mathbb{F}$ is invariant under the transormations $\mathbb{T}$ and $\mathbb{D}$ for $r\ne a, b$. The coefficients $c^{(j)}_{n,k}$ have to be obtained from boundary and fitting conditions at $r=a,b$. \\
To derive these fitting conditions we use eq. \eqref{mfpe} which in steady state is equivalent to
\begin{equation*}
     \frac{1}{\gamma}\rho_n(r) \vec{\nabla} U_n(r) + D \vec{\nabla} \rho_n(r) = \frac{J_n(R_s)}{4 \pi} - \left\{ \mathbb{W} \int_{R_s}^{r} \vect{\rho}(r') {\rm d} r' \right\}_{n}
\end{equation*}
where $J_n$ denotes the flux of particles in state $n$ through the sink surface. This expression is then integrated over a small vicinity of size $\varepsilon$ including the jump discontinuity at $r = a$. Taking the limit of $\varepsilon \rightarrow 0$ results in 
\begin{equation}
    \vect{\rho}^{(I)}(a) = {\rm diag}\left[\exp\left\{\frac{U_n}{K_B T} \right\}\right]\vect{\rho}^{(II)}(a)
    \label{dens_fit1}
\end{equation}
where $\vect{\rho}^{(I)}(a)$ is the density profile for $r \le a$ and $\vect{\rho}^{(II)}(a)$ is the density profile for $ a<r\le b$ right at the jump discontinuity of the potential barrier (consequently $\vect{\rho}^{(III)}$ denotes the density profile at $b<r$). Analogous considerations lead to fitting conditions for the derivative of the density profile
\begin{equation}
    \vec{\nabla}\vect{\rho}^{(I)}(a) =\vec{\nabla}\vect{\rho}^{(II)}(a) 
    \label{dens_fit2}
\end{equation}
and to fitting conditions for the density profile and its derivative at the second jump discontinuity of the potential.
The coefficients $c_{n,i}^{(j)}$ in eq. \eqref{solution} are calculated by applying the inverse transform $\vect{\rho}(r) = \mathbb{D}^{\dagger}\mathbb{T}^{-1}\tilde{\vect{\rho}}(r)$ and solving the system of linear equations arising from the fit conditions in eq. \eqref{dens_fit1}, \eqref{dens_fit2} and boundary conditions at $r=R_s$ and $r \rightarrow \infty$. Finally, the diffusion controlled reaction rate over the fluctuating barrier is calculated from $\vect{\rho}^{(I)}$ as
\begin{equation}
    k = 4 \pi D R_s^{2}\sum_n \left. \frac{\partial \rho_n^{(I)}(r)}{\partial r} \right|_{R_s}
    \label{rate_konstant}
\end{equation}
\section{Results and Discussion}
We used the preceding methods to study a system with a two state potential barrier and symmetric transition rates.\\
Results:
\begin{itemize}
    \item Density profile
    \item Absorption rate vs Switching Rate
    \item Limiting cases -> Resonant activation
    \item Powerlaw for resonant switching rate
\end{itemize}
Szabo {\it et al.}~\cite{Szabo1982}
\begin{minipage}{0.5 \textwidth}
\begin{figure}
\includegraphics[width= .5 \textwidth]{density_profile.pdf}
\caption{Selected density profiles for $a=6$, $b=10$, $U_1/K_B T = 10$. Decay length is varied and equals A): $r_d = 20$, B): $r_d=2$, and C): $r_d=0.5$.}
\label{fig1}
\end{figure}
\end{minipage}


\section{Concluding Remarks}

\acknowledgments
The authors thank the Alexander von Humboldt (AvH) Foundation and the Deutsche Forschungsgemeinschaft (DFG) 
for financial support. 

\bibliography{PRL.bib}

\end{document}
