%\documentclass[preprint,groupeaddress]{revtex4}
\documentclass[twocolumn,superscriptaddress]{revtex4}


\usepackage{amsmath} 
\usepackage{amssymb} 
 \usepackage{amsfonts}

\usepackage{graphicx} 

\usepackage{array}
\usepackage{multirow}
\usepackage{color}
\usepackage{transparent}
\usepackage{float}

\newcommand{\vect}[1]{\boldsymbol{\mathbf{#1}}}



\begin{document}
\bibliographystyle{apsrev4-1} 

\title{Diffusion-Controlled Reactions over Fluctuating Barriers} 

\author{Jakob J. Kolb}
\affiliation{Department of Physics, Humboldt Universit{\"a}t zu Berlin, Newtonstr. 15, 12489 Berlin, Germany, Germany}
\author{Stefano Angioletti-Uberti}
\affiliation{Department of Physics, Humboldt Universit{\"a}t zu Berlin, Newtonstr. 15, 12489 Berlin, Germany, Germany}
\author{Joachim Dzubiella}
%\thanks{To whom correspondence should be addressed. E-mail: joachim.dzubiella@helmholtz-berlin.de}
\affiliation{Department of Physics, Humboldt Universit{\"a}t zu Berlin, Newtonstr. 15, 12489 Berlin, Germany, Germany}
\affiliation{Soft Matter and Functional Materials, Helmholtz-Center Berlin, Hahn-Meitner Platz 1, 14109 Berlin, Germany}



\begin{abstract}
Recent studies on tunable nano-reactors with a thermosensitive polymer shell have shown curious effects in reaction rates
right at the polymer critical solution temperature.
The shell is presumably stochastically fluctuating between states with different permeability for the substrate.
To investigate this effect a simplified system of diffusing particles in the vicinity of a spherical sink shielded by a metastable potential barrier is investigated. We derive an implicit solution for the resulting Fokker-Planck equation to obtain the diffusion-controlled reaction rate and verify these results with Brownian dynamics computer simulations. The system shows resonant activation as previously seen with thermally activated escape over fluctuating barriers.


\end{abstract}

\maketitle

\section{Introduction}


\begin{figure}[H]
    \input{plots/Skizze.eps_tex}
    \caption{Sketch of the system consisting of sink and fluctuating barrier. Sink radius is $R_s$, potential barrier boarders are $r=a$ and $r=b$. Barrier transition rates between states $U_0$ and $U_1$ are $\gamma_{01}, \gamma_{10}$.}
\label{fig0}
\end{figure}


\section{Methods} We study the problem of diffusion limited reaction rates in the Smoluchowski Debye sense \cite{Smoluchowski1917a, Debye1942} in the presence of a fluctuating boxcar shaped potential barrier. The barrier switches between different states according to a discrete time reversible Markov process $\eta(t)$. Therefore the system evolution follows the following stochastic differential equation
\begin{equation}
    \frac{{\rm d} \vec{r}}{{\rm d} t} = \vec{\nabla}\frac{1}{\gamma}f(r)\eta(t) + \sqrt{2D}\vec{\varepsilon}(t)
    \label{SDE}
\end{equation}
where $\varepsilon(t)$ is white Gaussian noise with time correlation $\left< \varepsilon(t) \varepsilon(t') \right> = \delta(t-t')$ and $\eta(t) \in [U_0, U_1,\cdots U_n]$ and $f(r) = \Theta(r-a)-\Theta(r-b)$ define the hight and shape of the potential barrier. The intervals in $r$ with constant potential namely $R_s \le r < a$, $a \le r < b$ and $b\le r$ will be referred to as $(I)$, $(II)$ and $(III)$ in the following.\\
For the Brownian particles it is assumed that their total density equals one at $r \rightarrow \infty$ and that they vanish as their trajectory crosses the boundary of the sink at $r = R_s$. It is therefore appropriate to normalize the PDF of the system not to unity but to the number of particles per volume. \\
Now, equivalently to the SDE the system can be described in terms of a combined reaction-diffusion equation for the particle density function $\rho_n(\vec{r},t)$ of the discrete variable $n$ of the potential and the continuous variable $\vec{r}$ of the overdamped particles
\begin{equation}
    \frac{\partial}{\partial t}\vect{\rho}(\vec{r},t) = \left\{ \mathbb{F} + \mathbb{W} \right\} \vect{\rho}(\vec{r},t)
    \label{mfpe}
\end{equation}
with $\mathbb{F}$ being the Fokker-Planck operator
\begin{equation}
    \mathbb{F} = {\rm  diag}\left[\vec{\nabla} \frac{1}{\gamma} \left(\vec{\nabla} U_n f(r)\right) + D\vec{\nabla}^{2} \right].
    \label{FPO}
\end{equation}
$\mathbb{W}$ is the transition rate matrix of the Markov process for the barrier switching. $\vect{\rho}(\vec{r},t)=(\rho_0(\vec{r},t),\cdots,\rho_n(\vec{r},t))^{T}$ denotes the vector of particle density functions related to each state of the potential barrier. Since the underlying Markov process of $\mathbb{W}$ is time reversible the transition rate matrix satisfies detailed balance.This also implies that the particle density vector at infinity is equal to the equilibrium distribution $\vect{\rho}^{(eq)}$ of $\mathbb{W}$. \\
Therefore it is possible to find a similarity transform $\mathbb{T}_{ij}=[\rho_i^{(eq)}]^{1/2}\delta_{i,j}$ such that the resulting $\mathbb{T}^{-1}\mathbb{W}\mathbb{T} = \mathbb{S}$ is symmetric \cite{Oppenheim1977}. This symmetric matrix can then be diagonalized by an orthogonal transformation $\mathbb{D}$ resulting in $\mathbb{D}^{\dagger}\mathbb{S}\mathbb{D} = - {\rm diag}[\lambda_n]$. It can be shown \cite{VanKampen1992} that $\lambda_{i>0}>0$ and $\lambda_0=0$ with corresponding eigenvector $\mathbb{D}_{0,i}=[\rho_i^{(eq)}]^{1/2}$.  \\
Now we can give a steady state solution to eq. \eqref{mfpe} in terms of eigenfunctions of $\mathbb{W}$
\begin{align}
    \label{solution}
    \tilde{\rho}_{1}^{(j)}(r) &= c_{1,1}^{(j)} + c_{1,2}^{(j)} \frac{1}{r} \\
    \tilde{\rho}_{n \ne 1}^{(j)}(r) &= c_{n,1}^{(j)}\frac{1}{r} \exp\left[-r\sqrt{\frac{\lambda_n}{D}}\right] + c_{n,2}^{(j)}\frac{1}{r} \exp\left[r\sqrt{\frac{\lambda_n}{D}}\right]  \nonumber
\end{align}
separately for the regions $(I)$, $(II)$ and $(III)$ exploiting the fact that the Fokker-Planck operator $\mathbb{F}$ is invariant under the transormations $\mathbb{T}$ and $\mathbb{D}$ for $r\ne a, b$. The coefficients $c^{(j)}_{n,k}$ have to be obtained from boundary and fitting conditions at $r=a,b$. From this solution it is visible that the spacial influence of the potential fluctuations decays with a certain \textit{decay length} equal to
\begin{equation}
    r_d = \left\{\sqrt{\frac{\lambda_m}{D}}\right\}^{-1}
    \label{decay_length}
\end{equation}
that only depends on the diffusion constant of the Brownian particles and the largest nonzero eigenvalue of the transition matrix $\lambda_m$.\\
To derive these fitting conditions we use eq. \eqref{mfpe} which in steady state is equivalent to
\begin{equation*}
     \frac{1}{\gamma}\rho_n(r) \vec{\nabla} U_n(r) + D \vec{\nabla} \rho_n(r) = \frac{J_n(R_s)}{4 \pi} - \left\{ \mathbb{W} \int_{R_s}^{r} \vect{\rho}(r') {\rm d} r' \right\}_{n}
\end{equation*}
where $J_n$ denotes the flux of particles in state $n$ through the sink surface. This expression is then integrated over a small vicinity of size $\varepsilon$ including the jump discontinuity at $r = a$. Taking the limit of $\varepsilon \rightarrow 0$ results in 
\begin{equation}
    \vect{\rho}^{(I)}(a) = {\rm diag}\left[\exp\left\{\frac{U_n}{K_B T} \right\}\right]\vect{\rho}^{(II)}(a)
    \label{dens_fit1}
\end{equation}
where $\vect{\rho}^{(I)}(a)$ is the density profile for $r \le a$ and $\vect{\rho}^{(II)}(a)$ is the density profile for $ a<r\le b$ right at the jump discontinuity of the potential barrier (consequently $\vect{\rho}^{(III)}$ denotes the density profile at $b<r$). Analogous considerations lead to fitting conditions for the derivative of the density profile
\begin{equation}
    \vec{\nabla}\vect{\rho}^{(I)}(a) =\vec{\nabla}\vect{\rho}^{(II)}(a) 
    \label{dens_fit2}
\end{equation}
and to fitting conditions for the density profile and its derivative at the second jump discontinuity of the potential.
The coefficients $c_{n,i}^{(j)}$ in eq. \eqref{solution} are calculated by applying the inverse transform $\vect{\rho}(r) = \mathbb{D}^{\dagger}\mathbb{T}^{-1}\tilde{\vect{\rho}}(r)$ and solving the system of linear equations arising from the fit conditions in eq. \eqref{dens_fit1}, \eqref{dens_fit2} and boundary conditions at $r=R_s$ and $r \rightarrow \infty$. Finally, the diffusion controlled reaction rate over the fluctuating barrier is calculated from $\vect{\rho}^{(I)}$ as
\begin{equation}
    k = 4 \pi D R_s^{2}\sum_n \left. \frac{\partial \rho_n^{(I)}(r)}{\partial r} \right|_{R_s}
    \label{rate_konstant}
\end{equation}
\section{Results and Discussion}
We consider the most simple model to illustrate some basic features of the model. Therefore we set $U_0$ to zero and vary only $U_1$. Furthermore, we take the transition rates between these states to be symmetric i.e. $\gamma_{01}=\gamma_{10}$.
For this simplified system we calculate the radial steady state density profiles resulting from the reverse transform of eqs. \eqref{solution} and the solution of the system of linear equations for the remaining coefficients.
\begin{figure}[H]
\includegraphics[width= .5 \textwidth]{plots/density_profiles.pdf}
\caption{Steady state density profiles $\rho_0, \rho_1$ for states $U_0, U_1$ of the barrier and mean density profile $\rho_m$ vs. distance from center of sink. All parameters but decay length are fixed: $a=6$, $b=10$, $U_1 = 10 K_B T, U_0 = 0$ are constant and decay length is A) $r_d = 20$, B) $r_d=2$, and C) $r_d=0.5$.}
\label{fig1}
\end{figure}
It becomes obvious, that for a certain value of the decay length comparable to the barrier dimensions the density inside the barrier is considerably higher that what is is for decay lengths far off.
\begin{figure}[H]
\includegraphics[width= .5 \textwidth]{plots/v_rates.pdf}
\caption{normalized absorption rate vs. decay length for attractive fluctuating barrier. \newline Parameters are $U_1 = -10 K_B T, U_0= 0, g = 1$ and $t=2,5,10$. Approximation for very fast and very slow barrier fluctuations are depicted by dashed blue and red lines respectively. State points from the density profiles are marked by black crosses.}
\label{fig3}
\end{figure}
We find that in the limit of fast barrier fluctuations i.e. $r_d \ll 1$ the absorption rate becomes equal to
\begin{align}
    K &= \frac{4 \pi D ab}{ab - \varepsilon(b-a)(1-\exp[U_1/2])}; \\
    \varepsilon &= \frac{2(1+\exp[U_1/2])}{3+exp[U_1]}
    \label{kliminf}
\end{align}
whereas the naive Debye rate for an average potential barrier would not have the $\varepsilon$ correction.
\begin{figure}[H]
\includegraphics[width= .5 \textwidth]{plots/diff_plot.pdf}
\caption{A) difference in limiting cases of absorption rate vs barrier hight $U_1$ for different $t$ where $g=1$. \newline B) difference in limiting cases of absorption rate vs. $t$ for various positive $U_1$ at fixed $g=1$.}
\label{fig4}
\end{figure}
Ok, this looks interesting but at least for the second plot I have no idea how to explain this behaviour whatsoever.
\begin{figure}[H]
\includegraphics[width= .5 \textwidth]{plots/resonant_decay_length.pdf}
\caption{Resonant decay length $r_d^{(res)}$ vs. system size $t$ for fixed barrier hight $U_1=10 K_B T$ and barrier gap to width ratio $g=1$. The resonant decay length obviously shows a power law dependency $r_d^{(res)}=C\cdot t^{\kappa}$, where the exponent $\kappa$ is only weakly depending on the gap to width ratio for $g \approx 1$.}
\label{fig5}
\end{figure}

\begin{figure}[H]
\includegraphics[width= .5 \textwidth]{plots/powerlaw_parameters.pdf}
\caption{Dependence of power law prefactor $C$ and exponent $\kappa$ on the barrier gap to width ratio $g$. }
\label{fig6}
\end{figure}



\section{Concluding Remarks}

\acknowledgments
The authors thank the Alexander von Humboldt (AvH) Foundation and the Deutsche Forschungsgemeinschaft (DFG) 
for financial support. 

\bibliography{PRL.bib}

\end{document}
