\documentclass{a0poster}
\usepackage{fancytikzposter}

\usetemplate{3}

\usepackage[margin=\margin cm, paperwidth=84.1cm, paperheight=118.9cm]{geometry}

%% changing the fonts
\usepackage{cmbright}
%\usepackage[default]{cantarell}
%\usepackage{avant}
%\usepackage[math]{iwona}
\usepackage[math]{kurier}
\usepackage[T1]{fontenc}


\title{Diffusion-Controlled Reactions over Fluctuating Barriers}

\author{Jakob J. Kolb}


\begin{document}
\ClearShipoutPicture
\AddToShipoutPicture{\BackgroundPicture}

\noindent
\begin{tikzpicture}
    \initializesizeandshifts
    \titleblock{50}{1}
    \blocknode{Motivation}{
        ***Some sort of an Abstract***
    }
    \blocknode{System description}{
        \begin{minipage}[t]{.5 \textwidth}
            ***System description and sketch***
            have to figure out, how to fix the scalling issue with \textbackslash input
        \end{minipage}\begin{minipage}[t]{.5 \textwidth}
            \begin{tikzfigure}[Caption]
                \input{plots/Skizze.eps_tex}
            \end{tikzfigure}
        \end{minipage}
        
    }
    \blocknode{Analytical Tricks and stuff}{
        Lots of Formulas and explanations
    }
    \startsecondcolumn
    \blocknode{Density profiles}{
        \begin{minipage}[t]{.6 \textwidth}
        \begin{tikzfigure}[Caption]
            \includegraphics[width = 0.9\textwidth]{plots/density_profiles.pdf}
        \end{tikzfigure}
        \end{minipage}\begin{minipage}[t]{.4 \textwidth}
            just some text explaining the figure
        \end{minipage}
    }
    \blocknode{Absorption Rates}{
        \begin{minipage}[t]{.6 \textwidth}
             \begin{tikzfigure}[Caption]
                \includegraphics[width = 0.9 \textwidth]{plots/rep_barrier.pdf}
            \end{tikzfigure}
            \begin{tikzfigure}[Caption]
                \includegraphics[width = 0.9 \textwidth]{plots/att_barrier.pdf}
            \end{tikzfigure}
        \end{minipage}\begin{minipage}[t]{.4 \textwidth}
            some more text explaining these figure and giving hints to resonant activation phenomena
        \end{minipage}


    }
\end{tikzpicture}
\end{document}
