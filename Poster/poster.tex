\documentclass{a0poster}
\usepackage{fancytikzposter}

\usetemplate{3}

\usepackage[margin=\margin cm, paperwidth=84.1cm, paperheight=118.9cm]{geometry}

%% changing the fonts
\usepackage{cmbright}
%\usepackage[default]{cantarell}
%\usepackage{avant}
%\usepackage[math]{iwona}
\usepackage[math]{kurier}
\usepackage[T1]{fontenc}
\setheaddrawingheight{15}
\usebackgroundtemplate{2}
\setblockfillcolor{white}
\setbackgrounddarkcolor{white}
\setbackgroundlightcolor{colortwo!60!white}
\title{Diffusion-Controlled Reactions over Fluctuating Barriers}

\author{Jakob J. Kolb}


\begin{document}
\ClearShipoutPicture
\AddToShipoutPicture{\BackgroundPicture}

\noindent
\begin{tikzpicture}
    \initializesizeandshifts
    \titleblock{50}{1}
    \blocknode{Motivation}{
        \begin{itemize}
            \item Recent studies on tunable nano-reactors with a thermosensitive polymer shell 
                have shown curious effects in reaction rates right at the polymer critical solution temperature.
            \item The shell is presumably stochastically fluctuating between states with different permeability for the substrate.
        \end{itemize}
To investigate this effect a simplified system of diffusing particles in the vicinity of a spherical sink shielded by a metastable potential barrier is investigated. We derive an implicit solution for the resulting Fokker-Planck equation to obtain the diffusion-controlled reaction rate and verify these results with Brownian dynamics computer simulations. The system shows resonant activation as previously seen with thermally activated escape over fluctuating barriers.


    }
    \blocknode{System description}{
        \begin{minipage}[t]{.5 \textwidth}
            ***System description and sketch***
            have to figure out, how to fix the scalling issue with \textbackslash input
        \end{minipage}\begin{minipage}[t]{.5 \textwidth}
            \begin{tikzfigure}[Caption]
                \setlength{\unitlength}{\textwidth}
                \input{plots/Skizze.eps_tex}
            \end{tikzfigure}
        \end{minipage}
        
    }
    \blocknode{Fokker-Planck Description}{
        The system can be described in terms of a reaction diffusion system of spatially 
        depending particle densities in different states.

    }
    \startsecondcolumn
    \blocknode{Density profiles}{
        \begin{minipage}[t]{.5 \textwidth}
        \begin{tikzfigure}[Caption]
            \includegraphics[width = 0.9\textwidth]{plots/density_profiles.pdf}
        \end{tikzfigure}
        \end{minipage}\begin{minipage}[t]{.4 \textwidth}
            Examples of density profiles for different decay length.
            \begin{itemize}
                \item give an expression for the decay length and explain according to the figure
                \item Compare to the normal Debye solution ?
                \item 
            \end{itemize}
        \end{minipage}
    }
    \blocknode{Absorption Rates}{
        \begin{minipage}[t]{.5 \textwidth}
           \begin{tikzfigure}[Caption]
                \includegraphics[width = 1 \textwidth]{plots/att_barrier.pdf}
            \end{tikzfigure}
        \end{minipage}\begin{minipage}[t]{.5 \textwidth}
              \begin{tikzfigure}[Caption]
                \includegraphics[width = 1 \textwidth]{plots/rep_barrier.pdf}
            \end{tikzfigure}
        \end{minipage}

            some more text explaining these figure and giving hints to resonant activation phenomena
        \begin{itemize}
            \item \textit{Breathing Barrier} in the repulsive case
            \item \textit{Pumping Barrier} in the attractive case
 
        \end{itemize}
        }
    \blocknode{Limit of $t \ll 1$.}{
        Give the expansion and pose the question about why the resonant effect breaks down in the locally one dimensional limit.
    }
\end{tikzpicture}
\end{document}
